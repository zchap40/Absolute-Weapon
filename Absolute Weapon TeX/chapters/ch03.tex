
\chapter[The Atomic Bomb and Soviet-American Relations]{The Atomic Bomb and Soviet-American Relations}

\vspace{-2pt}

\noindent{\normalsize \textbf{Arnold Wolfers}}

\vspace{39pt}

As the Second World War drew to a close and the eclipse of German and Japanese power became certain, two new and harassing problems began to throw their shadow over the international scene: one, the future relationship between the United States and the Soviet Union; the other, the atomic bomb. Together they aroused in a war weary world the horrifying thought that failure to cope with them properly might lead to a third world war, and an atomic war at that.

Whether there exists today any direct connection between the difficulties besetting American-Soviet understanding and the American possession of atomic power may be doubted. If the Soviet leaders are disturbed by the increase of American military strength caused by the bomb or if they have been rendered more suspicious of American intentions because Russia has been excluded from the secrets surrounding atomic production, they have, in public utterances at least, given little expression to their feelings. The troubles which beset the statesmen and diplomats of the two countries in the matter of settling disputes antedate the atomic bomb; if they have increased in recent times, the termination of hostilities against the common enemy offers sufficient explanation.

Even so, the possession - now by one, later in all probability by both of these two giant powers - of a weapon with the destructive power of the atomic bomb cannot but profoundly affect their relations. Modern technological advances in the field of aviation and of rocket weapons have drawn the United States and the Soviet Union into military propinquity; they are now in a position to strike at each other from their home bases. What such proximity can do to the relations between nations the history of Europe over the past centuries only too clearly reveals. It will require the utmost care on the part of both countries if mutual fear of atomic attack is not to develop in them those attitudes which so often in the past have destroyed friendship and confidence between the nations of Europe.

This is not a matter concerning the Russians and ourselves alone. Friendly relations between the two countries which hold a predominant position of power in the world today constitute a guarantee of peace for all nations; conflict between them acts as a signal for nations - and even for groups within nations - to begin lining up for another world war. Thus, as tension between them rises or falls, so will the fear which the atomic weapon has put into the hearts of men increase or decline. Wars between other powers, of course, remain a possibility. The mere suspicion, if it should ever arise, that the Germans or the Japanese were in possession of atomic weapons might dispel any thought of Soviet-American conflict; but as things stand today and so long as Germany and Japan are kept under control, it seems unlikely that the atomic weapon would play a role in hostilities in which the Soviet Union and the United States were not both involved. If this is correct, a discussion limited to Soviet-American relations will not give a grossly distorted picture of the effects of the new weapon on general world conditions, though it cannot do full justice to the role of other countries.

As these lines are written the United States possesses a monopoly of atomic power. Britain and Canada, while sharing the secret, are not producing the atom bomb, nor is any other country in a position to do so. How long it will take the Soviet Union or other countries to break the monopoly nobody can predict; but it is safe to assume that before long dual or multiple possession of the bomb will have become a reality.

Until that day comes, and if only for a passing moment of history, this country occupies a unique position among the nations - one, in fact, that has no parallel in history. If this country, due to its naval and air superiority, enjoyed an unusual degree of immunity from attack even before the atomic bomb was invented, the solitary possession of this all-powerful weapon has put the cities and production centers of the entire world, including the mighty Soviet Union, at the mercy of our peaceful intentions. There may never have been a time when great powers were so dependent upon one major country.

One might argue that, given the obvious peacefulness and war-weariness of the American people, the Soviet Union has no reason to fear our monopoly or to seek to match us in the production of the bomb. Some may even suspect that the Russians must be harboring hostile intentions toward this country if they are disturbed by the present situation. That would be unfair to the Soviet Union. She has lost some of the freedom of action on which all great powers insist. She cannot risk undertaking any steps which we might interpret as a violation of our national interests. If war broke out today, she would be defenseless. History offers no example of a powerful country resigning itself voluntarily to such inferiority. It should be taken for granted and should cause no suspicion or resentment that the Russians are bending their efforts toward breaking with a minimum of delay the present American monopoly. Given the position of the two countries in the world, it is safe to assume that the Soviet Union, unless forced by circumstances beyond her control, will not rest content until she has succeeded in matching our atomic power too.\footnote{The Moscow magazine \textit{New Times}, as quoted by the \textit{New York Times}, in discussing the atomic bomb on September 3, speaks of ``many other countries... who will work with redoubled energy to invent weapons as good or better.'' \textit{New York Times}, September 4, 1945.

Mr. Molotov speaking before the Moscow Soviet on November 6, 1945, said, ``We shall have atomic energy too, and. many other things.'' \textit{Information Bulletin}, Embassy of the U.S.S.R., November 27, 1945.} Once again parity may become the watchword of disarmament negotiations, only this time bearing on the atom bomb and Soviet-American relations rather than on the naval strength of Britain and the United States.

Since everything points to an early end of our monopoly, we have every reason to ask ourselves whether some significant use could not and should not be made of it while it lasts. The fact that this country is temporarily enjoying absolute security from atomic attack means little because no major conflagration was likely to occur so soon after the close of a world war anyway.

It is being asked whether the spectacular increase in our military power, occurring at the very time when ticklish postwar problems are being thrashed out between the Allies, should not be helping our diplomats to obtain results more nearly in line with American views and principles. The evidence so far indicates that the atomic bomb has exerted no such influence. Rather than being a suitable instrument through which to obtain concessions from the Russians, it may have been an impediment to our diplomacy. There are good reasons why this should be so. Current negotiations with the Soviet Union bear on matters which from the viewpoint of the American public are of secondary interest; they bear on ``far-away regions,'' to use the words Neville Chamberlain applied to Czechoslovakia. The United States will not attack Russia with atom bombs over such issues as democracy in Eastern Europe or ``autonomy movements'' in Asia, and the Soviet leaders know it. American and British statesmen, as a matter of fact, have assured the Russians that they do not have the remotest intention of using the bomb as a means of diplomatic pressure.\footnote{Foreign Secretary Ernest Bevin addressing the House of Commons stated, ``I have never once allowed myself to think that I could arrive at this or that decision because Britain was in possession of the atomic bomb, or whether she was not.'' \textit{New York Times}, November 8, 1945.

Secretary of State James F. Byrnes on November 16, 1945: ``The suggestion that we are using the atomic bomb as a diplomatic or military threat against any nation is not only untrue in fact but is a wholly unwarranted reflection upon the American Government and people.'' \textit{New York Times}, November 17, 1945.} In saying so, they are promising little. It may be praiseworthy of them not to want to swing the ``big stick'', but it would not be much of a stick if they did. All they could achieve would be to arouse resentment and to provoke the Russians to more vigorous resistance to their desires. The mere suspicion on the part of the Russians that the English-speaking statesmen might be counting on the ``persuasive'' influence of the bomb, despite contrary professions, may be hurting the pride of the Soviet leaders and making them less conciliatory. Whether our possession of the bomb has made the Soviet leaders more cautious in their policy toward certain regions, such as China, in which this country is known to regard itself as being particularly interested is, of course, impossible to tell. All one can say is that there is no evidence that the Russians would have acted more aggressively if we had not possessed the bomb.

If the monopoly cannot and should not be made to serve as an instrument of diplomatic pressure, must the idea of actually using the bomb as a military weapon also under all circumstances be ruled out? The fact is that some people in this country are wondering whether there might not be purposes which would justify, if not an atomic attack on the Soviet Union, then at least the threat of such an attack.\footnote{Mr. A. Sokoloff writing in the Moscow \textit{New Times} of November 18, 1945, says, ``The atomic bomb is a signal for reactionnaires all over the world to agitate for a new crusade against the Soviet Union.'' He attributes to these groups in the English-speaking countries a design to reduce the Soviet Union to the rank of a second-rate power through the use of the atom bomb.} If we should become involved in an atomic war after the monopoly has been lost, more people might ask themselves whether out of sentimentality, complacency or ignorance an opportunity, unique and never to recur, had not been lost.

To the credit of the American people it can be said that where the question has been raised at all the attention has centered on ways and means by which atomic bombing or the threat of such bombing could be made to serve the interests of mankind and of the peace of the world. A few isolated voices have been heard to suggest that we launch a preventive war against the Soviet Union for the sake of national security. Such opinions may be held by people who are so firmly convinced of the inevitability of a Soviet-American war that they would not shrink from the idea of striking now when perhaps for the last time American cities could hope to survive another war. However, the idea of a preventive war is so abhorrent to American feeling that no government in this country, to judge from the state of public opinion today, could hope to gain popular support for such an adventure. Only if there was growing fear that once in possession of the bomb the Soviet Union would seek expansion by force, might a sweeping change in public opinion become possible.

That does not answer the question of whether the use of the atomic bomb in the service of some great humanitarian crusade might not have a broader appeal. Here and there one finds people, not cynical nationalists but high-minded and idealistic internationalists, playing with the idea of such a crusade. They will argue that since ``world government now'' can alone prevent the suicide of civilization, it has become an objective worthy of the greatest sacrifices. As long as this country has the atomic monopoly it has the power, never before possessed by any nation, to break any resistance to the establishment of such a world government. If the Soviet Union should refuse to join, we would be justified, according to those who hold this view, in using atomic coercion against her. Why, they ask, if we felt entitled to destroy two Japanese cities for the sake of shortening a war, should it not be right to take similar action against the Russians if mankind can be saved in no other way from the greatest of all catastrophes?

One might brush off this type of argument simply by pointing out that the American people could never be persuaded to such a course or one might rule it out as being too immoral for serious consideration. However, it may be more important to demonstrate the futility of such a crusade even in terms of the objectives of its proponents. Surely nobody would dare to justify an attack on a nation with which we were at peace unless he believed that it would save the world from the deadly threat inherent in atomic power.

Let us then, for the sake of argument, assume that this country were to propose to the Soviet Union and the other nations the immediate establishment of a world government with a federal and democratic constitution and that the Russians were to refuse, possibly on the grounds that the Soviet regime and Soviet principles would be threatened by such a world authority. We would then have to proceed to threaten Russian cities with an atomic attack and be ready, if the Soviet government did not yield, to follow up our threat with actual atomic bombardment. The Russians, their cities being defenseless, might conceivably bow to our threat and join the world federation under duress. We could not, however, expect our threat to induce them to allow troops of foreign nations or foreign government agencies to take control of their territory and resources. What then would we have achieved? Even as a member of the world federation, the Soviet Union could resume her efforts to attain atomic power. Nothing but continued coercion or threats of coercion would stand between us and the catastrophe which we would have set out to render impossible.

An actual atomic attack on the Soviet Union - if one dare contemplate as ruthless a step as that - might appear to offer better chances for a permanent elimination of the danger of atomic war. If it led to a crushing defeat and consequent unconditional surrender of the Russians, victory would bring in its wake complete control of their territory and resources, a control similar to that which we now exercise over Japan and Germany. But would we and the nations which had associated themselves with us know what to do with the Soviet Union if we had her in our power? Would not the danger we were setting out to ban reappear in a more threatening form as soon as our occupation armies were withdrawn? The Germans have shown what a vengeful and embittered people will do if and when they are offered an opportunity to pay back the humiliation which they believe they have suffered. More recent experience has also shown how little the American people are prepared to undertake the task of prolonged military control; as a matter of fact, none but a fascist regime would want to train and indoctrinate tens of thousands of men for the purpose of holding down the revolt of a country of the size and potentialities of the Soviet Union.

The whole idea of an offensive use of the bomb during the period of our monopoly can therefore safely be laid aside as utterly impractical. Since there is also little danger of our having to use it defensively in the years ahead, it would seem as if our sole possession of the atomic weapons was not going to be of much service to us or the world. There may be another way, however, of putting the monopoly to use while it lasts. We are today in a position to give away what others regard as a great privilege. We can, if we desire, offer to end our monopoly. The question is whether something substantial for our security or the peace of the world could be gained by bargaining away the advantages which we now hold but must expect to lose in the near future anyhow. There could certainly be no moral objection to such a deal, since we would generously be seeking to eliminate the threat of atomic warfare.

The term ``bargaining away'' as applied here should not be understood to mean bilateral negotiations by which this country would make direct concessions to the Soviet Union. Such a procedure was ruled out when the problem of the atomic weapon was put into the hands of the United Nations Organization. Any ``bargaining away'' of American advantages, if it occurs, will take the form of the United States accepting international agreements arising from deliberations of the Security Council or, what is practically the same thing, the United Nations Commission on Atomic Energy Control.\footnote{See p. \pageref{V-UNSC} below.}

It is not necessary to discuss in detail here the advantages of such international procedure over bilateral Soviet-American negotiations. The last chapter will be devoted entirely to the services which can or cannot be derived from international efforts in respect to atomic power. They bear on Soviet-American relations in several ways. Quite obviously it would be more difficult to obtain the consent of this country to sacrifices made directly to the Soviet Union than to American contributions to the common peace efforts of the United Nations.

Furthermore, by approaching the problem of protection against atomic weapons through an international organization, countries other than the two major powers not only gain a chance of participation but an opportunity to help bring about agreement between the two most important members. Finally, it is hoped that both this country and the Soviet Union will make greater efforts to reach agreement if in so doing they can strengthen the UNO.

However, the choice of an international instead of a bilateral procedure of negotiation cannot do away with the underlying problem which is the distribution of atomic power between the United States and the Soviet Union. This country as the sole possessor of the bomb is alone in a position to make immediate sacrifices or contributions. The Soviet Union is today the one country among the United Nations from which we must expect early and independent atomic production. It is therefore the one country from which, if we are to make concessions, we must insist on obtaining reliable safeguards. Whatever international agreement may be negotiated within the framework of the UNO will, thus, in the beginning at least constitute in essence a Soviet-American agreement, reinforced by the participation of others. It goes without saying that any agreement on atomic power would have to take care of whatever dangers might arise from countries like Germany or Japan which are outside of the Organization.

Theoretically this country could have offered a far more sweeping contribution to the solution of the atomic problem than anything ever hinted at in the Truman-Attlee-King declaration and the subsequent Moscow resolution and could in return have asked for correspondingly sweeping contributions from the Russians. Specifically, our government might have declared that the United States was ready to scrap all existing stockpiles of atomic bombs as well as all the plants in which they were produced. In return it would have had to demand that all other members of the Organization, including the Soviet Union, commit themselves, under stringent international guarantees, never to undertake the production of atomic bombs. Here again one is tempted to forego further discussion on the grounds that the consent of the American people could never have been obtained for such a scheme; but that would seem to pin on the American people all the blame for defeating what might be the panacea for the ills of the atomic age. For this reason it is worthwhile, as in the case of the humanitarian crusade which we discussed earlier, to show that the idea is not merely utopian but unsuited to the purpose which it would be designed to serve.

No scrapping of American plants and stockpiles could return the world to the happier days of the pre-atomic age. The ``know-how'' and therefore the potential existence of atomic weapons is here to stay. By ridding itself of all atomic power the United States would expose itself to the danger that the Soviet Union or some other country might violate its commitments and emerge as the sole possessor of the bomb. At the same time this chance of attaining a monopoly might make the temptation to violate international agreements almost irresistible. As a matter of fact, it is unlikely that our disarmament would induce the Soviet Union to abstain from those activities which would give her the ``know-how'' and experience. Another objection to this scheme is worth mentioning. Efforts would no doubt be made to preserve the production of atomic power for peacetime uses; but it might prove technically impossible to do so while destroying the means of producing atomic weapons.

There is another way by which, in return for commitments and guarantees, we could offer to end our monopoly, although this one can safely be passed over with few comments. We might offer to distribute our stockpiles of atom bombs among the United Nations and specifically help the Russians and others to erect plants for the production of bombs. Yet nothing would be achieved by such procedure other than to hasten the advent of a situation which the Russians and possibly other nations expect to bring about at an early date anyway. Since the Soviet Union has little to fear from our monopoly while it lasts, we could not hope to obtain from her long-term commitments which she would not be equally ready to make after getting into atomic production without our help. By putting this deadly weapon into the hands of others we might help create an atmosphere of good will from which, however, we could expect no more than passing gains. To rule out this radical solution does not mean condemning as useless or impracticable the idea of gradually lifting the veil of secrecy which today surrounds the production of atomic energy and atomic weapons. The revelation of secrets will tend to shorten the duration of our monopoly, but it might constitute a reasonable and limited concession in return for which the Soviet Union might permit the UNO to start experimenting with inspection schemes suitable to future conditions of dual possession of the bomb.

No attention need be given to the idea voiced here and there that we disclose our secrets to the Russians in return for a promise on their part not to make use of them. It would be folly to expect them to make any such promise since if they did they would be condemning their country to permanent military inferiority. Britain is in a different position. Although in on the secret she may decide to forego the luxury of establishing plants of her own in the belief that she is sufficiently protected by our possession of the bomb.

It is being widely held that there is still another way by which our monopoly could be brought to an end. Instead of substituting for it either total atomic disarmament or multiple possession of the bomb, we could aim at what is being called the internationalization of atomic weapons. This would suggest a transfer of our atomic monopoly to the United Nations Organization. On closer scrutiny any scheme of UNO possession of the atomic weapons, however, turns out to be not a solution \emph{sui generis} but another form of either American monopolistic possession or of dual possession of the bomb. This can be demonstrated by an analysis of Mr. Stassen' s suggestion that all stockpiles of atomic bombs be handed over to an international police force and that further production be stopped.\footnote{Address of Harold E. Stassen delivered before the Academy of Political Science, November 8, 1945. \textit{New York Times}, November 9, 1945.}

Under the present United Nations Charter such a police force could operate and undertake atomic attacks only against lesser powers and only with the consent of the United States and the Soviet Union. If such action did take place, the Soviet Union would have gained little or nothing by the fact that the bombs had been transferred from American to international possession. She would be no worse off if the stockpiles remained where they are today and were dropped from American planes operating in the service of the UNO.

The situation would be quite different in the case of a Soviet-American war, the only contingency concerning atomic warfare which as far as one can see today need seriously concern the Russians and ourselves. Neither under the present Charter of the UNO nor for that matter under any charter conceivable today would the International Police Force be entitled to take action against either the United States or the Soviet Union. The main question therefore, is what would happen to the atomic bombs held by the units of the police force in case of such a war. According to Mr. Stassen's scheme they would be the only atomic weapons in existence at the time. The answer is clear. Whatever legal provisions or prohibitions had been enacted prior to such a war, both countries, acting under military necessity, would be forced to seek control of the bombs as soon as war between them appeared imminent. Failure to do so would expose a country to the disastrous consequences of an atomic monopoly in the hands of its opponents. It follows that as far as both the Soviet Union and this country are concerned everything would depend on the geographical location of the ``five different suitable bases'' among which, according to Mr. Stassen the International Police Force would distribute its stockpiles of bombs. If they were safely within our reach, the American monopoly for all practical purposes would have remained untouched. If, instead, they were so distributed that we and the Russians would have a chance of gaining control of equal shares, the situation would be one of dual possession similar to that which would have existed if we had given half of our stockpiles to the Soviet Union in the first place.

Similar considerations would apply to the plants which produce the bombs. If their ownership were transferred to the UNO, the effect on Soviet-American relations would depend entirely on the location of these plants. If they remained in this country, our monopoly, as far as any threat to the Soviet Union is concerned, would not have been touched. Or does anyone believe that in case of war we would fail to use plants which were within our reach? The establishment of an international police force and its equipment with atomic bombs may prove to be a worthwhile objective for many reasons; it cannot solve the problem which the atomic weapon has introduced into Soviet-American relations.

The discussion of the ``bargaining value'' of our atomic monopoly has led to negative conclusions. The monopoly has no value which would allow it to be exchanged for immediate and substantial guarantees against the future dangers of atomic power. We can no more end our monopoly for the good of mankind than we can use the atomic bomb for that purpose. The line which our government, together with others, has taken offers the only practical alternative. It consists - as a later chapter will show - in using for the preparation and negotiation of agreements the breathing spell which our sole possession of the bomb gives to the world. Such agreements would be designed to minimize the dangers inherent in a situation in which more than one country possesses atomic weapons. All the efforts now under way within the UNO are in the nature of such preparatory spade work. While they cannot prevent the advent of a condition of dual or multiple possession, they should, at least, allow this country to examine dispassionately its position on the day when its monopoly will end. Attempts to push beyond what may seem a modest goal or to try to lay obstacles in the way of Russian parity with the United States would disturb Soviet-American relations and thus increase the danger they were designed to eliminate.

Once the Soviet Union - and perhaps other countries - starts producing atomic bombs, thereby putting an end to our monopoly, a truly revolutionary change will have occurred in the military position of this country. While it may still prove capable of avoiding defeat, never again will it be able to fight a major war without being exposed to vast destruction. No international agreements however stringent will remove this threat entirely. With every day that passes we are moving gradually from a position of unusual safety to a kind of earthquake zone which will be rendered livable for our urban population only by the hope and confidence that the outbreak of another war will be prevented.

The change in the position of the Soviet Union will be considerable too, although it will be less spectacular. Possession of the bomb cannot return to her cities the security from annihilation which some of them at least enjoyed before our discovery of atomic weapons. It should, however, prove much of a relief to the Russians to gain the power of retaliation in kind and to feel, if for prestige only, that they had gone a long way toward matching our military power.

If it were certain that the U.S.A. and the U.S.S.R. would at all times act in a ``spirit of unanimity and accord'', as Stalin has called it, relations between the two countries would be little affected by the termination of our monopoly. In that case the two countries would have no reason to compare each other's military power, atomic or other. Russian atomic weapons would, if used at all, supplement our own and merely serve to make the threat of UNO sanctions against third countries, such as Germany or Japan, more effective. This is what people must have in mind when they speak of the Russians and ourselves agreeing to put our atomic power into the service of world peace. Unfortunately, the two peoples do not and cannot feel sure that accord between them will prevail at all times. Constant efforts will be required if the two countries are to view each other's possession of this lethal weapon with anything like a spirit of equanimity.

Russian atomic power is bound to have profound effects on American psychology. What they will be cannot be predicted with any degree of certainty since nothing like it has faced this country before. Possibly the change in outlook and sentiment will not occur immediately, particularly if in the light of friendly relations with the Soviet Union the threat should seem far-off. One need, however, only imagine the impression it would make on our urban population if a serious crisis in Soviet-American relations should be accompanied by the sudden realization that an atomic surprise attack was not beyond the realm of possibility.

It might be thought that some insight into the kind of reaction to expect could be gained from a study of present day Russian psychology; but aside from the fact that such a study would be almost impossible under existing conditions, differences between the two countries would make it of slight significance. Theoretically, Russia's situation today is more dangerous than ours will be later. If the United States at this time were contemplating an attack on her, she would have no way of threatening retaliation. Yet it would not be surprising to find that there is little alarm in the Soviet Union. Surely the Russians feel reasonably confident that we will not attack them and that they have it in their power to avoid a clash at least until our monopoly has been ended. Furthermore, with its strict control over all means of information, the Soviet government can prevent and may actually be preventing the Russian people from realizing the new threat to their lives and cities.

If one wishes to draw conclusions from historical precedent, the experience of Britain in the late thirties should prove far more revealing. Her situation then shows striking similarities with what ours will be in the future. At that time the British people awoke to the fact that Germany had created an air force capable of striking at the cities of England. As a result, intensified fear of war gave impetus to the desire to appease Hitler. One can easily see how serious it would be if the same kind of reaction should set in here and exercise similar effects on American foreign policy.

The two situations are not entirely alike, since we could be more confident if not of Russia's peaceful intentions then at least of our ability to deter her. It made some sense for the Germans to believe that Britain was incapable of retaliating effectively in kind; the Russians cannot hope to make their country immune to atomic counter-attack. It seems probable, nevertheless, that this country, so averse to war anyway, will show greater reluctance to take up arms against the Russians once they possess the means of destroying our cities.

This is particularly true since our disputes with the Soviet Union are likely to center around Russian claims or moves concerning regions far removed from the United States. Like Britain in 1938, this country might become hesitant to risk for the settlement of troubles in ``faraway places'' not merely war but the very existence of its urban populations. If American security and world peace should at any time require that the spread of Russian influence or control be checked in such regions, excessive American fear of the atom bomb might seriously interfere with our peace strategy. Those who would spread panic at the mere thought of atomic warfare must realize that they might undermine the influence for peace and world order which this country now possesses. The mere suspicion on the part of the nations of Europe and Asia that the United States had became intimidated by Russian atomic power and could therefore no longer be counted upon for protection might lead them to bow more willingly to Soviet demands. Nobody would want this country to assume unnecessary risks of destruction; but it would not serve peace if one of the major powers of the world were paralyzed by fear and thus diverted from the course which it would otherwise have pursued.

If it were asked why Russian foreign policy is not being equally weakened today when we alone have atomic bombs, the answer is that she has several advantages which we do not possess. We have already mentioned the fact that the Russian people may be far less aware of the danger. But even if they were, the Soviet system of government allows far less scope for the pressure of public opinion with the result that the apprehensions of the Russian people may exercise no marked influence on Soviet foreign policy. Furthermore, the international situation of the two countries differs in such a way that the question of whether to appease the United States may never arise in Moscow. The Soviet Union, as recent events have demonstrated, is far less satisfied with the existing \emph{status quo} than is the United States. If unilateral action to change the \emph{status quo} occurs in the future, it is far more likely to originate in the Soviet Union than here. As a consequence the choice between defending the \emph{status quo} or pursuing a policy of appeasement will, if it occurs at all, present itself to us rather than to the Russians.

Fortunately the experience of the thirties contains a warning not only to any would-be appeasers and defenders of the established order but equally to any country which might believe it could change the \emph{status quo} by force without thereby incurring the risk of war. Hitler deceived himself, with disastrous consequences to his country, when he assumed that British fear of bombardment and reluctance to become involved in a war over Central Europe would outlast any provocation. Even if the Soviet leaders should at some future date feel strongly about the need for further Russian expansion Nazi experience with the English-speaking countries coupled with Japanese experience at Hiroshima and Nagasaki could hardly fail to exercise on them a restraining or cautioning influence.

When speculating about the change of psychology which dual possession of the bomb may bring about, some hold the hopeful view that the two countries, together with the rest of the world, will be drawn closer together by the common danger. They believe that a sense of solidarity may develop in the face of the unprecedented threat which the atomic weapon represents to civilization. It would be rash to discard this possibility. The Russians and we, concerned about our cities and industries, might be led to combine in a vigorous common effort to bring atomic power under control. However, it would be a mistake to overlook the other possibility, if not probability, that our fear of Russian bombs and their fear of American bombs will prove more powerful than our common anxiety about the atomic bomb in general. If that should turn out to be the case, the new weapon will tend to strain the relations between the two countries rather than to associate them in a common enterprise.

Those who take this second and more pessimistic view incline toward the belief that Russia's possession of the bomb will unleash a dangerous and unbridled Soviet-American armament race which will further strain and poison relations between the two countries. Whether this is likely to happen depends to some extent on the meaning we give to the term ``unbridled armament race.'' If all it is supposed to indicate is a situation in which the Russians are influenced in their armament policy by the state of American military power and vice versa, then we are engaged in such an armament race already. Nobody could maintain that Russian efforts to produce atomic weapons - or a big navy for that matter - are dictated solely by anxieties regarding Germany, Japan, or even Great Britain. Similarly, our preparedness is obviously not being decided without consideration for our security from Russian attack. The policy of each country in regard to atomic power will certainly follow the sane line, without necessarily harming the relations between the two countries.

It would be a different matter if the U.S.A. and the U.S.S.R. were to become engaged in a competitive struggle for arms superiority of the kind that developed between the European powers in the years immediately preceding the two world wars. There is no reason why dual possession of the bomb should produce a situation so obviously fraught with danger. Arms races of that type have in the past been the result not of new and powerful weapons but of a deterioration of relations between nations which led them to expect an early outbreak of hostilities. If Soviet-American relations were ever allowed to degenerate to a state of enmity, an unbridled armament race would follow as a consequence.

Such a race would not be limited to a struggle for more and better atomic weapons, although that might become its most spectacular aspect. As a matter of fact, it was shown earlier\footnote{See above, pp. \pageref{I-SuperiorNos1}-\pageref{I-SuperiorNos2}.} that an atomic race after reaching a certain point offers relatively modest military advantages. Instead, major benefits might be found along such lines as the greater dispersion of targets or the alignment with more and stronger allies. While armaments of the kind which both countries are planning at present, and which merely take the power of the other country into consideration, might be held within limits through agreements on the limitation or reduction of armaments, it is to be feared that an unbridled armament race would eventually lead the two countries into a policy of evasion if not of open violation of any commitment which might stand in the way of their quest for superiority. Nothing shows more clearly how much the danger represented by dual possession of the bomb depends on the future nature of Soviet-American relations.

Even while the American monopoly lasts our statesmen must be planning to meet this danger along every possible line. It would be a grave error if a solution were expected from a single approach with neglect of others. The time may be short during which we can prepare and erect barricades of protection; but there are several ``lines of defense'' which we can start building simultaneously.

The first line is directly connected with Soviet-American relations. It consists in proper efforts on our part to settle our disputes with the Soviet Union peacefully and to avoid adding new ones. In this way only can we hope to remove the incentives to war as well as those fears of a Soviet-American war which are turning atomic power into a veritable nightmare. The importance of this approach to the problem cannot be exaggerated, though any attempt to discuss it here would transcend the limits of our subject. This much, however, should be said. The peaceful settlement of disputes is not a one-way affair. This country
can succeed only if the Soviet Union is equally eager to eradicate the danger of atomic war and is equally convinced that continued conflict with this country would eventually bring down the calamity of war upon ourselves and the world. A policy of one-sided concession, instead of bringing us nearer to our goal, might have the opposite effect. It might lead the Soviet leaders to believe that we would continue to retreat indefinitely and that further demands or even unilateral acts on their part would, therefore, not endanger the peace. It would be equally wrong to regard every concession to the Soviet Union as an act of appeasement or to interpret every Russian claim as evidence of an insatiable desire for expansion. That would close the door to all efforts at conciliation and at satisfaction of reasonable demands the Russians may make. Wise statesmanship will have to seek a mode of conduct which will neither tempt the Soviet government to overstep the limits we can in safety and decency concede nor provoke actions taken out of sheer resentment or suspicion of our intentions. As we turn to the consideration of other lines of defense it should be particularly emphasized that their usefulness may be nullified if they disturb Soviet-American relations.

The second line of defense is not strictly of a Soviet-American character. It consists of international agreements and control. We are already committed to this line; the UNO is embarked on efforts to eliminate - or to reduce - the dangers of atomic weapons. Whatever success is achieved in this respect will benefit this country and the Soviet Union as it will all other members of the Organization. If little is said about this aspect of our problem here, it is because the general treatment of the subject of international control in the last chapter will indicate what protection the two countries may expect from this line of defense.\footnote{See below Chapter V.} It should, however, be mentioned here once more that the success of the UNO must depend primarily upon the Russians and ourselves; the world is looking to Washington and Moscow with the hope that they will agree to international rules and machinery removing the dangers of dual possession of atomic power.

In view of what has been said about the first line of defense, it is worth repeating that attempts to establish international controls might defeat themselves if they led to new conflict between ourselves and the Soviet Union. One example will suffice to demonstrate what this implies. It may be true, theoretically, that the removal of the veto rights of the great powers would pave the way for more reliable safeguards against atomic attack. But the Soviet Union has good reasons for believing that the veto constitutes an essential element of her security. It makes it impossible for the rest of the world to conspire and ``gang up'' against her in a coalition disguised as a world organization. The Russians seem to fear nothing more than that. Therefore, if this country were to advocate the abolition of the veto rights which it accepted earlier as the basis for big power collaboration in an international organization, it would risk aggravating our relations with the Soviet Union most seriously. This would in turn mean undermining the first line of defense. Even worse would be the effect of any official move to scrap the UNO and to replace it by a world government. The Russians have shown themselves more suspicious of the agitation for world government, now under way here and in Great Britain, than of our atomic monopoly or our atomic secrets.

If it were safe to assume that international controls and friendly settlement of disputes would at all times succeed in preventing Soviet-American hostilities or the use of atomic weapons in the course of such hostilities, there would be no need for a third line of defense. There is, however, in the history of international relations little that could induce responsible governments to act on such an assumption. One might argue that it is better to put one's faith unconditionally in the first two lines of defense rather than to undermine them by a lack of confidence; but that would be more of a gamble than governments could dare undertake. The Russians, as a matter of fact, would not be making efforts to get into production of the bomb if they believed that Soviet-American friendship coupled with international agreements could offer them sufficient protection.

The third line of defense is of a military character. It consists in all the steps a country can take in order to deter another country from risking war or from attacking it with atomic weapons. If we should fail either to eliminate atomic weapons from the arsenals of national governments or to remove the incentives which might under certain conditions lead the Russians to risk war with us, our hopes for peace will rest on our ability to deter them from taking the fatal decision.

It should perhaps be added that a policy of determent in regard to Russia has nothing to do with any imputation of aggressive or warlike motives. The gravest danger to us lies in the fact that without proper precautions on our part the Soviet Union might some day stumble into a war with us. Misjudging the situation, the Russians might advance to a position from which it would be hard for them to retreat. They might decide to go through with the action they had started, believing that we would either not oppose them or, if we did, be incapable of doing them much harm. From their point of view the same danger would present itself in a different light. They would fear that if we did not regard the risks for us as being too great we might oppose by force action they were undertaking under the compulsion of vital necessities. Neither country has any reason to resent what the other may do to minimize the chances of an outbreak of Soviet-American hostilities which would be the greatest calamity imaginable for both of them.

In the atomic age the threat of retaliation in kind is probably the strongest single means of determent. Therefore, the preparation of such retaliation must necessarily occupy a decisive place in any over-all policy of protection against the atomic danger. Neither we nor the Russians can expect to feel even reasonably safe unless an atomic attack by one will be certain to unleash a devastating atomic counter-attack by the other. However, once we are living under the threat of atomic attack, even the most reliable preparations for retaliation in kind may not prove sufficient to give us a sense of security. We are too much aware of the risks which the Nazi dictator was willing to take to feel satisfied that the Soviet leaders would under all circumstances shrink from sacrificing their cities. We may be doing them an injustice; the fact remains that only recently dictatorially ruled and dissatisfied nations took up arms at the risk of immense sacrifices. They did so at a time when their rulers felt sure of ultimate victory and were willing to pay the price it required to attain it. We shall have far less ground for anxiety, therefore, if we can feel confident that the Russians will not expect victory to come from the sacrifice of their cities. Hitler might have gone to war even if he had not believed that Germany would escape wholesale destruction from the air; it is hard to believe that he could have overcome the opposition of his generals to a war in which they would have seen no chance of victory.

Obviously, if the Russians fear that we might attack them some day, they too will seek to deter us not merely by holding themselves ready for retaliation in kind but by depriving us of the hope of ultimate victory. Efforts by both countries along this same line, if equally successful, would bring about a situation in which a war ending in stalemate would appear most likely. Nothing could be less tempting to a government, provided it were in possession of its senses, than a war of mutual destruction ending in a stalemate. It would not be surprising, therefore, if a high degree of Soviet-American ``equality in deterring power'' would prove the best guarantee of peace and tend more than anything else to approximate the views and interests of the two countries. Successful efforts by both countries along the ``third line of defense'' might thus help to bolster the first and second lines which were discussed previously.

There are some who despair of our ability to deter the Russians. They take the view that once the Soviet Union succeeds in producing the bomb she will hold all the trumps. Others assume, on the contrary, that our head start in atomic production coupled with our general technological superiority guarantees us immunity from Russian atomic power. It should be evident that no intelligent and far-sighted American policy in regard to the Soviet Union and the atomic bomb, least of all an adequate military policy can be formulated unless some light can be thrown on this matter. Extreme views might lead either to a defeatist attitude little conducive to vigorous protective efforts or to a spirit of complacency, the unhappy results of which are sufficiently known.

The question of what chances the United States and the Soviet Union may have in the future of deterring each other, should that ever become necessary, can be answered only if we have some idea of what a war between them would be like. The risks of destruction and defeat which the two countries would face if they engaged in hostilities with each other depends on the character of the war. The outlook for determent will be brighter if these risks are extensive and apparent.

It is not a happy task to try to visualize a war, the outbreak of which would mark a tragedy exceeding in horror any that man has experienced. Some would have us abstain from attempting it lest we arouse the sleeping demons of war. Their apprehensions, however, are not justified by history. Of the many writers who have discussed the causes of the two world wars none has suggested that the Western Powers talked themselves into them or brought them about by an excess of early thought about their probable nature. The opposite is more likely true. Obviously any attempt to imagine such a future war, even in its roughest outlines, must at this time be highly speculative and tentative. The Jules Vernes of the atomic age may come to look foolish very quickly!

It needs few words to dispose of the idea that our present superiority in atomic production need give us marked advantages far into the period of dual possession of the bomb. In an earlier chapter it was pointed out that a stage may be reached by both countries beyond which the advantage of possessing larger stockpiles and better atomic weapons would decline rapidly.\footnote{See above pp. \pageref{I-SuperiorNos1}-\pageref{I-SuperiorNos2}; p. \pageref{II-Superior}.} This does not mean that in a protracted war our impressive and possibly lasting technical and industrial superiority would not pay high military dividends. The later discussion of the non-atomic aspects of a war in an atomic age should bring this out more clearly.

In respect to alliances there might be a tendency to overestimate the value of Russia's head start. The Soviet Union has allied herself with Britain and France as well as with some of her small neighbors; we have concluded no formal alliances. However, Russia's alliances, particularly those with France and Britain, are not directed against the United States. Instead, Britain and Canada, sharing our atomic secret, constitute a kind of military combination with the United States as far as atomic preparedness is concerned. As a matter of fact, both the Russians and we might find it difficult to induce other countries to participate in a Soviet-American war. Those that did would risk becoming targets of atomic attack. It is important in this connection to note that, because of the veto power of the Big Five, membership in the United Nations Organization has committed no country to participate in a war against either the Soviet Union or the United States, Only through specific military alliances could such commitments be obtained. It is far more difficult for this country, both constitutionally and traditionally, to conclude alliances than it is for the Soviet Union. Whether sympathies with our cause or national interest would in the end lead more countries to line up with us would depend on too many changing factors to be predictable.

The Russians, once they possess the bomb, have a number of unquestionable military advantages which go back to their form of government. The problem is whether they would suffice to elevate the Soviet Union above the level of risks which might deter her.

Only a dictatorial government has a chance of successfully launching a surprise attack on its opponent. Preparations for such action and the action itself could be undertaken by the Soviet government without prior public discussion or congressional debate. Much of the prevailing pessimism in this country can be traced to the idea that our cities will become constantly exposed to the threat of annihilating ``Pearl Harbors.'' Two things can be said to relieve this anxiety: The first, already mentioned, is the fact that no surprise attack on this country would allow Russian cities to escape devastating retaliation in kind unless our military leaders had been criminally negligent.\footnote{See above pp. \pageref{II-Retaliation1}-\pageref{II-Retaliation2}, p. \pageref{II-Retaliation3}.} There is no reason why democracy should make such negligence necessary. The second has to do with the character of the surprise attack itself.

If a surprise attack were to come out of a clear political sky, it would put even the most carefully planned preparations to a severe test. Past experience, however, does not suggest the likelihood of such an event. Even if a Russian government should ever feel tempted to imitate the Nazis or the Japanese, it must be remembered that the ``surprise attacks'' carried out by those two nations were preceded in every case by months if not years of tension and mounting portents of war. It is hard to believe that this country, fearing for the fate of its cities and urban population, would not use such periods of crisis to make its arrangements for retaliation immune to the initial atomic attack.

In this connection something needs to be said about the possibility of a Russian ``surprise attack by planted bombs'' which is creating considerable anxiety here. If it were an effective method of defeating this country, it would be one which a dictatorially ruled country and no other might decide to employ. However, as was stated earlier, it would be hard to believe that before the number of bombs was large such action undertaken by or for a foreign government would not be detected. What the reaction in this country would be, once the first bomb was discovered and particularly if Communist Russia were involved, is not hard to imagine. Not only would saboteurs have a bitter time thereafter but retaliation in kind, difficult though it might be, would not be out of the question.

In a more general way preparations for sabotage undertaken in a period of peace constitute a form of ``armament'' for which democracies like our own are little adapted. The fear of Russia's indulging in them, though it might be quite unjustified, would become strong in this country if Soviet-American relations were ever to become seriously strained. Nothing could do more to threaten our ability to retaliate in kind than ``fifth column'' activities directed toward putting out of action either our weapons or the men who service them. The fact that Nazi Germany did not succeed in carrying out large-scale sabotage measures in this country or believed it to be in her interest not to undertake them does not prove that the Soviet Union might not in case of war or as a prelude to such a war be able and prepared to incite serious disturbances over here. Communists and Communist sympathizers are passionately opposed to any action directed against the Soviet Union and seem always ready to assume that the responsibility for conflict lies on the side opposed to the Russians. It would be a sad consequence of the dual possession of atomic power if unreasoned fear of such sabotage should come to poison political and social relations in this country. One would hope that more confidence would be placed in efforts to convince all groups of the population that their country was preparing or undertaking defensive action only and that readiness for retaliation in kind was the only means by which the cities and the densely populated working class communities of this country could hope to escape annihilation. Internal security measures should be able to cope with the rest. Nothing would lead one to believe that this country could or would compete with the Soviet Union in the field of fifth column warfare.

The Russians can derive further benefit from their form of government and economic system when it comes to dispersing the targets of atomic attack. While there is some doubt whether our government could hope to do anything substantial about decentralization of our cities or production centers, the Soviet government, if it decided to do so, might be able to go to almost any length. How much it will actually undertake in this respect remains to be seen.

Thus it appears that in a number of respects the Soviet Union will be in a better position than we. Some were not mentioned, such as the greater facility with which a totalitarian regime can, if it wishes, evade international inspection schemes. None of these advantages, however, provide the Russians with any substantial guarantee of immunity to atomic attack unless we should fail to take the necessary measures in regard to retaliation in kind. If it is enough to instill the fear of such retaliation, our policy of determent can be made effective despite the handicaps under which we have been found to operate.

However, determent, as we said earlier, may require that to the threat of destruction be added the threat that despite all sacrifices victory would not be attained. We must seek to discover, therefore, what chances of victory the two countries could expect to have. A war under conditions of dual possession of atomic power could be won by the country which was more capable of accepting punishment; its opponent might collapse or surrender under the sheer impact of atomic attack. In that case the war might take on the character of a brief atomic blitz campaign. If instead neither party were to give up despite the horrors and losses inflicted by atomic weapons the war would be drawn out and call for non-atomic operations and the invasion of enemy territory. It is necessary to assess the winning chances of the two countries in respect to both types of war. If neither had reason to expect victory from a blitz campaign, the decisive deterring factor would be the lack of hope of winning a protracted war.

It is hardly necessary to inquire whether this country would dare attack Russia because it hoped to be more capable of standing destruction. The idea of the United States starting a Soviet-American war appears preposterous in itself. But aside from all other considerations, we have certainly been too much impressed by the way the Russians were able to take punishment in the last war to have any illusions in that respect. Even atomic bombardment could hardly exceed very much the damage which the Germans inflicted on the western and southern parts of the Soviet Union; yet the Russians fought on.

The chances of winning a war against this country by the use of atomic means alone might look more promising to the Russians. This country has had experience neither with air bombardment nor with the kind of guerrilla warfare by lightly armed and independent units which the Russians used last time and which might prove necessary again in a war in which the main production centers were undergoing destruction. It is also true that we showed ourselves more reluctant than the Russians to accept great losses of men - a fact easily explained, however, by our ability to spend the costs and time necessary to substitute machines for men.

While it is obviously impossible to predict what punishment we could take or what our fighting power would be after our major cities had been wiped off the map, one thing remains certain: there could be no more serious threat to our policy of determent than if we were to create the impression that we ``could not take it.'' The consequences of Hitler's failure to understand what the British could take are still fresh in our memory. Nothing in the last war suggests that the American people would shrink from any sacrifices which were necessary to achieve victory. One thing this country apparently ``could not take'' is the idea of accepting ultimate defeat. If anything needs to be emphasized for the sake of peace, it is this.

Assuming that neither country could expect to defeat the other by means of an atomic blitz campaign and the spectacular methods of surprise attack and sabotage which might accompany it, the chances of winning a protracted war with this country might decide what course the Soviet leaders would pursue. It seems hardly doubtful that the advantages which the Soviet Union was found to possess would lose much of their weight in a long war and that one advantage on our side might at least balance them. It consists in the more favorable geographical position of this country. When it comes to warding off invasion or to invading enemy territory, the insular position of this country would reassert itself in its old defensive glory. The Soviet Union would be severely handicapped if she attempted to breach the defenses of this country and sought to penetrate into American territory. Airborne invasion - possibly across the polar regions - or amphibious operations across the oceans are under no circumstances an easy enterprise. With her cities and production centers suffering atomic bombardment, the Soviet Union despite meticulous preparations should find it difficult to carry them to success. Her expectation of success would depend largely on the defensive counter-measures which we had undertaken. There would be little danger to us from invasion attempts if we were able to rescue a large part of our naval and air power from atomic destruction. Since non-atomic weapons would come to play a decisive role in all operations accompanying or following upon atomic attacks and counterattack, our general technical and industrial superiority, if it survived atomic bombardment, would add to our geographical advantages.

The land masses of the Soviet Union, with their extended boundaries, could hardly be made equally immune to external penetration. Our forces, even if reduced to light armament, should be able to strike at the Russian homeland. It does not follow that such invasion of Russian soil would bring certain or easy victory. History offers ample evidence that the contrary is more likely to be true. Our policy of determent, however, does not depend on whether we can defeat Russia; to be successful it need only prevent the Russians from expecting to defeat us.

Even if the Russians did not fear ultimate defeat of the kind Hitler suffered, similarities between the situation of their country in a war with us and that of Nazi Germany in the last war could hardly fail to impress itself on them. They too could expect to enjoy considerable advantages in respect to preparedness in the initial stages of a war with us. They would, however, risk finding the odds against them if they became engaged in a protracted war. The similarity would become even more striking if, as another land power with easy access to foreign territory, the Soviet Union planned to overrun some of the weaker countries which surround her. The result might again be that defense against invasion would become more difficult. As a matter of fact, the Russians might plan an atomic blitz campaign in which the time-consuming occupation of weaker countries would be unnecessary and constitute a wasteful diversion of effort. This, by the way, suggests that the possible consequences of atomic warfare on the weaker countries need to be carefully explored. A few tentative remarks may help to indicate the importance of the problem for the military calculations of this country and the Soviet Union.

The military situation of the lesser countries, at least if they possess no atomic power of their own, will certainly continue to be unenviable. If any of them should become involved in a Soviet-American atomic war, the survival not merely of their cities but of a major part of their population would come to depend on discussions of the two major belligerents which they could not hope to influence. If, for instance, the Soviet Union, with her easy access to some of these countries, decided to overrun them, they would become exposed to American atomic bombardment. Such occupation might appear to the Russians to offer military advantages if, in expectation of a long struggle, they hoped to divert some of our attacks to targets outside of their borders or believed they could win control of undamaged productive facilities while their own were being destroyed.

In view of these dangers, the prevailing opinion appears to be that the military position of lesser countries, precarious enough in the past, has now become desperate. Some go so far as to suggest that the weak countries of Europe and Asia night as well save the money they are spending on obsolete non-atomic weapons and, in case of a Soviet-American war, run for shelter by joining the side which would have the best chance of overrunning them first. This side would obviously be the Soviet Union. If this were the policy which we would have to expect these countries to pursue, the effects on our policy of determent of Russia would depend on how much military benefit the Soviet Union would hope to gain from the alignment with these weaker countries.

There are some reasons, however, why the weaker countries may discover their prospects of keeping out of a war between the two giant powers, or of defending themselves if attacked, more promising than before the atomic age. We have mentioned the fact that the Soviet Union might prefer to stake her military fortunes on an atomic blitz campaign in which the two major belligerents would fight across other countries without having to conquer them. Also there would be considerable inducement to spare the productive facilities of lesser countries in the hope that they might eventually be substituted for those destroyed at home. Finally, a great power, suffering heavily from enemy atomic bombardment, might ill afford to divert as much strength to the conquest of foreign territory as Germany was able to do in the early years of the last war or to risk engaging heavily armed forces at great distances when they would depend for their supply and reserves on home bases and communications which were open to total destruction. If this proved to be true, the defensive power of lesser countries would have become greater than it was in the Second World War and their non-atomic weapons would not have become obsolete. Also, as a consequence, the position of the Soviet Union in the heart of Europe would have lost some of its military advantages.

Those considerations, of course, apply only to a Soviet-American or a similar war in which two major powers, both in possession of atomic weapons, would face each other. The outcome of a war which the Soviet Union was fighting on one side and lesser powers without sufficient atomic weapons and without American aid on the other would be a foregone conclusion. Tho opponents of Russia would be in such a case in all probability have to capitulate even before the war had started. One need not wonder, therefore, if in the rimlands of Eurasia the old idea of a ``balance of power'' as a major protective device had lost none of its traditional popularity!

If some of the weaker nations should come to possess atomic weapons of their own, their position would, of course, be strengthened. They would become worthwhile allies for both the Soviet Union and ourselves. Tho stature of a country like France, who could throw her weight to one side or the other, would grow considerably. But whether the Russians or we would stand to gain by such a development would depend on many unpredictable factors.

The suggestion that the two major belligerents might hesitate to expend their atomic bombs on targets within the weaker countries does not imply that there would be a general tendency not to use atomic weapons at all. While the fear or certainty of retaliation would, as in the case of poison gas, serve as a potent deterrent, it would be dangerous to set too much hope on such abstention in the case of a Soviet-American war. The Russian prospects of winning such a war by the use of non-atomic weapons only were shown to be slim, particularly if we had maintained our naval and air supremacy. The Soviet Union would, therefore, almost inevitably pin her hopes on an atomic blitz campaign which by its terror and destruction might overwhelm us after all. Our best defense must remain our ability to discourage any Russian expectation of such a blitz victory.

Little comfort could be gained from this discussion of the ``third line of defense'' if all it had proved were that we could hope to ward off defeat at the hands of the Russians provided we were ready to fight on while our cities were being wiped off the map. But that is not the main conclusion. Rather has it appeared that a well-planned and comprehensive policy of determent aimed at preventing the Soviet Union from risking a war with this country offers appreciable chances of success.

Nobody would want to suggest that we content ourselves with the protection offered by such a policy. But if both countries by their respective military and psychological preparations establish a kind of ``equality of determent'' between them, agreement on measures of international control which permitted them to remain roughly on a par with each other should be able to follow.

The end of our monopoly when it comes will make our security and that of all countries which count on our protection far more precarious than it is today; but there is no reason for panic at the thought that once the Russians have the bomb we shall depend for the very existence of our civilization on the wise and successful pursuit of three major objectives of our foreign policy: on peaceful relations with the Soviet Union, on international controls of atomic power and, last but not least, on our ability to deter the Soviet Union from any action which would lead her into a war with us.
