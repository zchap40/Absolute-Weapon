
\booktitle{The Absolute Weapon}

\subtitle{\Huge{\textit{Atomic Power and World Order}}}

\AuAff{\LARGE{by FREDERICK S. DUNN}\\
\LARGE{\textsc{BERNARD BRODIE $\cdot$ ARNOLD WOLFERS}}\\
\LARGE{\textsc{PERCY E. CORBETT $\cdot$ WILLIAM T. R. FOX}}}

\AuAff{}
\AuAff{\LARGE{Edited by Bernard Brodie}}

%\AuAff{Bernard Brodie, editor\\
%\LARGE{Yale Institute of International Studies}\\
%\LARGE{Frederick S. Dunn, Director}}

\placedate{Yale Institute of International Studies\\
New Haven, Connecticut\\
February 15, 1946}

%% Print Half Title and Title Page:
\halftitlepage
\titlepage

%%%%%%%%%%%%%%%%%%%%%%%%%%%%%%%%%%%%%%%%%%%%%%%%%%%%%%%%%%%%%%%%
%% Copyright Page

%\begin{copyrightpage}{<provide-copyright-year>}
%Title, etc
%\end{copyrightpage}


\tableofcontents



\begin{contributors}

\name{Bernard Brodie,} Yale Institute of International Studies

\name{Arnold Wolfers,} Yale Institute of International Studies

\name{Percy E. Corbett,} Yale Institute of International Studies

\name{William T. R. Fox,} Yale Institute of International Studies

\vspace{20pt}

Introduction by,

\name{Frederick S. Dunn,} Yale Institute of International Studies, Director

\end{contributors}



%%%%%%%%%%%%%%%%%%%%%%%%%%%%%%%%%%%%%%%%%%%%%%%%%%%%%%%%%%%%%%%%
% Optional Acknowledgments:

\acknowledgments
Typeset in \LaTeX~by A. S. Sadek, September 2022.



\begin{introduction}
\addtocontents{toc}{\textit{Frederick S. Dunn}\par}{~}

\vspace{-25pt}

{\huge \textbf{The Common Problem}}

\vspace{40pt}

\noindent{\normalsize \textbf{Frederick S. Dunn}}\\

\vspace{-17pt}
\begin{quote}[poetry]
\raggedleft
``The common problem - yours, mine, everyone's -

Is not to fancy what were fair in life

Provided it could be; but, finding first

What may be, then find how to make it fair

Up to our means - a very different thing!"
\source{\small \textbf{Robert Browning, \textit{Bishop Blougram's Apology}}}
\end{quote}

\vspace{10pt}

Whatever else the successful explosion of the first atomic bomb at Alamagordo signified, it was a victory of the most startling and conclusive sort for scientific research. By a huge effort of combined action, the physical scientists and engineers had succeeded in compressing into a mere sliver of time perhaps several decades of work in applying the energy of the atom to military purposes.

But having achieved this miracle, the scientists themselves were not at all sure that mankind was the gainer by their desperate labors. At least some of them had ardently hoped that their research would prove nothing more than the impossibility of reaching the goal. On the surface of things, the capacity of atomic energy for mass destruction far exceeded any immediately realizable value in enhancing human comfort and welfare. Moreover, like all physical forces, it was morally indifferent and could just as easily serve evil purposes as good. Unless some means could be found for separating out and controlling its powers of annihilation, the scientists' most striking victory of all time threatened on balance to become the heaviest blow ever struck against humanity.

About one thing the physical scientists had no doubt whatever, and that was the surpassing urgency of the problem. They went to extraordinary lengths to stir up the public to a realization of the magnitude of the danger confronting the world. They resorted to extramundane terms to make the non-scientist see that the new physical force was really something different, that it was even a different kind of difference. If they showed perhaps too great a tendency to expect mechanical answers to the problem of how to control this new and terrifying force, that was understandable since they were accustomed to that kind of answer in their own field. But in their efforts to drive home the urgency of the problem, they were serving a high and important purpose.

The more perceptive members of the military profession were equally disturbed, although for slightly different reasons. Whatever value for peacetime uses atomic energy might have, it had been developed as a weapon of war, and its first shattering effects had been felt in that sphere. What bothered the generals and admirals most was the startling efficiency of this new weapon. It was so far ahead of the other weapons in destructive power as to threaten to reduce even the giants of yesterday to dwarf size. In fact to speak of it as just another weapon was highly misleading. It was a revolutionary development which altered the basic character of war itself.

In the pre-atomic days of the 1940's things been bad enough, but one did not have to contemplate very seriously the probable annihilation of both victor and vanquished. Now, even the strongest states were faced with the prospect that they might no longer be able, by their own strength, to save their cities from destruction. Not only might their regular rivals on the same level be equipped with powers of attack hundreds of times greater than before, but possibly some of the nations' lower down in the power scale might get hold of atomic weapons and alter the whole relationship of great and small states. It was becoming very hard to see how a tolerable war could be fought any more.

Unless atomic warfare could be limited, no single state, no matter how strong its military forces might be, could be at all certain to avoid being mortally wounded in a future war. There was not and very likely would not be a sure defense against atomic attack, or any reliable way of keeping bombs away from a nation's territory. A great power might, it is true, by building up to the limit of its strength, have a good chance of winning a war in the end, but what good was that if in the meantime the urban population of the nation had been wiped out? Even military men were beginning to think that perhaps it would be a good idea to look very carefully into the possibilities of restricting atomic warfare by international action.

In any case it was not the task of either the physical scientist or the military strategist to find means of subjecting the new force to effective control. That was clearly a political problem, to be undertaken by the experts in political relationships.

After a few early flights of fancy, most of the political analysts lapsed into a discreet silence on the subject. It was quickly apparent that they had been handed one of the toughest problems which the members of their guild had ever had to face. The profound significance of atomic energy as a physical force called for political thinking on a commensurate scale. Initial probings with the ordinary tools of political analysis brought disappointingly small results. Each sortie into some promising opening either ended up against a solid wall or led into another tangle of seemingly insoluble problems. No clue could be found to a simple formula which would offer repose to men's minds while opening up new vistas of unruffled prosperity. In fact there was reason to believe that nothing of the sort ever would be found and that the job was one of arduous and patient examination of a whole mosaic of related problems extending indefinitely into the future.

One was met right at the beginning with two dilemmas of really imposing dimensions. The first of these arises out of the nature of the procedures available for the common regulation of the actions of free nations. On the one hand, any scheme for international control of atomic warfare must be put into effect by \emph{voluntary} agreement. There is no supreme power to impose it from above. On the other hand, it seemed extremely improbable that states possessing bombs or the capacity to make them would voluntarily restrict their power to carry on atomic warfare merely on the promises of other states to do likewise. Because of the nature of the bomb, any state which broke its word and surreptitiously manufactured atomic weapons could put itself in a position to exert its will over all those who kept their pledge. The more states observed the agreement, the greater the reward to the transgressor.

The second dilemma arises out of the time element in the carrying on of atomic warfare. On the one hand, since no state by its own strength can be sure of staving off a bomb attack, there is a growing conviction that effective control of atomic warfare must come through international action. On the other hand, the speed of attack by bombs can be so great that there would not appear at first sight to be sufficient time for any mechanism of international collective action to operate successfully. Before the air age, one could have counted on a fairly long period of grace between the time when all aggressor's intentions became evident and the time when he could attack in full force. The development of air bombardment shortened this period considerably, and the coming of atomic warfare promises to reduce it almost to zero. If a nation suddenly threatened by atomic bomb attack has to wait while an international agency arrives at a decision as to what counter measures should be taken, the chances of saving its cities would seem to be very small indeed.

Both of these dilemmas are directly concerned with the procedures whereby nations arrive at means of regulating their actions with respect to each other. Both of them receive attention in the chapters that follow. At the present time it is only necessary to make some very general observations about the treaty mechanism and the kinds of strains it might be expected to bear when put to the task of controlling atomic warfare.

Current popular beliefs regarding the efficacy of treaties are prone to be both too optimistic and too pessimistic as to what can be accomplished by them. On the one hand, there is a tendency to believe that practically any international problem can be solved if only the nations concerned can be cajoled into signing a treaty. On the other hand, the spectacular failures of some treaties in the past have led to the widespread conviction that governments in general are very casual about their international obligations and will disregard them whenever they are inconvenient. It is not unusual to find both of these views being held by the same person.

Neither of them finds much support in practice. Those who believe that a treaty is the answer to everything overlook the dreary wasteland of ineffective agreements that have been drafted in disregard of the limits to the loads which the treaty mechanism can bear. Those who make light of treaty commitments in general seem to ignore the fact that the vast majority of such engagements are continuously, honestly, and regularly observed even under adverse conditions and at considerable inconvenience to the parties.

Another common belief is that treaties contain or can be made to contain, single, definite answers to all questions of concrete application, and that strains on treaty observance are merely questions of moral behavior. Treaty failures, in other words, are regarded as lapses in virtue, and it is assumed that the way to avoid them is to strengthen the moral fiber of nations.

It would be foolish to deny that over the years there have been plenty of cases of deliberate bad faith in the non-execution of treaties. The writers on international law have been sighing about it for centuries. Yet it is not helpful just to charge off to the fickleness of sovereigns the many treaty failures that have occurred, and stop there. Most of the time there are quite understandable reasons why treaties fail to work out as expected, and in numerous cases it would be difficult if not impossible to place moral responsibility for such failure.

A good many notorious cases of treaty violation have been concerned with treaties of peace imposed on vanquished nations after a war. Where such treaties place onerous conditions on the losers, as they almost always do, it can be safely predicted that they will be faithfully carried out only so long as the victors have both the power and the inclination to enforce them. Where these grow weak and observance slackens off, the erstwhile victors will certainly cry, ``bad faith" but the other side will see only a just recovery of their former position.

Treaties of alliance have had a decidedly spotty record. Since the possible effect of an alliance is to draw a third party into a war which is not of his doing, the strain on the treaty is very great unless \emph{both} allies feel at the time that they are equally threatened. It seems too much to expect that a nation which has no interest in the outcome of a war will risk its very life merely to fulfill a promise contained in a treaty of alliance. It may well do so if the risk of losing is not very great, but one should not expect this if the odds are clearly against victory.

Where conditions have changed radically and unexpectedly since a treaty was signed, a nation which suffers real injury by such change will on occasion refuse to be bound by its promises. While it is true that under international law the injured state is not justified in doing so without the acquiescence of the other side, nevertheless the absence of any disinterested method of enforcing treaty changes to accord filth changes in surrounding circumstances can cause great hardship and will sometimes induce the injured party to take things into its own hands. In these cases it usually happens that the nation opposing any change will raise aloft the banner of \emph{pacta sunt servanda} as the basic norm of all international relations, yet to the other side it will seem that insistence upon the letter of the treaty is merely black reaction dressed up in the white garments of morality.

Efforts to limit armaments by treaty have certainly not enjoyed a brilliant success. On the other hand, it cannot be said that they have uniformly failed. The more recent criticism leveled against the Washington Treaty for the Limitation of Naval Armament of 1922 was not that it was ineffective but that it was so largely observed. One lesson seems clear and that is that not much can be expected from attempts at limitation of armament which are not closely tied in with the international political pattern of the times or which go counter to the basic policies of any of the top-level powers. It is not so much the ingenuity displayed in working out the details of a disarmament scheme that matters as the way in which it accords with the prevailing balance in the relationships of the powers.

There are many reasons for treaty failure not directly connected with the subject of the treaty itself. Most of these arise out of difficulties of language and uncertainties of intention. Treaties deal with future contingent events. No matter how carefully they are drafted, there are always unforeseen situations arising in which the meaning of the treaty is in doubt. The surrounding circumstances are constantly changing, and every new appearance of an old situation has its degree of novelty. The language by which treaties are drafted is the language of common use, made up of words often heavily laden with ambiguity and possessing extensive twilight zones of murky meaning. The drafters of treaties spend long and dreary days and nights trying to forecast all possible contingencies, yet the ink is scarcely dry on the signatures when new and troublesome situations begin to appear. Each novel case raises a conflict over classification. Statesman White is quite certain that it goes into this verbal category while Statesman Black just as firmly insists that it goes into that one. The fact that each one's interpretation happens to accord with the interests of his own country does not remove the fact that both honestly believe they are right. So far as the dictionaries show, they are.

This fact is familiar enough in the performance of compacts between individuals, but usually there are ample procedures for arriving at a settlement of disputes in accordance with the commonly accepted values of the community. In the international society the procedures are rudimentary and normally cannot be invoked unless both parties, including the one which would gain more by having no decision, consent to the process. Furthermore, the body of universally accepted notions as to what justice requires in the performance of treaties is painfully small.

When one thinks of all the reasons why treaties may fail to fulfill their intended purposes, one may well wonder why nations continue to enter into them. It is said that the first known treaty was made about 3000 B.C. between the kings of Umma and Lagash in settlement of a boundary dispute. No one knows how many treaties have been entered into in the intervening 5000 years but it is undoubtedly a colossal figure. While the total has been liberally sprinkled with instances of bad faith and broken engagements, it is still true that the great majority have been carried out by the parties in good order and have served their respective purposes reasonably well.

Clearly there is nothing in this long experience which compels the conclusion that the treaty process is incapable of bearing the load which would be put upon it by an attempt to control atomic warfare by international action. Treaties are tools which will perform well under certain conditions and badly under others. If a favorable set of conditions can be coaxed into existence, there is no reason to despair of finding a treaty structure that will withstand the strains which are likely to occur.

It is true, nevertheless, that a limitation agreement would fall into the class of treaties which are subjected to the greatest strains, and which not infrequently give way under them. For one thing, the subject matter deals directly with the security of the state, and on such questions every state will, if it can, hold on to the final decision itself. That does not, of course, rule out the possibility of common action, since states are quite capable of appreciating the advantages of such action, but it does put an outside limit on the distance to which a state will go in achieving it.
\footnote
The greatest strain, of course, would come from the nature of the bomb itself, and the enormous advantage that would be gained by surreptitious violation. So great would be the temptation to evade the treaty that governments would be extremely reluctant to put much faith in it if it rested on nothing more than the reciprocal promises of other states. Before divesting themselves of such a great source of power, they would certainly require assurances that they would be safeguarded against attack by a state that had secretly violated its promises, This is the well-known ``safeguards" problem and it is probably the most difficult one which the atomic energy commission will have to face.

It is in fact a very old problem. The Greeks knew about it, and their system of hostages was in effect a means of assuring fulfillment of treaty terms beyond the mere promise of the signatories.\footnote{This custom continued down to fairly recent times, the last well-known case being that of the Treaty of Aix-la-Chapelle, October 18, 1748, which provided that two English lords were to be handed over to France until the restoration of Cape Breton Island and the English conquests in the East and West Indies. See Coleman Phillipson, \textit{Termination of War and Treaties of Peace}, London, 1916, p. 208.} A safeguard of almost equal antiquity was the oath. This was particularly prevalent in the Middle Ages when religious faith was strong and the spiritual supremacy of the Pope over all sovereigns was universally admitted. The conclusion of treaties was marked by religious ceremonies and the trucing of the oath, the potential violator being threatened with major excommunication. There is no doubt about the fact that this added considerable strength to the sense of obligation of the signatories. But eventually this safeguard lost its power, due partly to a diminution of faith, partly to the changed position of the state in reference to the Church, but perhaps chiefly to the fact that it was not really reliable since the person under oath might possibly be absolved from it.\footnote{See P. C. Borda, \textit{De l'Inex\'ecution des Trait\'es}, Paris, 1922, pp. 37-38.} Nevertheless, the custom has continued down to the present day of using terms of religious significance to give as much weight as possible to treaty obligations, for example, ``the sanctity of treaties", ``solemn covenants solemnly arrived at", ``sacred obligations", etc.

Other forms of safeguards used today are the occupation of territory, as in the case of the Rhineland after the First World War, the guarantee by third powers of the fulfillment of a treaty, and the pledging of certain sources of revenue for the execution of a treaty, as Venezuela did to the European powers in 1902. An interesting form of indirect safeguard is the general exchange of military and naval attaches as a method of removing fears of unfriendly war preparations in derogation of treaties of friendship.

The only one of the familiar safeguards which seems to offer any promise in the international control of atomic energy is that of inspection. If it were possible to back up a limitation agreement with a system of disinterested inspection operating on a world-wide basis, the parties to the agreement would have a way of continuously reassuring themselves that no preparations were under way within any state to evade the agreement. But if this were to be the only safeguard, it would have to be practically infallible, in fact as well as in appearance; otherwise the states living up to the treaty would be lulled into a sense of false security and the door opened to easy violation by a potential troublemaker. Furthermore, unless every state confidently believed in the infallibility of the inspection system, individual nations which had grown suspicious might feel impelled to resort to secret production of atomic weapons as a precautionary measure.

This type of safeguard has a precedent in the inspection system developed in connection with the international control of narcotics.\footnote{This is discussed later in Chapter V, pp. \pageref{V-narco1}-\pageref{V-narco2}.} While this scheme resulted in bringing to light a number of violations, it was by no means infallible, and was scarcely effective at all against violations condoned by national authorities.

Some scientists impressed by the great technical difficulties in the way of a really effective inspection system have taken a very gloomy view of the possibilities of such a safeguard. Others who are more impressed by the problems of concealing the large-scale operations involved in the production of atomic weapons are far less pessimistic. The information so far made available is not sufficient to enable the layman to reach a satisfactory conclusion on the question. Nevertheless one thing seems clear: no one has any doubt but that each state has the power to make certain of what is going on within its own borders in the production and use of fissionable materials. If that is true for every state, then it necessarily follows that global control is not impossible from a technical standpoint, since means could be found for making use of the various national systems as the basis for international control. But this is a political rather than a scientific problem. The members of the atomic energy commission may well find it worth their while to explore it thoroughly.

What all this comes down to is the following: There is no reason to believe that the treaty mechanism is inherently incapable of bearing the load which would be associated with the international control of atomic weapons. Nevertheless, this load would necessarily be very great indeed, and there is no likelihood that nations would willingly narrow their freedom of action in relation to atomic energy merely on the naked promise of other states to do likewise. The potential advantages to be gained by a successful evasion of such a treaty are apparently so stupendous that very powerful safeguards would have to be provided against possible violations. None of the ordinary types of safeguards seem strong enough to provide this assurance.

One possible way of meeting this problem would be to eliminate all existing atomic weapons, destroy all means of production and prohibit all future steps toward production. This idea has wide public support and is in fact set forth in the Truman-Attlee-King declaration and the Moscow resolution as one of the ultimate aims of the work of the atomic energy commission. But in moving in this direction, one is met by a third dilemma of imposing proportions. On the one hand, having no bombs in existence would seem to remove any opportunity to embark on an adventure in atomic warfare. On the other hand, if no bombs are in existence, then any state which successfully evades the agreement and produces bombs would have a complete monopoly of them. Under such conditions the opportunities for world dominance would be breath-taking. Hence we come to the paradox that the further we go by international agreement in the direction of eliminating bombs and installations, the stronger becomes the temptation to evade the agreement. The feeling of security which one imagines would come from a bombless world would seem to be a fleeting one.

This suggests that the basic problem is somewhat different from that of just getting rid of bombs. It is rather a question of how to reduce to the lowest possible minimum the potential advantages to be gained by a successful evasion of a limitation agreement. If the threat to security comes from the prize that is available to a violator of a treaty, then the sensible thing to do would be to take away the value of the prize. Obviously this would not be an easy thing to do, but one has at hand a new and powerful aid for accomplishing it and that is atomic energy itself.

It happens that the atomic bomb is one of the most persuasive deterrents to adventures in atomic warfare that could be devised. It is peculiarly well adapted to the technique of retaliation. One must assume that, so long as bombs exist at all, the states possessing them will hold themselves in readiness at all times for instant retaliation on the fullest possible scale in the event of an atomic attack. The result would be that any potential violator of a limitation agreement would have the terrifying contemplation that not only would he lose his cities immediately on starting an attack, but that his transportation and communication systems would doubtless be gone and his industrial capacity for producing the materials of war would be ruined. If in spite of all this he still succeeded in winning the war, he would find that he had conquered nothing but a blackened ruin. The prize for his violation of his agreement would be ashes!

Hence there does seem to be available a safeguard strong enough to act as a real deterrent against possible evasion of a limitation agreement. But it is powerful medicine and should not be the sole means of assuring the observance of the treaty. Some kind of inspection system would still be extremely helpful. And the first line of defense would always have to be the constant exercise of farsighted, conciliatory diplomacy in order to avoid the building up of tensions that might tempt nations to seek a solution through the use of force. Thus we come to the final paradox that while the best way to avoid atomic warfare is to get rid of war itself, the strongest present ally in the effort to get rid of war is the capacity to resort to atomic warfare at a moment's notice.

\noindent\hfil\rule{0.4\textwidth}{.4pt}\hfil

\vspace{4pt}

The development of the atomic bomb has wrought profound changes in three major fields: (1) in the military affairs of nations, (2) in their political relationships, and (3) in the organized international machinery for peace and security. Each one of these is dealt with in the following text and there is a final chapter on the problem of international control of atomic weapons. There are still large gaps in the information that is essential to arriving at satisfactory answers to specific questions. The authors of the following text are acutely aware of these gaps and are anxious not to claim anything more for their contributions than that they are preliminary essays in an exceedingly difficult and complex subject. But it is time for responsible scholars to speak out to the best of their ability and not wait until all the evidence is in on every question. Only through the hard work of many minds is it likely that the means shall be found to remove the threat of disaster now facing us, a threat the like of which has never been seen before in the history of this planet.

\end{introduction}
