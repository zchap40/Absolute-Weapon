
\chapter[Effect on International Organization]{Effect on International Organization}

\vspace{-2pt}

\noindent{\normalsize \textbf{Percy E. Corbett}}

\vspace{39pt}

The preceding chapters show clearly enough that from now on the security of nations will depend on the possibility of dissuading governments from using atomic weapons as instruments of national policy. The dissuasion may come from the establishment of such a balance in the possession of and ability to use these weapons that only the most foolhardy counsellor would advocate their use. Or, eventually, it may come from a supranational agency equipped with legal authority and the actual power to enforce its decisions. Such alternative methods of control are studied later. Our point for the moment is simply that in the presence of these new weapons nations cannot achieve security for and by themselves. Even a large superiority in stocks and in methods of reaching targets will provide nothing like a satisfying guarantee against devastating attack or crushing retaliation.

As the knowledge spreads that there is no longer any geographic remoteness which offers immunity, and that no nation in the world can, merely by accumulating offensive and defensive armaments, maintain its way of life and guarantee its physical security, the ancient and rooted obstacles to international organization are \emph{pari passu} losing their strength. The current attempt to work out through the United Nations a method of eliminating or at least regulating ``atomic weapons and all other major weapons adaptable to mass destruction" has met with no open resistance. In other words, the direct attack on this vast new problem via international organization has evoked something approaching universal approval. The remaining differences of opinion turn on the type and degree of international organization that will be necessary to handle the problem. Even more significant is the evidence of a growing conviction that all indirect means of avoiding war, particularly atomic war, must be worked to the utmost. Clearly the more frightful war becomes for victor and vanquished alike the more effort must be devoted to the peaceful settlement of disputes and to remedying conditions that make for war.

These trends in general thought were already manifest in the plans for collective security and for economic and social co-operation which culminated in the San Francisco Charter, and in the reception which those plans found the world over. They have been strengthened in the interval of reflection which has followed the first shocked reaction to the reality of atomic weapons.

One instantaneous effect of the bomb that fell on Hiroshima on August 6, 1945, was a revival of the federalist movement. Men who had previously thought of a world state as something too remote to be worth striving for, were converted overnight to the view that the race could not survive unless states gave up their sovereignty and merged in one universal union. There were even some whose attachment to national individuality and international variety had made them hostile to the whole notion of world government, but who now faced with the dread potentialities of the new weapon, proclaimed the sudden conviction that the peoples must unite or perish from the earth. A new clich\'e was added to our stereotyped vocabulary, namely, that the atom bomb had made an anachronism of the San Francisco Charter.

That there was ample excuse for intellectual and spiritual disturbance cannot be denied. There is, it is true, very little evidence to support the advocates of immediate world federation in their apparent belief that the atom bomb has frightened away all the obstacles to the consummation of their desires. In any event, terror is hardly the perfect basis for union. But one thing is clear. A powerful conviction is abroad in the world that, unless means can be devised to prevent the competitive national production of atomic weapons, the existing plans for collective security will be worthless.

It does not follow that the design so laboriously worked out at San Francisco is already archaic. The three governments that developed the bomb took a contrary view. They believed that the United Nations Organization was the very instrumentality through which the nearest approach to efficient control of atomic fission could be achieved. In the announcement issued from Washington on November 15, 1945, they expressed their belief that ``no system of safeguards that can be devised will of itself provide an effective guarantee against production of atomic weapons by a nation bent on aggression." Declaring that ``the only complete protection for the civilized world from the destructive use of scientific knowledge lies in the prevention of war", they went on to pin their hopes of lasting peace explicitly and firmly to the United Nations Organization and to ask that institution to devise ways and mans of insuring that atomic energy shall be used only for peaceful purposes. This declaration of faith was accompanied, however, by an admission that the authority of the Organization will need to be consolidated and extended.

Those who urge a super-state now will probably interpret this admission as a promise of rapid evolution in the United Nations towards world government. But extension does not necessarily mean anything more than the addition of a special instrumentality to assist in the control of atomic energy; while the appeal to consolidate can be read merely as a fresh injunction to faithful compliance with obligations under the existing Charter.

It is true that in England opposition and government alike have evinced new willingness to discuss the sacrifice of national sovereignty.\footnote{See speeches in the House of Commons by Mr. Eden on November 22, and by Mr. Bevin on November 23, 1945.} But there has been no official response from Washington to this overture; while from the Soviet Union - an indispensable partner in any project involving the merger of state sovereignty in supranational organization - the repercussions have been definitely negative.\footnote{The editorial in \textit{Pravda} dated December 2, 1945, reported on the following day in the \textit{New York Times}, is typical of Russian comment on the suggestion thrown out by Mr. Eden and Mr. Bevin.} Nor is it clear that obstacles will be thrown in the path towards world government solely by great powers. Far as the small states have gone in the subordination of their external autonomy to the United Nations Organization, some of them will object to closer union. Mr. Herbert Evatt, Australian Minister of External Affairs, in a speech made in New York on November 27, 1945, issued a \emph{caveat} which will probably be echoed by statesmen of other middle or small nations. World government, he is reported to have said, if it means some form of federal union, is ``impossible of acceptance. The plain fact is that the nations and peoples of the world are not yet prepared to surrender the rights of self-government in order to be governed by a central executive and a central legislature on which most of them would have a tiny and very insignificant representation."\footnote{\textit{New York Times}, November 28, 1945.}

The official response, then, to the challenge of the atom bomb, is not an inclination to scrap the San Francisco Charter and to substitute for it a federal world constitution, but rather to use the machinery already under construction for the solution of what is admittedly the greatest international problem of our time. The program announced at Washington by the American, British and Canadian governments was concurred in by the Soviet Union at the Conference of Foreign Ministers held at Moscow in December, 1945. With only the Philippines protesting the somewhat cavalier manner in which the General Assembly of the United Nations was being instructed by the great powers, that body, sinking any procedural pride in its desire for an effective control system, adopted on January 24, 1946, the formal resolution asked of it.

Ever since the Truman-Attlee-King announcement of November 15, 1945, the suggestion had been heard that any agency set up under the United Nations to deal with the subject of atomic energy should be appointed by and responsible to the General Assembly rather than the Security Council. A variety of arguments were put forward to support this contention. One was the universal interest not only in protection against atomic weapons but in the potential peaceful uses of the new source of energy. Another was the absence of a great-power veto in the Assembly. It appears to have been thought that a body with purely advisory powers, as the Assembly is, might set up and control an agency entrusted with the most critical of security problems.

What the announcement of November 15, 1945, contemplated was not a control agency itself but simply a commission to make recommendations on ways and means of preventing the use of atomic energy for other than peaceful purposes. Mr. Truman, in a press conference five days after the announcement, suggested that all nations should have a voice in selecting this commission, and that its members should be designated by the General Assembly.\footnote{\textit{New York Times}, November 21, 1945.} But it was only in the most formal way that this suggestion survived the Moscow meeting of the three Foreign Ministers. The dominant opinion there was apparently that even at the stage of mere proposals for subsequent adoption or rejection by the interested states, the Security Council should play the leading role.

So, while the Moscow Conference indeed arranged that the General Assembly should act as formal creator, it laid down the membership, functions and responsibility of the commission to be created. Membership is limited to the eleven states represented on the Security Council with the addition of Canada so long as Canada is not on the Council. In matters affecting security the Council is to issue directions to the commission and the commission is to be accountable to the Council. So jealously is the supremacy of the Council safeguarded, that all reports and recommendations are to be submitted by the commission to that body, which in its discretion may transmit them to the General Assembly, to other agencies, or to the members of the United Nations Organization.

The General Assembly's part in planning for the eventual control of atomic energy will thus be completely subject to the authority of the Security Council. And if such precautions are taken to insure the Council's control in the mere planning phase, it may be taken for granted that any administrative agency set up as a result of the planning will be completely subordinated to that body. The Council will delegate to the General Assembly or to agencies responsible to the Assembly only matters bearing exclusively on the peaceful uses of atomic energy. In view of the nature of the matter in hand, and of the division of functions under the San Francisco Charter, this policy is appropriate and even inevitable.

The present commission is not an agency to control atomic armaments. Its function is solely to devise a plan of control. That is likely to be a long task. It may conceivably end in failure. At the best, we probably have before us a fairly prolonged period in which all nations remain free to invent and produce - though not to use - any kind or quantity of atomic weapons within their several capacities.

What are the probable effects on international organization of the existence of atomic weapons in this indefinite period before a system of control can come into operation?

The United Nations Organization has become a reality. It is already at work trying to dispose without violence of a complex of knotty problems in world politics. All its members are legally bound to settle their international disputes by peaceful means and not to resort to the threat or use of force in any way inconsistent with the purposes of the United Nations.\footnote{See the Charter, Art. 2, paras. 3 and 4.} That would be a fairly good beginning even for an organization specifically designed to prevent the aggressive use of atomic weapons. It has the advantage of prohibiting all forms of force - something not to be overlooked in our present preoccupation with a single new form.

The prohibition is subject, however, to an exception. Article 51 lays down the principle that ``Nothing in the present Charter shall impair the inherent right of individual or collective self-defense if an armed attack occurs against a member of the United Nations, until the Security Council has taken the measures necessary to maintain international peace and security."

The limited nature of this exception should be carefully noted. It is available only in the case of armed \emph{attack} and only when and so long as the Security Council has failed to take adequate measures. Furthermore, as the remainder of the Article makes clear, action taken in alleged self-defense is subject to scrutiny by the Council. If the Council finds that such action was not self-defense within the limited meaning of the text, this finding would amount to a decision that the member had resorted to an illegal use of force. The member, unless one of the five enjoying the right of veto, would then be subject to such enforcement measures as the Council might decide to be necessary for the restoration of international peace and security. The legal difference between the five permanent members of the Council and other members of the United Nations Organization would hardly be matched by so great a difference in actual fact, since any given member would usually be able to count on the support of at least one of the five great powers. This would be particularly likely in cases of ``collective self-defense", which means joint defense under a regional or other limited arrangement. Most of such arrangements would involve one or another of the permanent members of the Security Council; and the permanent member's veto would normally be available to prevent any preventive or punitive action.

The ``inherent right" of self-defense will be no less precious in an age of atomic weapons than it has been in the past. It becomes doubtful, indeed, whether the limitation of the right to cases of ``armed attack" can be sustained if such weapons are available to an aggressor. Can a state, satisfied that another state is preparing to bombard its cities with atomic projectiles, and seeing no adequate preventive measures undertaken by the Security Council, be expected to wait until the first bombs have landed before taking steps to protect itself?

The question should, perhaps, be broadened. What measures can the Security Council take ``to maintain or restore international peace and security" once an attack with atomic weapons has been launched? Such devastation is likely to be wrought in the attack that the victim's need will be restoration from the ground up. Its security will have been shattered at the first blow. If so, the only protective measures that will make any sense must be measures to prevent attack. Unless, in other words, the Security Council has always at its command the means of preventing the aggressive use of atomic weapons, its function as the agent of collective security will amount to relatively little in a world in which such weapons are freely produced. Any attack with atomic weapons by a state legally subject to its control will mean that it has failed in its task. We may indeed go further than this and say that a threat of aggressive use by a state actually possessing a stock of such weapons will have to be recognized as bringing into operation (for what it is worth) the right of self defense. Otherwise the law-abiding nation will be exposed to swift annihilation.

We have been assuming for the moment that atomic weapons may be freely produced or acquired. Our argument is that under these conditions the Security Council's protective function is moved back to the prevention of attack. Even in a world without such weapons, the Council would always make great efforts to prevent war breaking out rather than delay its action until hostilities had begun. Now, far more imperatively than before, security from mass destruction demands that the attack shall not be launched. It therefore becomes important to estimate the Council's chances of accumulating such actual power as will make it an effective preventive force.

Article 43 of the Charter imposes on all members the obligation to negotiate with the Security Council agreements specifying the forces and facilities which they are to make available for the maintenance of international security. Later, in Article 45, members undertake to hold air contingents immediately available for urgent military measures in behalf of the United Nations. By these agreements the Security Council might be enabled to mobilize enough power, including even forces using atomic weapons, to insure that an aggressor (other than one of the Permanent Members of the Council - a large exception to be dealt with later) would ultimately be defeated and devastated. If so, this would probably be a strong deterrent.

But will it be possible to conclude and operate the detailed agreements determining national participation in the maintenance of security until specific arrangements have been made for the shared control of atomic energy? The fear and distrust accompanying a competitive development of atomic weapons will hardly provide an atmosphere conducive to working out the network of agreements and plans contemplated in Articles 43-47 of the Charter. In any event, nations attempting to keep a weapon secret are not likely to place it at the disposal of an international agency. At the best, they may agree to use it themselves in behalf of the Security Council. This would not enable the Military Staff Committee as a joint body either to plan or to direct its operations intelligently.

The conclusion would seem to be that the Security Council will have great difficulty in playing a significant role in collective security until a system is worked out, setting narrow limits to the production and distribution, and still narrower limits to the use, of atomic weapons. Failure to devise such a system may indeed destroy the fundamental condition of peace, namely, a working harmony of the United States, the Soviet Union, and Britain.

The joint announcement of November 15, 1945, makes the point that ``complete protection from the destructive use of scientific knowledge" can only be secured by preventing war. The authors of the announcement realized, however, that war might well result from a race in atomic armament. That is why they were not content to rely upon the general effort of the United Nations Organization as guardian of peace, but proposed that it should devise special machinery for the specific task of preventing the destructive use of atomic energy. They were nevertheless wise to insist upon the necessity of success in the general activity of the Organization in promoting the settlement of disputes, strengthening the rule of law, and remedying social and economic conditions which contribute to international conflict. Failing success on this broad front, no system of specific safeguards can be expected to prevent recourse to any kind of force available to states.

Every addition to the destructive power of armaments increases the need for strengthening the agencies and procedures of peaceful adjustment between nations. Not the least of the dangers connected with the atom bomb is that the unsolved problem of its control may lay a blight on all the activities of the United Nations Organization and its entire prospect of consolidation and development. The whole future of the Organization is bottled up with the success or failure of the current effort to find an international solution of the problems posed by the most recent and most formidable achievement of science and engineering. The result of failure would be a situation threatening the world's peace; and the United Nations would be compelled either to cope with this situation or confess its bankruptcy. Coping with the situation could mean nothing else but resuming the effort to establish a control system. This is not a case where the Organization can admit failure and turn to something else.

Left out of account so far is the possibility that a solution might be found outside the United Nations Organization. If the commission established on January 24, 1946, fails to devise an acceptable system of control, conceivably the four or five great powers may be able to work one out among themselves. Putting the control in an agency independent of the United Nations might even have the advantage, it has been suggested, of by-passing the thorny problem of changing the voting rules in the Security Council.

Theoretically this would result in a position where the United Nations Organization could operate precisely as planned at San Francisco. The entire problem of atomic weapons would be removed from its competence, at least in the first instance. Unless the Charter were amended, members could still start proceedings to avert a threat to the peace arising out of this problem. But so long as the control system worked efficiently, the Security Council might perhaps devote itself to preventing illegal use of other instruments of force; and all the other organs of the United Nations could get on with their judicial, economic, and social tasks. In the total scheme of world security the United Nations Organization would occupy a secondary position, since the focus of attention would inevitably be the machinery engaged in controlling the use of atomic energy. This would not be a serious objection, since the important thing is that war should be prevented, not the name of the agencies by which this is to be accomplished.

It would seem likely, however, that what we have called a secondary position in the scheme of world security would be a position of no significance at all. The primacy of the new weapons among the means of destruction will tend to make any agency controlling them not only the focus of attention but the operative center of collective security. Means calculated to prevent their aggressive use will be adequate to prevent any aggression. To the same agency must go that other major business of the Security Council and Military Staff Committee, namely the formulation of plans for the regulation of armaments ``and possible disarmament."\footnote{Articles 26 and 27 of the Charter.} This is major business not primarily because of the wide demand for relief from a wasteful financial burden, but because the prospect of peace is admittedly small in a world of nations arming at discretion. Tho whole business of arms regulation and reduction must be handled together. Separate agencies regulating atomic and non-atomic armaments make as little sense as separate agencies preventing atomic and non-atomic aggression.

The conclusion suggested is that either the atomic control scheme will have to be brought under the United Nations or the security function in general be assigned to the body regulating atomic energy. But if the security function is detached from the United Nations Organization and assigned to a small group consisting exclusively of the great powers, it will have to be performed without those advantages of broad participation which the Organization was designed to insure. The peace would be kept by a naked great-power dictatorship. Any group controlling atomic weapons has in its hands the means of governing the world. If this group is to be also the legally constituted agency of collective security, it is highly important that it should include, as the Security Council does, a substantial representation of the smaller states. To organize it otherwise would be to violate principles proclaimed throughout the war by the democratic nations.\footnote{E.g., The Moscow Declaration, point 4, and the fifth paragraph of the Teheran Declaration.}

If this reasoning is sound, no satisfactory solution of the international problems raised by atomic fission can be found outside the framework of the United Nations Organization. It has been maintained in an earlier chapter that the crux of the whole problem is the necessity of such an arrangement as will give to the Soviet Union and the United States a mutual sense of security. That view does not conflict with the thesis that the arrangement must be one that will give other countries as well a sense of security. To achieve that essential purpose it must be an arrangement in which they participate.

The commission set up by the United Nations is instructed to make proposals ``for the elimination from national armaments of atomic weapons and of all other major weapons adaptable to mass destruction", and ``for effective safeguards by way of inspection and other means to protect complying states against the hazards of violations and evasions."

These instructions represent a necessary and ultimate objective. Nothing less would satisfy the anxious hopes of peace-loving peoples. But a literal ``elimination from national armaments", coupled with ``effective safeguards", may well take a long tine. Practical considerations may dictate an intervening stage of limitation rather than elimination, with the obligation not to use the weapons except with the approval of the United Nations. In this stage, as in the final and ideal one, that part of the plan of control which has to do with the production, possession and use of atomic weapons will necessarily come under the direction of the Security Council. Since that body is not in perpetual session, though ``so organized as to be able to function continuously",\footnote{San Francisco Charter, Article 28, 1.} it will have to entrust the routine of control, including inspection, either to such an existing subordinate agency as the Military Staff Committee or to a specially created subordinate body. Clearly the continuous function of inspection cannot be subject to veto; and one advantage of treating it as a technical, administrative matter handled by a body other than, though responsible to, the Security Council is that, if this is done, no question of changing voting rules established with great difficulty need arise.

On the other hand, any question of enforcement against a nation found to be violating the control regulations will have to be dealt with by the Security Council. Unless the veto of permanent members is abolished, no enforcement can operate against them or against their client states. In a world that has learned how to make and use atomic weapons, as before, the security of all will depend on the good faith of the great powers or on such strength as each nation can muster from its own or allied resources. The United Nations Organization falls short of world government by a margin which includes the United States, the Soviet Union, Britain, China and France. The abolition of the veto would, legally speaking, eliminate this margin. Whether it would make any practical difference is another and a highly debatable question.

There seems to be little prospect that the great-power veto will be given up in any near future, even for the limited purpose of controlling atomic armaments. Statements which we have every reason to regard as approved by the Soviet government sharply oppose any such amendment of the San Francisco Charter, and present indications do not encourage the view that the United States Congress would take any more kindly to the idea than does Moscow.

Even if so great an addition to the legal authority of the Security Council were politically possible, it would not automatically deliver the world from the terrifying risk of atomic war. The greatest states would still exercise a dominating influence in the Organization, and even though the necessary majority were obtained there would still be grave reluctance to launch enforcement measures against one of them. To do so would still be painfully like the beginning of war. It would still be possible for a determined aggressor to play off one interest against another and delay action until it believed itself in a position to defy the world. Such risks may be mitigated to some extent by organization, but only organized power based on willing consent and a deep sense of community can reduce then substantially. It is easy to design machinery; but the more essential condition of peace in an atom-splitting age, as before, is underlying acceptance of common values. Until such acceptance is achieved, the machinery, though far from useless, will be frail. Its justification is that it may help to preserve conditions in which the agreement on common values can grow, thus providing the foundations indispensable to reliable organization.

The legal situation, within the United Nations Organization, then, is that no state is obliged to join in any action against any of the five permanent members of the Security Council. The veto means that action against one of these is not within the legal powers of the Organization. There is little likelihood that this situation will change in the near future. As a control agency over atomic weapons, the Organization thus has the obvious weakness of providing no sanction enforceable against those very states which are most capable of accumulating this type of armament. The Organization can provide means of ascertaining danger and identifying a treaty-breaker. At its very first session the Security Council heard disputes in which two great powers, the U.S.S.R. and Britain, were accused of endangering the peace of the world. But, so long as the veto survives, the ultimate external deterrent operating on the five permanent members of the Security Council will be the prospect that a violation of their agreements will bring down upon them retaliation which the United Nations Organization cannot order under the present terms of the Charter.

The legal position being what it is - and the legal position corresponds to the political difficulty of establishing a world government strong enough to coerce great powers - there will be a natural tendency, on the part of states fearing conflict with one of the great powers, to seek assurance of help outside the provisions of the Charter. They may find this in bilateral treaties of alliance, or in regional pacts, or in both. The search for reinsurance against the possible breakdown of a general security system was a familiar phenomenon during the life of the League of Nations, and it was well under way again before the end of World War II. The San Francisco Charter gives formal recognition to those realities in world politics which provide the motive for this search; and the advent of atomic weapons has done nothing to check the tendency. It may, however, do something to change the direction in which states will look for supplementary guaranties.

The overall trend that seems most likely will be for states to group themselves around that neighbor who combines the greatest capacity to launch atomic attack with the greatest capacity to survive it. This trend will probably not alter the constellation of hemispheric security in the Americas; but it may radically change the shape of things in Europe. The present movement there is towards an Eastern grouping around the Soviet Union, and a Western grouping around Britain and France. But even if France soon wins the secret of manufacturing atomic weapons, and if she and Britain merge any productive capacity which they may be able to develop, they will find themselves, as soon as the Soviet Union is in production, in a position which at least on the defensive side will be inferior to that of Russia. The Soviet system combines two features that will be useful in atomic warfare, namely totalitarian central government and ample space for dispersion. Since this will mean a higher probability of survival, it may increase the drawing-power of Moscow as compared with that of London and Paris. The Western grouping will be weakened, while the primacy of Russia in Europe will be still further emphasized. The result for Britain - and for France also if she does not enter the Russian orbit - must be increased reliance on America.

Such a clear-cut polarization of power around the two great continental countries, the Soviet Union and the United States, offers scant prospect of a peaceful world co-operating in the common purpose of increased welfare. What chance there is of averting it lies, it seems, in the fullest and speediest possible development of all the conciliatory, judicial, economic and social activities planned for the United Nations Organization, coupled with the constant effort to devise such a system of control over the use of atomic energy as will overcome the fear that the new discoveries have brought upon the world.

\newpage

\par

~~