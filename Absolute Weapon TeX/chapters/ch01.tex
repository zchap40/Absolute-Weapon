
\chapter[War in the Atomic Age]{War in the Atomic Age}

\maketitle

\vspace{-2pt}

\noindent{\normalsize \textbf{Bernard Brodie}}

\vspace{39pt}

Most of those who have held the public ear on the subject of the bomb have been content to assume that war and obliteration are now completely synonymous, and that modern man must therefore be either obsolete or fully ripe for the millennium. No doubt the state of obliteration - if that should indeed be the future fate of nations which cannot resolve their disputes - provides little scope for analysis. A few degrees difference in nearness to totality is of relatively small account. But in view of man's historically tested resistance to drastic changes in behavior, especially in a benign direction, one may be pardoned for wishing to examine the various possibilities inherent in the situation before taking any one of them for granted.

It is already known to us all that a war with atomic bombs would be immeasurably more destructive and horrible than any the world has yet known. That fact is indeed portentous, and to many it is overwhelming. But as a datum for the formulation of policy it is in itself of strictly limited utility. It underlines the urgency of our reaching correct decisions, but it does not help us to discover which decisions are in fact correct.

Men have in fact been converted to religion at the point of the sword, but the process generally required actual use of the sword against recalcitrant individuals. The atomic bomb does not lend itself to that kind of discriminate use. The wholesale conversion of mankind away from those parochial attitudes bound up in nationalism is a consummation devoutly to be wished and, where possible, to be actively promoted. But the mere existence of the bomb does not promise to accomplish it at an early enough time to be of any use. The careful handling required to assure long and fruitful life to the Age of Atomic Energy will in the first instance be a function of distinct national governments, not all of which, incidentally, reflect in their behavior the will of the popular majority.

Governments are of course ruled by considerations not wholly different from those which affect even enlightened individuals. That the atomic bomb is a weapon of incalculable horror will no doubt impress most of them deeply. But they have never yet responded to the horrific implications of war in a uniform way. Even those governments which feel impelled to the most drastic self-denying proposals will have to grapple not merely with the suspicions of other governments but with the indisputable fact that great nations have very recently been ruled by men who were supremely indifferent to horror, especially horror inflicted by them on people other than their own.

Statesmen have hitherto felt themselves obliged to base their policies on the assumption that the situation might again arise where to one or more great powers war looked less dangerous or less undesirable than the prevailing conditions of peace. They will want to know how the atomic bomb affects that assumption. They must realize at the outset that a weapon so terrible cannot but influence the degree of probability of war for any given period in the future. But the degree of that influence or the direction in which it operates is by no means obvious. It has, for example, been stated over and over again that the atomic bomb is \emph{par excellence} the weapon of aggression, that it weights the scales overwhelmingly in favor of surprise attack. That if true would indicate that world peace is even more precarious than it was before, despite the greater horrors of war. But is it inevitably true? If not, then the effort to make the reverse true would deserve a high priority among the measures to be pursued.

Thus, a series of questions present themselves. Is war more or less likely in a world which contains atomic bombs? If the latter, is it \emph{sufficiently} unlikely - sufficiently, that is, to give society the opportunity it desperately needs to adjust its politics to its physics? What are the procedures for effecting that adjustment within the limits of our opportunities? And how can we enlarge our opportunities? Can we transpose what appears to be an immediate crisis into a long-term problem, which presumably would permit the application of more varied and better-considered correctives than the pitifully few and inadequate measures which seem available at the moment?

It is precisely in order to answer such questions that we turn our attention to the effect of the bomb on the character of war. We know in advance that war, if it occurs, will be very different from what it was in the past, but what we want to know is: how different, and in what ways? A study of those questions should help us to discover the conditions which will govern the pursuit of world security in the future and the feasibility of proposed measures for furthering that pursuit. At any rate, we know that it is not the mere existence of the weapon but rather its effects on the traditional pattern of war which will govern the adjustments which states will make in their relations with each other.

\noindent\hfil\rule{0.4\textwidth}{.4pt}\hfil

\vspace{4pt}

The Truman-Attlee-King statement of November 15, 1945 epitomized in its first paragraph a few specific conclusions concerning the bomb which have evolved as of that date: ``We recognize that the application of recent scientific discoveries to the methods and practice of war has placed at the disposal of mankind means of military destruction hitherto unknown, against which there can be no adequate military defense, and in the employment of which no single nation can in fact have a monopoly."

This observation, it would seem, is one upon which all reasonable people would now be agreed. But it should be noted that of the three propositions presented in it the first is either a gross understatement or meaningless, the second has in fact been challenged by persons in high military authority, and the third, while generally admitted to be true, has nevertheless been the subject of violently clashing interpretations. In any case, the statement does not furnish a sufficient array of postulates for the kind of analysis we wish to pursue.

It is therefore necessary to start out afresh and examine the various features of the bomb, its production, and its use which are of military importance. Presented below are a number of conclusions concerning the character of the bomb which seem to this writer to be inescapable. Some of the eight points listed already enjoy fairly universal acceptance; most do not. After offering with each one an explanation of why he believes it to be true, the writer will attempt to deduce from these several conclusions or postulates the effect of the bomb on the character of war.

\begin{enumerate}[I.]

\item \textbf{The power of the present bomb is such that any city in the world can be effectively destroyed by one to ten bombs.}

\end{enumerate}

While this proposition is not likely to evoke much dissent, its immediate implications have been resisted or ignored by important public officials. These implications are two-fold. First, it is now physically possible for air forces no greater than those existing in the recent war to wipe out all the cities of a great nation in a single day - and it will be shown subsequently that what is physically possible must be regarded as tactically feasible. Secondly, with our present industrial organization the elimination of our cities would mean the elimination for military purposes of practically the whole of our industrial structure. But before testing these extraordinary implications, let us examine and verify the original proposition.

The bomb dropped on Hiroshima completely pulverized an area of which the radius from the point of detonation was about one and one-quarter miles. However, everything within a radius of two miles was blasted with some burning and between two and three miles the buildings were about half destroyed. Thus the area of total destruction covered about four square miles, and the area of destruction and substantial damage extended over some twenty-seven square miles. The bomb dropped on Nagasaki, while causing less damage than the Hiroshima bomb because of the physical characteristics of the city, was nevertheless considerably more powerful. We have it on Dr. J. Robert Oppenheimer's authority that the Nagasaki bomb ``would have taken out ten square miles, or a bit more, if there had been ten square miles to take out."\footnote{``Atomic Weapons and the Crisis in Science", \textit{Saturday Review of Literature}, November 24, 1945, p. 10.} From the context in which that statement appears it is apparent that Dr. Oppenheimer is speaking of an area of total destruction.

The city of New York is listed in the \emph{World Almanac} as having an area of 365 square miles. But it obviously would not require the pulverization of every block of it to make the whole area one of complete chaos and horror. Ten well-placed bombs of the Nagasaki type would eliminate that city as a contributor to the national economy, whether for peace or war, and convert it instead into a catastrophe area in dire need of relief from outside. If the figure of ten bombs be challenged, it need only be said that it would make very little difference militarily if twice that number of bombs were required. Similarly, it would be a matter of relative indifference if the power of the bomb were so increased as to require only five to do the job. Increase of power in the individual bomb is of especially little moment to cities of small or medium size, which would be wiped out by one bomb each whether that bomb were of the Nagasaki type or of fifty times as much power. No conceivable variation in the power of the atomic bomb could compare in importance with the disparity in power between atomic and previous types of explosives.

The condition at this writing of numerous cities in Europe and Japan sufficiently underlines the fact that it does not require atomic bombs to enable man to destroy great cities. TNT and incendiary bombs when dropped in sufficient quantities are able to do a quite thorough job of it. For that matter, it should be pointed out that a single bomb which contains in itself the concentrated energy of 20,000 tons of TNT is by no means equal in destructive effect to that number of tons of TNT distributed among bombs of one or two tons each. The destructive radius of any one bomb increases only with the cube root of the explosive energy released, and thus the very concentration of power in the atomic bomb detracts from its overall effectiveness. The bomb must be detonated from an altitude of at least 1,000 feet if the full spread of its destructive radius is be to realized, and much of the blast energy is absorbed by the air above the target. But the sum of initial energy is quite enough to afford such losses.

If should be obvious that there is much more than a logistic difference involved between a situation where a single plane sortie can cause the destruction of a city like Hiroshima and one in which at least 500 bomber sorties are required to do the same job. Nevertheless, certain officers of the U.S. Army Air Forces, in an effort to ``deflate" the atomic bomb, have observed publicly enough to have their comments reported in the press that the destruction wrought at Hiroshima could have been effected by two days of routine bombing with ordinary bombs. Undoubtedly so, but the 500 or more bombers needed to do the job under those circumstances would if they were loaded with atomic bombs be physically capable of destroying 500 or more Hiroshimas in the same interval of time. That observation discounts certain tactical considerations. These will be taken up in due course, but for the moment it is sufficient to point out that circumstances do arise in war when it is the physical carrying capacity of the bombing vehicles rather than tactical considerations which will determine the amount of damage done.

\begin{enumerate}[resume*]

\item \textbf{No adequate defense against the bomb exists, and the possibilities of its existence in the future are exceedingly remote.}

\end{enumerate}

This proposition requires little supporting argument in so far as it is a statement of existing fact. But that part of it which involves a prediction for the future conflicts with the views of most of the high-ranking military officers who have ventured opinions on the implications of the atomic bomb. No layman can with equanimity differ from the military in their own field, and the present writer has never entertained the once-fashionable view that the military do not know their own business. But, apart from the question of objectivity concerning professional interests - in which respect the record of the military profession is neither worse nor better than that of other professions - the fact is that the military experts have based their arguments mainly on presumptions gleaned from a field in which they are generally not expert, namely, military \emph{history}. History is at best an imperfect guide to the future, but when imperfectly understood and interpreted it is a menace to sound judgment.

The defense against hostile missiles in all forms of warfare, whether on land, sea, or in the air, has thus far depended basically on a combination of, first, measures to reduce the number of missiles thrown or to interfere with their aim (i.e., defense by offensive measures) and, secondly, ability to absorb those which strike. To take an obvious example, the large warship contains in itself and in its escorting air or surface craft a volume of fire power which usually reduces and may even eliminate the blows of the adversary. Unlike most targets ashore, it also enjoys a mobility which enables it to maneuver evasively under attack (which will be of no value under atomic bombs). But unless the enemy is grotesquely inferior in strength, the ship's ability to survive must ultimately depend upon its compartmentation and armor, that is, on its ability to absorb punishment.

The same is true of a large city. London was defended against the German V-1 or ``buzz-bomb" first by concerted bombing attacks upon the German experimental stations, industrial plants, and launching sites, all of which delayed the V-1 attack and undoubtedly greatly reduced the number of missiles ultimately launched. Those which were nevertheless launched were met by a combination of fighter planes, antiaircraft guns, and barrage balloons. Towards the end of the eighty-day period which covered the main brunt of the attack, some 75 per cent of the bombs launched were being brought down, and, since many of the remainder were inaccurate in their flight, only 9 per cent were reaching London.\footnote{Duncan Sandys, \textit{Report on the Flying Bomb}, pamphlet issued by the British Information Services, September, 1944, p. 9.} These London was able to ``absorb"; that is, there were casualties and damage but no serious impairment of the vital services on which depended the city's life and its ability to serve the war effort.

It is precisely this ability to absorb punishment, whether one is speaking of a warship or a city, which seems to vanish in the face of atomic attack. For almost any kind of target selected, the so-called ``static defenses" are defenses no longer. For the same reason too, mere reduction in the number of missiles which strike home is not sufficient to save the target, though it may have some effect on the enemy's selection of targets. The defense of London against V-1was considered effective, and yet in eighty days some 2,300 of those missiles hit the city. The record bag was that of August 28, 1944, when out of 101 bombs which approached England 97 were shot down and only four reached London. But if those four had been atomic bombs, London survivors would not have considered the record good. Before we can speak of a defense against atomic bombs being effective, \label{I-frustrate} \emph{the frustration of the attack for any given target area must be complete}. Neither military history nor an analysis of present trends in military technology leaves appreciable room for hope that means of completely frustrating attack by aerial missiles will be developed.

In his speech before the Washington Monument on October 5, 1945, Fleet Admiral Chester W. Nimitz correctly cautioned the American people against leaping to the conclusion that the atomic bomb had made armies and navies obsolete. But he could have based his cautionary note on better grounds than he in fact adopted. ``Before risking our future by accepting these ideas at face value," he said, ``let us examine the historical truth that, at least up to this time, there has never yet been a weapon against which man has been unable to devise a counter-weapon or a defense.\footnote{For the text of the speech see the \textit{New York Times}, October 6, 1945, p. 6. See also the speech of President Truman before Congress on October 23, 1945, in which he said: ``Every new weapon will eventually bring some counter-defense against it."}

Apart from the possible irrelevancy for the future of this observation - against which the phrase ``at least up to this time" provides only formal protection - the fact is that it is not historically accurate. A casual reading of the history of military technology does, to be sure, encourage such a doctrine. The naval shell gun of 1837, for example, was eventually met with iron armor, and the iron armor in turn provoked the development of the ``built-up" gun with greater penetrating power; the submarine was countered with the hydrophone and supersonic detector and with depth charges of various types; the bombing airplane accounted for the development of the specialized fighter aircraft, the highly perfected antiaircraft gun, and numerous ancillary devices. So it has always been, and the tendency is to argue that so it always will be.

In so far as this doctrine becomes dogma and is applied to the atomic bomb, it becomes the most dangerous kind of illusion. We have already seen that the defense against the V-1 was only \emph{relatively} effective, and something approaching much closer to perfect effectiveness would have been necessary for V-1 missiles carrying atomic bombs. As a matter of fact, the defense against the V-2 rocket were of practically zero effectiveness, and those who know most about it admit that thus far there has been no noteworthy progress against the V-2.\footnote{See Ivan A. Getting, ``Facts About Defense," \textit{Nation}, Special Supplement, Dec. 22, 1945, p. 704. Professor Getting played a key part in radar development for antiaircraft work and was especially active in measures taken to defend London against V-1 and V-2.}

These, to be sure, were new weapons. But what is the story of the older weapons? After five centuries of the use of hand arms with fire-propelled missiles, the large numbers of men killed by comparable arms in the recent war indicates that no adequate answer has yet been found for the bullet.\footnote{The new glass-fiber body armor, ``Doron", will no doubt prove useful but is not expected to be of more than marginal effectiveness.} Ordinary TNT, whether in shell, bomb, or torpedo, can be ``countered" to a degree by the dispersion of targets or by various kinds of armor, but the enormous destruction wrought by this and comparable explosives on land, sea, and in the air in World War II is an eloquent commentary on the limitations of the defenses. The British following the first World War thought they had in their ``Asdic" and depth charges the complete answer to the U-boat, but an only slightly improved U-boat succeeded in the recent war in sinking over 23 million gross tons of shipping. So the story might go on endlessly. It has simply become customary to consider an ``answer" satisfactory when it merely diminishes or qualifies the effectiveness of the weapon against which it is devised, and that kind of custom will not do for the atomic bomb.

Despite such statements as that of Canadian General A. G. L. McNaughton that means with which to counter the atomic bomb area already ``clearly in sight",\footnote{\textit{New York Herald Tribune}, October 6, 1945, p. 7.} it seems pretty well established that there is no \emph{specific} reply to the bomb. The physicists and chemists who produced the atomic bomb are apparently unanimous on this point: that while there was a scientific consensus long before the atomic bomb existed that it could be produced, no comparable opinion is entertained among scientists concerning their chances of devising effective counter-measures. The bomb itself is as free from direct interference of any kind as is the ordinary bomb. When the House Naval Affairs Committee circulated a statement that electronic means were already available for exploding atomic bombs ``far short of their objective without the necessity of locating their position",\footnote{\textit{New York Times}, October 12, 1945, p. 1.} scientists qualified to speak promptly denied this assertion and it was even disowned by its originators.

Any active defense at all must be along the lines of affecting the carrier, and we have already noted that even when used with the relatively vulnerable airplane or V-1 the atomic bomb poses wholly new problems for the defense. A nation which had developed strong defenses against invading aircraft, which had found reliable means of interfering with radio-controlled rockets, which had developed highly efficient counter-smuggling and counter-sabotage agencies, and which had dispersed through the surrounding countryside substantial portions of the industries and populations normally gathered in urban communities would obviously be better prepared to resist atomic attack than a nation which had either neglected or found itself unable to do these things. But it would have only a relative advantage over the latter; it would still be exposed to fearful destruction.

In any case, technological progress is not likely to be confined to measures of defense. The use of more perfect vehicles and of more destructive bombs in greater quantity might very well offset any gains in defense. And the bomb already has a fearful lead in the race.

Random and romantic reflections on the miracles which science has already wrought are of small assistance in our speculations on future trends. World War II saw the evolution of numerous instruments of war of truly startling ingenuity. But with the qualified exception of the atomic bomb itself (the basic principle of which was discovered prior to but in the same year of the outbreak of war in Europe), all were simply mechanical adaptations of scientific principles which were well known long before the war. It was no doubt a long step from the discovery in 1922 of the phenomenon upon which radar is based to the use of the principle in an antiaircraft projectile fuse, but here too realization that it might be so used considerably antedated the fuse itself.

The advent of a ``means of destruction hitherto unknown" - to quote the Truman-Attlee-King statement - is certainly not new. The steady improvement of weapons of war is an old story, and the trend in that direction has in recent years been accelerated. But thus far each new implement has, at least initially, been limited enough in the scope of its use or in its strategic consequences to permit some timely measure of adaptation both on the battlefield and in the minds of strategists and statesmen. Even the most ``revolutionary" developments of the past seem by contrast with the atomic bomb to have been minor steps in a many-sided evolutionary process. This process never permitted any one invention in itself to subvert or even to threaten for long the previously existing equilibrium of military force. Any startling innovation either of offense or defense provoked some kind of answer in good time, but the answer was rarely more than a qualified one and the end result was usually a profound and sometimes a politically significant change in the methods of waging war.\footnote{For a discussion of developing naval technology over the last hundred years and its political significance see Bernard Brodie, \textit{Sea Power in the Machine Age}, Princeton, N.J., 2nd. ed. 1943.}

With the introduction, however, of an explosive agent which is several million times more potent on a pound for pound basis than the most powerful explosives previously known, we have a change of quite another character. The factor of increase of destructive efficiency is so great that there arises at once the strong presumption that the experience of the past concerning eventual adjustment might just as well be thrown out the window. Far from being something which merely ``adds to the complexities of field commanders", as one American military authority put it, the atomic bomb seems so far to overshadow any military invention of the past as to render comparison ridiculous.

\begin{enumerate}[resume*]

\item \textbf{The atomic bomb not only places an extraordinary military premium upon the development of new types of carriers but also greatly extends the destructive range of existing carriers.}

\end{enumerate}

World War II saw the development and use by the Germans of rockets capable of 220 miles range and carrying approximately one ton each of TNT. Used against London, these rockets completely baffled the defense. But for single-blow weapons which were generally inaccurate at long distances even with radio control, they were extremely expensive. It is doubtful whether the sum of economic damage done by these missiles equalled the expenditure which the Germans put into their development, production, and use. At any rate, the side enjoying command of the air had in the airplane a much more economical and longer-range instrument for inflicting damage on enemy industry than was available in the rocket. The capacity of the rocket-type projectile to strike without warning in all kinds of weather with complete immunity from all known types of defenses guaranteed to it a supplementary though subordinate role to bomber-type aircraft. But its inherent limitations, so long as it carried only chemical explosives, were sufficient to warrant considerable reserve in predictions of its future development.

\label{I-range1}

However, the power of the new bomb completely alters the considerations which previously governed the choice of vehicles and the manner of using them. A rocket far more elaborate and expensive than the V-2 used by the Germans is still an exceptionally cheap means of bombarding a country if it can carry in its nose an atomic bomb. The relative inaccuracy of aim - which continued research will no doubt reduce - is of much diminished consequence when the radius of destruction is measured in miles rather than yards. And even with existing fuels such as were used in the German V-2, it is theoretically feasible to produce rockets capable of several thousands of miles of range, though the problem of \emph{controlling} the flight of rockets over such distances is greater than is generally assumed.

Of more immediate concern than the possibilities of rocket development, however, is the enormous increase in effective bombing range which the atomic bomb gives to \emph{existing types of aircraft}. That it has this effect becomes evident when one examines the various factors which determine under ordinary - that is, non-atomic bomb - conditions whether a bombing campaign is returning military dividends. First, the campaign shows profit only if a large proportion of the planes, roughly 90 per cent or more, are returning from individual strikes.\footnote{The actual figure of loss tolerance depends on a number of variables, including replacement rate of planes and crews, morale factors, the military value of the damage being inflicted on the enemy, and the general strategic position at the moment. The 10 per cent figure used for illustration in the text above was favored by the war correspondents and press analysts during the recent war, but it must not be taken too literally.} Otherwise one's air force may diminish in magnitude more rapidly than the enemy's capacity to fight. Each plane load of fuel must therefore cover a two-way trip, allowing also a fuel reserve for such contingencies as adverse winds and combat action, thereby diminishing range by at least one-half from the theoretical maximum.

But the plane cannot be entirely loaded with fuel. It must also carry besides its crew a heavy load of defensive armor and armament. Above all, it must carry a sufficient load of bombs to make the entire sortie worth while - a sufficient load, that is, to warrant attendant expenditures in fuel, engine maintenance, and crew fatigue. The longer the distance covered, the smaller the bomb load per sortie and the longer the interval between sorties. To load a plane with thirty tons of fuel and only two tons of bombs, as we did in our first B-29 raid on Japan, will not do for a systematic campaign of strategic bombing. One must get closer to the target and thus transfer a greater proportion of the carrying capacity from fuel to bombs.\footnote{It should be noticed that in the example of the B-29 raid of June 15, 1944, cited above, a reduction of only one-fourth in the distance and therefore in the fuel load could make possible (unless the plane was originally overloaded) a tripling or quadrupling of the bomb load. Something on that order was accomplished by our seizure of bases in the Mariannas, some 300 miles closer to the target than the original Chinese bases and of course much easier supplied. The utility of the Mariannas bases was subsequently enhanced by our capture of Iwo Jima and Okinawa, which served as emergency landing fields for returning B-29s and also as bases for escorting fighters and rescue craft. Towards the end of the campaign we were dropping as much as 6,000 tons of bombs in a single raid on Tokyo, thereby assuring ourselves high military dividends per sortie investment.} What we then come out with is an effective bombing range less than one-fourth the straight-line cruising radius of the plane under optimum conditions. In other words a plane capable, without too much stripping of its equipment, of a 6,000-mile non-stop flight would probably have an effective bombing range of substantially less than 1,500 miles.

With atomic bombs, however, the considerations described above which so severely limit bomb range tend to vanish. There is no question of increasing the number of bombs in order to make the sortie profitable. One per plane is quite enough. The gross weight of the atomic bomb is secret, but even if it weighed two to four tons it would still be a light load for a B-29. It would certainly be a sufficient pay load to warrant any conceivable military expenditure on a single sortie. The next step then becomes apparent. Under the callously utilitarian standards of military bookkeeping, a plane and its crew can very well be sacrificed in order to deliver an atomic bomb to an extreme distance. We have, after all, the recent and unforgettable experience of the Japanese \textit{Kamikaze}.\footnote{On several occasions the U.S. Army Air Forces also demonstrated its willingness to sacrifice availability of planes and crews - though not the lives of the latter - in order to carry out specific missions. Thus in the Doolittle raid against Japan of April 1942, in which sixteen Mitchell bombers took off from the carrier \textit{Hornet} it was known beforehand that none of the planes would be recovered even if they succeeded in reaching China (which several failed to do for lack of fuel) and that the members of the crews were exposing themselves to uncommon hazard. And the cost of the entire expedition was accepted mainly for the sake of dropping 16 tons of ordinary bombs! Similarly, several of the Liberators which bombed the Ploesti oil fields in August 1943 had insufficient fuel to return to their bases in North Africa and, as was foreseen, had to land in neutral Turkey where planes and crews were interned.} Thus, the plane can make its entire flight in one direction, and its range would be almost as great with a single atomic bomb as it would be with no bomb load whatever. The non-stop flight during November 1945 of a B-29 from Guam to Washington, D.C., almost 8,200 statute miles, was in this respect more than a stunt. It was a rough indication of the extreme \emph{effective} bombing range with atomic bombs of types of aircraft already in use.\footnote{See \textit{New York Times}, November 21, 1945, p. 1. It should be noticed that the plane had left about 300 gallons, or more than one ton, of gasoline upon landing in Washington. It was of course stripped of all combat equipment (e.g., armor, guns, ammunition, gun-directors, and bomb-sights) in order to allow for a greater gasoline load. Planes bent on a bombing mission would probably have to carry some of this equipment, even if their own survival was not an issue, in order to give greater assurance of their reaching the target.}

Under the conditions just described, any world power is able from bases within its own territories to destroy all the cities of any other world power. It is \emph{not} necessary, despite the assertions to the contrary of various naval and political leaders including President Truman, to seize advanced bases close to enemy territory as a prerequisite to effective use of the bomb.\footnote{See President Truman's speech before Congress on the subject of universal military training, reported in the \textit{New York Times}, October 24, 1945, p. 3.} The lessons of the recent Pacific war in that respect are not merely irrelevant but misleading, and the effort to inflate their significance for the future is only one example of the pre-atomic thinking prevalent today even among people who understand fully the power of the bomb. To recognize that power is one thing; to draw out its full strategic implications is quite another.

The facts just presented do not mean that distance loses all its importance as a barrier to conflict between the major power centers of the world. It would still loom large in any plans to consolidate an atomic bomb attack by rapid invasion and occupation. It would no doubt also influence the success of the bomb attack itself. Rockets are likely to remain of lesser range than aircraft and less accurate near the limits of their range, and the weather hazards which still affect aircraft multiply with distance. Advanced bases will certainly not be valueless. But it is nevertheless a fact that under existing technology the distance separating, for example, the Soviet Union from the United States offers no direct immunity to either with respect to atomic bomb attack, though it does so for all practical purposes with respect to ordinary bombs.\footnote{Colonel Clarence S. Irvine, who commanded the plane which flew non-stop from Guam to Washington, was reported by the press as declaring that one of the objects of the flight was ``to show the vulnerability of our country to enemy air attack from vast distances." \textit{New York Times}, November 21, 1945, p. 1.}

\label{I-range2}

\begin{enumerate}[resume*]

\item \textbf{Superiority in air forces, though a more effective safeguard in itself than superiority in naval or land forces, nevertheless fails to guarantee security.}

\end{enumerate}

This proposition is obviously true in the case of very long range rockets, but let us continue to limit our discussion to existing carriers. In his \textit{Third Report to the Secretary of War}, dated November 12, 1945, General H. H. Arnold, commanding the Army Air Forces, made the following statement: ``Meanwhile [i.e., until very long range rockets are developed], the only known effective means of delivering atomic bombs in their present stage of development is the very heavy bomber, and that is certain of success only when the user has air superiority".\footnote{See printed edition of the \textit{Report}, p. 68. In the sentence following the one quoted, General Arnold adds that this statement is ``perhaps true only temporarily", but it is apparent from the context that the factor he has in mind which might terminate its ``truthfulness" is the development of rockets comparable to the V-2 but of much longer range. The present discussion is not concerned with rockets at all.}

This writer feels no inclination to question General Arnold's authority on matters pertaining to air combat tactics. However, it is pertinent to ask just what the phrase ``certain of success" means in the sentence just quoted, or rather, how much certainty of success is necessary for each individual bomb before. an atomic bomb attack is considered feasible. In this respect one gains some insight into what is in General Arnold's mind from a sentence which occurs somewhat earlier on the same page in the \textit{Report}: ``Further, the great unit cost of the atomic bomb means that as nearly as possible every one must be delivered to its intended target." Here is obviously the major premise upon which the conclusion above quoted is based, and one is not disputing General Arnold's judgment in the field of his own specialization by examining a premise which lies wholly outside of it.

When the bombs were dropped on Hiroshima and Nagasaki in August 1945, there were undoubtedly very few such bombs in existence - which would be reason enough for considering each one precious regardless of cost. But their development and production up to that time amounted to some 2 billions of dollars, and that figure would have to be divided by the number made to give the cost of each. If, for example, there were 20 in existence, the unit cost would have to be reckoned at \$100,000,000. That, indeed, is a staggering sum for one missile, being approximately equivalent to the cost of one \emph{Iowa} class battleship. It is quite possible that there were fewer than 20 at that time, and that the unit cost was proportionately higher. For these and other reasons, including the desirability for psychological effect of making certain that the initial demonstration should be a complete success, one can understand why it was then considered necessary, as General Arnold feels it will remain necessary, to ``run a large air operation for the sole purpose of delivering one or two atomic bombs."\footnote{\emph{Ibid}., p. 68.}

But it is of course clear that as our existing plant is used for the production of more bombs - and it has already been revealed that over three-fourths of the 2 billion dollars went into capital investment for plants and facilities\footnote{According to the figures provided the MacMahon Committee by Major General Leslie R. Groves, the total capital investment spent and committed for plants and facilities as of June 30, 1945 was \$1,595,000,000. Total operating costs up to the time the bombs were dropped in August were \$405,000,000. The larger sum is broken down as follows:

\vspace{10pt}

\begin{tabular}{@{}lr@{}}
\toprule
Manufacturing facilities alone				&	\$1,242,000,000\\
Research								&	\$186,000,000\\
Housing for workers						&	\$162,500,000\\
Workmen's compensation and medical care	&	\$4,500,000\\
\midrule
\textbf{Total}							&	\$1,595,000,000\\
\botrule
\end{tabular}

\vspace{10pt}

One might question the inclusion of the last item as a part of ``capital investment", but it is in any case an insignificant portion of the whole.} - the unit cost will decline. Professor Oppenheimer has estimated that even with existing techniques and facilities, that is, allowing for no improvements whatever in the production processes, the unit cost of the bomb should easily descend to something in the neighborhood of \$1,000,000.\footnote{\textit{loc. cit.}, p. 10.}

Now a million dollars is a large sum of money for any purpose other than war. Just what it means in war may be gauged by the fact that it amounts to substantially less than the cost of two fully equipped Flying Fortresses (B-17s, not B-29s), a considerable number of which were expended in the recent war without waiting upon situations in which each sortie would be certain of success. The money cost of the war to the United States was sufficient to have paid for 2 or 3 hundred thousand of our million dollar bombs. It is evident, therefore, that in the future it will not be the unit cost of the bomb but the number of bombs actually available which will determine the acceptable wastage in any atomic bomb attack.\footnote{This discussion recalls the often repeated canard that admirals have been cautious of risking battleships in action because of their cost. The 13 old battleships and 2 new ones available to us just after Pearl Harbor reflected no great money value, but they were considered precious because they were scarce and irreplaceable. Later in the war, when new battleships had joined the fleet and when we had eliminated several belonging to the enemy, no battleships were withheld from any naval actions in which they could be of service. Certainly they were not kept out of the dangerous waters off Normandy, Leyte, Luzon, and Okinawa.}

Thus, if Country A should have available 5,000 atomic bombs, and if it should estimate that 500 bombs dropped on the cities of Country B would practically eliminate the industrial plant of the latter nation, it could afford a wastage of bombs of roughly 9 to 1 to accomplish that result. If its estimate should prove correct and if it launched an attack on that basis, an expenditure of only 5 billions of dollars in bombs would give it an advantage so inconceivably overwhelming as to make easy and quick victory absolutely assured - provided it was able somehow to prevent retaliation in kind. The importance of the latter proviso will be elaborated in the whole of the following chapter. Meanwhile it should be noted that the figure of 5,000 bombs cited above is, as will shortly be demonstrated, by no means an impossible or extreme figure for any great power which has been producing atomic bombs over a period of ten or fifteen years.

To approach the same point from another angle, one might take an example from naval warfare. The commander of a battleship will not consider the money cost of his 16-inch shells (perhaps \$3,000 each at the gun's breech) when engaging an enemy battleship, He will not hesitate, at least not for financial reasons, to open fire at extreme range, even if he can count on only one hit in thirty rounds. The only consideration which could give him pause would be the fear of exhausting his armor-piercing ammunition before he has sunk or disabled the enemy ship. The cost of each shell, to be sure, is much smaller than the cost of one atomic bomb, but the amount of damage each hit accomplishes is also smaller - disproportionately smaller by a wide margin.

In calculations of acceptable wastage, the money cost of a weapon is usually far overshadowed by considerations of availability; but in so far as it does enter into those calculations, it must be weighed against the amount of damage done the enemy with each hit. A million dollar bomb which can do a billion dollars worth of damage - and that is a conservative figure - is a very cheap missile indeed. In fact, one of the most frightening things about the bomb is that it makes the destruction of enemy cities an immeasurably cheaper process than it was before, cheaper not alone in terms of missiles but also in terms of the air forces necessary to do the job. Provided the nation using them has enough such bombs available, it can afford a large number of misses for each hit obtained.

To return to General Arnold's observation, we know from the experience of the recent war that very inferior air forces can penetrate to enemy targets if they are willing to make the necessary sacrifices. The Japanese aircraft which raided Pearl Harbor were considerably fewer in number than the American planes available at Pearl Harbor. That, to be sure, was a surprise attack preceding declaration of hostilities, but such possibilities must be taken into account for the future. At any rate, the Japanese air attacks upon our ships off Okinawa occurred more than 3 years after the opening of hostilities, and there the Japanese, who were not superior in numbers on any one day and who did indeed lose over 4,000 planes in 2 months of battle, nevertheless succeeded in sinking or damaging no fewer than 253 American warships. For that matter, the British were effectively raiding targets deep in Germany, and doing so without suffering great casualties, long before they had overtaken the German lead in numbers of aircraft. The war has demonstrated beyond the shadow of a doubt that the sky is much too big to permit one side, however superior, to shut out enemy aircraft completely from the air over its territories.

The concept of ``command of the air", which has been used altogether too loosely, has never been strictly analogous to that of ``command of the sea". The latter connotes something approaching absolute exclusion of enemy surface craft from the area in question. The former suggests only that the enemy is suffering losses greater than he can afford, whereas one's own side is not. But the appraisal of tolerable losses is in part subjective, and is also affected by several variables which may have little to do with the number of planes downed. Certainly the most important of those variables is the amount of damage being inflicted on the bombing raids. An air force which can destroy the cities in a given territory has for all practical purposes the fruits of command of the air, regardless of its losses.

Suppose, then, one put to the Army Air Forces the following question: If 3,000 enemy bombers flying simultaneously but individually (i.e., completely scattered)\footnote{The purpose of the scattering would be simply to impose maximum confusion on the superior defenders. Some military airmen have seriously attempted to discount the atomic bomb with the argument that a hit upon a plane carrying one would cause the bomb to explode, blasting every other plane for at least a mile around out of the air. That is not why formation flying is rejected in the example above. Ordinary bombs are highly immune to such mishaps, and from all reports of the nature of the atomic bomb it would seem to be far less likely to undergo explosion as a result even of a direct hit.} invaded our skies with the intention of dividing between them as targets most of the 92 American cities which contain a population of 100,000 or over (embracing together approximately 29 per cent of our total population), if each of those planes carried an atomic bomb, and if we had 9,000 alerted fighters to oppose them, how much guarantee of protection could be accorded those cities? The answer would undoubtedly depend on a number of technical and geographic variables, but under present conditions it seems to this writer all too easy to envisage situations in which few of the cities selected as targets would be spared overwhelming destruction.

That superiority which results in the so-called ``command of the air" is undoubtedly necessary for successful strategic bombing with ordinary bombs, where the weight of bombs required is so great that the same planes must be used over and over again. In a sense also (though one must register some reservations about the exclusion of other arms) General Arnold is right when he says of atomic bomb attack: ``For the moment, at least, absolute air superiority in being at all times, combined with the best antiaircraft ground devices, is the only form of defense that offers any security whatever, and it must continue to be an essential part of our security program for a long time to come."\footnote{\textit{Ibid}., p. 68.} But it must be added that the ``only form of defense that offers any security whatever" falls far short, even without any consideration of rockets, of offering the already qualified kind of security it formerly offered.

\begin{enumerate}[resume*]

\item \textbf{Superiority in numbers of bombs is not in itself a guarantee of strategic superiority in atomic bomb warfare.}

\end{enumerate}

\label{I-SuperiorNos1}

Under the technical conditions apparently prevailing today, and presumably likely to continue for some time to come, the primary targets for the atomic bomb will be cities. One does not shoot rabbits with elephant guns, especially if there are elephants available. The critical mass conditions to which the bomb is inherently subject place the minimum of destructive energy of the individual unit at far too high a level to warrant its use against any target where enemy strength is not already densely concentrated. Indeed, there is little inducement to the attacker to seek any other kind of target. If one side can eliminate the cities of the other, it enjoys an advantage which is practically tantamount to final victory, provided always its own cities are not similarly eliminated.

The fact that the bomb is inevitably a weapon of indiscriminate destruction will carry no weight in any war in which it is used. Even in World War II, in which the bombs used could to a large extent isolate industrial targets from residential districts within an urban area, the distinctions imposed by international law between ``military" and ``non-military" targets disintegrated entirely.\footnote{This was due in part to deliberate intention, legally permitted on the Allied side under the principle of retaliation, and in part to a desire of the respective belligerents to maximize the effectiveness of the air forces available to them. ``Precision bombing" was always a misnomer, though some selectivity of targets was possible in good weather. However, such weather occurred in Europe considerably less than half the time, and if the strategic air forces were not to be entirely grounded during the remaining time they were obliged to resort to ``area bombing". Radar, when used, was far from being a substitute for the human eye.}

How large a city has to be to provide a suitable target for the atomic bomb will depend on a number of variables - the ratio of the number of bombs available to the number of cities which might be hit, the wastage of bombs in respect to each target, the number of bombs which the larger cities can absorb before ceasing to be profitable targets, and, of course, the precise characteristics and relative accessibility of the individual city. Most important of all is the place of the particular city in the nation's economy. We can see at once that it does not require the obliteration of all its towns to make a nation wholly incapable of defending itself in the traditional fashion. Thus, the number of \emph{critical} targets is quite limited, and the number of hits necessary to win a strategic decision - always excepting the matter of retaliation - is correspondingly limited. That does not mean that additional hits would be useless but simply that diminishing returns would set in early; and after the cities of say 100,000 population were eliminated the returns from additional bombs expended would decline drastically.

We have seen that one has to allow for wastage of missiles in warfare, and the more missiles one has the larger the degree of wastage which is acceptable. Moreover, the number of bombs available to a victim of attack will always bear to an important degree on his ability to retaliate, though it will not itself determine that ability. But, making due allowance for these considerations, it appears that for any conflict a specific number of bombs will be useful to the side using it, and anything beyond that will be luxury. What that specific number would be for any given situation it is wholly impossible to determine. But we can say that if 2,000 bombs in the hands of either party is enough to destroy entirely the economy of the other, the fact that one side has 6,000 and the other 2,000 will be of relatively small significance.

We cannot, of course, assume that if a race in atomic bombs develops each nation will be content to limit its production after it reaches what it assumes to be the critical level. That would in fact be poor strategy, because the actual critical level could never be precisely determined in advance and all sorts of contingencies would have to be provided for. Moreover, nations will be eager to make whatever political capital (in the narrowest sense of the term) can be made out of superiority in numbers. But it nevertheless remains true that superiority in numbers of bombs does not endow its possessor with the kind of military security which formerly resulted from superiority in armies, navies, and air forces.

\label{I-SuperiorNos2}

\begin{enumerate}[resume*]

\item \textbf{The new potentialities which the atomic bomb gives to sabotage, must not be overrated.}

\end{enumerate}

With ordinary explosives it was hitherto physically impossible for agents to smuggle into another country, either prior to or during hostilities, a sufficient quantity of materials to blow up more than a very few specially chosen objectives. The possibility of really serious damage to a great power resulting from such enterprises was practically nil. A wholly new situation arises, however, where such materials as U-235 or Pu-239 are employed, for only a few pounds of either substance is sufficient, when used in appropriate engines, to blow up the major part of a large city. Should those possibilities be developed, an extraordinarily high premium will be attached to national competence in sabotage on the one hand and in counter-sabotage on the other. The F.B.I. or its counterpart would become the first line of national defense, and the encroachment on civil liberties which would necessarily follow would far exceed in magnitude and pervasiveness anything which democracies have thus far tolerated in peacetime.

However, it would be easy to exaggerate the threat inherent in that situation, at least for the present. From various hints contained in the \emph{Smyth Report}\footnote{Henry D. Smyth, \textit{Atomic Energy for Military Purposes; The Official Report on the Development of the Atomic Bomb under the Auspices of the United States Government, 1940-1945}, Princeton University Press, paragraphs 12.9-12.22.} and elsewhere,\footnote{General Arnold, for example, in his \textit{Third Report to the Secretary of War} asserted that at present the only effective means of delivering the atomic bomb is the ``very heavy bomber." See printed edition, p. 68.} it is clear that the engine necessary for utilizing the explosive, that is, the bomb itself, is a highly intricate and fairly massive mechanism. The massiveness is not something which we can expect future research to diminish. It is inherent in the bomb. The mechanism and casing surrounding the explosive element must be heavy enough to act as a ``tamper", that is, as a means of holding the explosive substance together until the reaction has made substantial progress. Otherwise the materials would fly apart before the reaction was fairly begun. And since the \emph{Smyth Report} makes it clear that it is not the tensile strength of the tamper but the inertia due to mass which is important, we need expect no particular assistance from metallurgical advances.\footnote{One might venture to speculate whether the increase in power which the atomic bomb is reported to have undergone since it was first used is not due to the use of a more massive tamper to produce a more complete reaction. If so, the bomb has been increasing in weight rather than the reverse.}

The designing of the bomb apparently involved some of the major problems of the whole ``Manhattan District" project. The laboratory at Los Alamos was devoted almost exclusively to solving those problems, some of which for a time looked insuperable. The former director of that laboratory has stated that the results of the research undertaken there required for its recording a book of some fifteen volumes.\footnote{Robert J. Oppenheimer, \emph{loc}. \emph{cit}., p. 9.} The detonation problem is not even remotely like that of any other explosive. It requires the bringing together instantaneously in perfect union of two or more subcritical masses of the explosive material (which up to that moment must be insulated from each other) and the holding together of the combined mass until a reasonable proportion of the uranium or plutonium atoms have undergone fission. A little reflection will indicate that the mechanism which can accomplish this must be ingenious and elaborate in the extreme, and certainly not one which can be slipped into a suit case.

It is of course possible that a nation intent upon perfectIng the atomic bomb as a sabotage instrument could work out a much simpler device. Perhaps the essential mechanism could be broken down into small component parts such as are easily smuggled across national frontiers, the essential mass being provided by crude materials available locally in the target area. Those familiar with the present mechanism do not consider such an eventuation likely. And if it required the smuggling of whole bombs, that too is perhaps possible. But the chances are that if two or three were successfully introduced into a country by stealth, the fourth or fifth would be discovered. Our federal police agencies have made an impressive demonstration in the past, with far less motivation, of their ability to deal with smugglers and saboteurs.

Those, at any rate, are some of the facts to consider when reading a statement such as Professor Harold Urey was reported to have made: ``An enemy who put twenty bombs, each with a time fuse, into twenty trunks, and checked one in the baggage room of the main railroad station in each of twenty leading American cities, could wipe this country off the map so far as military defense is concerned."\footnote{The \textit{New Republic}, December 31, 1945, p. 885. The statement quoted is that used by the \textit{New Republic}, and is probably not identical in wording with Prof. Urey's remark.} Quite apart from the question of whether twenty bombs, even if they were considerably more powerful than those used at Hiroshima and Nagasaki, could produce the results which Professor Urey assumes they would, the mode of distribution postulated is not one which recommends itself for aggressive purposes. For the detection of one or more of the bombs would not merely compromise the success of the entire project but would give the intended victim the clearest and most blatant warning imaginable of what to expect and prepare for. Except for port cities, in which foreign ships are always gathered, a surprise attack by air is by every consideration a handier way of doing the job.

\begin{enumerate}[resume*]

\item \textbf{In relation to the destructive powers of the bomb, world resources in raw materials for its production must be considered abundant.}

\end{enumerate}

Everything about the atomic bomb is overshadowed by the twin facts that it exists and that its destructive power is fantastically great. Yet within this framework there are a large number of technical questions which must be answered if our policy decisions are to proceed in anything other than complete darkness. Of first importance are those relating to its availability.

The manner in which the bomb was first tested and used and various indications contained in the \emph{Smyth Report} suggest that the atomic bomb cannot be ``mass produced" in the usual sense of the term. It is certainly a scarce commodity in the sense in which the economist uses the term ``scarcity", and it is bound to remain extremely scarce in relation to the number of TNT or torpex bombs of comparable size which can be produced. To be sure, the bomb is so destructive that even a relatively small number (as compared with other bombs) may prove sufficient to decide a war, especially since there will be no such thing as a ``near miss" - anything near will have all the consequences of a direct hit. However, the scarcity is likely to be sufficiently important to dictate the selection of targets and the circumstances under which the missile is hurled.

A rare explosive will not normally be used against targets which are naturally dispersed or easily capable of dispersion, such as ships at sea or isolated industrial plants of no great magnitude. Nor will it be used in types of attack which show an unduly high rate of loss among the attacking instruments - unless, as we have seen, the target is so important as to warrant high ratios of loss provided one or a few missiles penetrate to it. In these respects the effects of scarcity in the explosive materials are intensified by the fact that it requires certain minimum amounts to produce an explosive reaction and that the minimum quantity is not likely to be reduced materially, if at all, by further research.\footnote{The figure for critical minimum mass is secret. According to the \textit{Smyth Report}, it was predicted in May 1941 that the critical mass would be found to lie between 2 kg and 100 kg (paragraph 4.49), and it was later found to be much nearer the minimum predicted than the maximum. It is worth noting, too, that not only does the critical mass present a lower limit in bomb size, but also that it is not feasible to use very much more than the critical mass. One reason is the detonating problem. Masses above the critical level cannot be kept from exploding, and detonation is therefore produced by the instantaneous assembly of subcritical masses. The necessity for \emph{instant and simultaneous} assembly of the masses used must obviously limit their number. The scientific explanation of the critical mass condition is presented in the \textit{Smyth Report} in paragraphs 2.3, 2.6, and 2.7. One must always distinguish, however, between the chain reaction which occurs in the plutonium-producing pile and that which occurs in the bomb. Although the general principles determining critical mass are similar for the two reactions, the actual mass needed and the character of the reaction are very different in the two cases. See also \textit{ibid}., paragraphs 2.35, 4.15-17, and 12.13-15.}

The ultimate physical limitation on world atomic bomb production is of course the amount of ores available for the derivation of materials capable of spontaneous atomic fission. The only basic material thus far used to produce bombs is uranium, and for the moment only uranium need be considered

Estimates of the amount of uranium available in the earth's crust vary between 4 and 7 parts per million - a very considerable quantity indeed. The element is very widely distributed, there being about a ton of it present in each cubic mile of sea water and about one-seventh of an ounce per ton (average) in all granite and basalt rocks, which together comprise about 95 per cent by weight of the earth's crust. There is more uranium present in the earth's crust than cadmium, bismuth, silver, mercury, or iodine, and it is about one thousand times as prevalent as gold. However, the number of places in which uranium is known to exist in concentrated form is relatively small, and of these places only four are known to have the concentrated deposits in substantial amounts. The latter deposits are found in the Great Bear Lake region of northern Canada, the Belgian Congo, Colorado, and Joachimsthal in Czechoslovakia. Lesser but nevertheless fairly extensive deposits are known to exist also in Madagascar, India, and Russian Turkestan, while small occurrences are fairly well scattered over the globe.\footnote{See ``The Distribution of Uranium in Nature," an unsigned article published in the \textit{Bulletin of the Atomic Scientists of Chicago}, No. 4 (Feb. 1, 1946), p. 6. See also U.S. Bureau of Mines: \textit{Minerals Yearbook, 1940}, p. 766; \textit{ibid., 1943}, p. 828; H. V. Ellsworth: \textit{Rare Element Minerals in Canada}, Geological Survey of Canada, 1932, p. 39.}

The pre-war market was dominated by the Belgian Congo and Canada, who agreed in 1939 to share it in the ratio of 60 to 40,\footnote{\textit{Minerals Yearbook, 1939}, p. 755.} a proportion which presumably reflected what was then thought to be their respective reserves and productive capacity. However, it now appears likely that the Canadian reserves are considerably greater than those of the Congo. In 1942 the Congo produced 1,021 tons of unusually rich ore containing 695.6 tons of U$_3$O$_8$ - or about 590 tons of uranium metal.\footnote{\textit{Ibid}., p. 828. See also A. W. Postel, \textit{The Mineral Resources of Africa}, University of Pennsylvania, 1943, p. 44.} In general, however, the ores of Canada and the Congo are of a richness of about one ton of uranium in from fifty to one hundred tons of ore. The Czechoslovakian deposits yielded only fifteen to twenty tons of uranium oxide (U$_3$O$_8$) annually before the war.\footnote{\textit{The Mineral Industry of the British Empire and Foreign Countries, Statistical Summary, 1935-37}, London, 1938, p. 419.} This rate of extraction could not be very greatly expanded even under strained operations - since the total reserves of the Joachimsthal region are far smaller than those of the Congo or Canada or even Colorado.

The quantity of U-235 in pre-metallic uranium is only about 0.7 per cent (or 1/140th) of the whole. To be sure, plutonium-239, which is equally as effective in a bomb as U-235, is derived from the more plentiful U-238 isotope, but only through a chain reaction that depends on the presence of U-235, which is broken down in the process. It is doubtful whether a given quantity of uranium can yield substantially more plutonium than U-235.\footnote{The \emph{Smyth Report} is somewhat misleading on this score, in that it gives the impression that the use of plutonium rather than U-235 makes it possible to utilize 100 per cent of the U-238 for atomic fission energy. See paragraphs 2.26 and 4.25. However, other portions of the same report give a more accurate picture, especially paragraphs 8.18 and 8.72-73.} It appears also from the \emph{Smyth Report} that the amount of U-235 which can profitably be extracted by separation of the isotopes is far below 100 per cent of the amount present, at least under present techniques.\footnote{Among numerous other hints is the statement that in September 1942 the plants working on the atomic bomb were already receiving about one ton daily of uranium oxide of high purity (paragraph 6.11). Making the conservative assumption that this figure represented the minimum quantity of uranium oxide being processed daily during 1944-45, the U-235 content would be about 115 pounds. The actual figure of production is still secret, but from all available indices the daily production of U-235 and Pu-239 is even now very considerably below that amount.}

What all these facts add up to is perhaps summarized by the statement made by one scientist that there is a great deal more than enough fissionable material in known deposits to blow up all the cities in the world, though he added that there might not be enough to do so if, the cities were divided and dispersed into ten times their present number (the size of cities included in that comment was not specified). Whatever solace that statement may bring is tempered by the understanding that it refers to \emph{known} deposits of \emph{uranium} ores only and assumes no great increase in the efficiency of the bombs. But how are these factors likely to change?

It is hardly to be questioned that the present extraordinary military premium on uranium will stimulate intensive prospecting and result in the discovery of many new deposits. It seems clear that some of the prospecting which went on during the war was not without result. The demand for uranium heretofore has been extremely limited and only the richer deposits were worth working - mainly for their vanadium or radium content - or for that matter worth keeping track of.\footnote{``Material for U-235", \textit{The Economist} (London), November 3, 1945, pp. 629-30.} So far as uranium itself was concerned, no encouragement for prospecting existed.

It is true that the radioactivity of uranium affords a very sensitive test of its presence, and that the data accumulated over the last fifty years make it appear rather unlikely that wholly new deposits will be found comparable to those of Canada or the Congo. But it is not unlikely that in those regions known to contain uranium, further exploration will reveal much larger quantities than had previously been suspected. It seems hardly conceivable, for example, that in the great expanse of European and Asiatic Russia no additional workable deposits will be discovered.

In that connection it is worth noting that the cost of mining the ore and of extracting the uranium is so small a fraction of the cost of bomb production that (as is \emph{not} true in the search for radium) even poorer deposits are decidedly usable. Within certain wide limits, in other words, the relative richness of the ore is not critical. In fact, as much uranium can be obtained as the nations of the world really desire. Gold is commonly mined from ores containing only one-fifth of an ounce per ton of rock, and there are vast quantities of granite which contain from one-fifth to one ounce of uranium per ton of rock.

Although the American experiment has thus far been confined to the use of uranium, it should be noted that the atoms of thorium and protoactinium also undergo fission when bombarded by neutrons. Protoactinium can be eliminated from consideration because of its scarcity in nature, but thorium is even more plentiful than uranium, its average distribution in the earth's crust being some twelve parts per million. Fairly high concentrations of thorium oxide are found in monazite sands, which exist to some extent in the United States, Ceylon, and the Netherlands East Indies, but to a much greater extent in Brazil and British India. The \emph{Smyth Report} states merely that thorium has ``no apparent advantage over uranium" (paragraph 2.21), but how important are its disadvantages is not stated. At any rate, it has been publicly announced that thorium is already being used in a pilot plant for the production of atomic energy set up in Canada.\footnote{\textit{New York Herald Tribune}, December 18, 1945, p. 4. Incidentally, the Canadian pile is the first one to use the much-discussed ``heavy-water" (which contains the heavy hydrogen or deuterium atom) as a moderator in place of the graphite (carbon) used in the American piles.}

In considering the availability of ores to particular powers, it is always necessary to bear in mind that accessibility is not determined exclusively by national boundaries. Accessibility depends on a combination of geographic, political, and power conditions and on whether the situation is one of war or peace. During wartime a great nation will obviously enjoy the ore resources both of allied countries and of those territories which its armies have overrun, though in the future the ores made available only after the outbreak of hostilities may not he of much importance. Because of the political orientation of Czechoslovakia towards the Soviet Union, the latter will most likely gain in peacetime the use of the Joachimsthal ores,\footnote{However, Mr. Jan Masaryk, Czechoslovak Foreign Minister, asserted in a speech before the Assembly of the U.N.O. on January 17, 1946 that ``no Czechoslovak uranium will be used for destructive purposes." \textit{New York Times}, January 18, 1946, p. 8.} just as the United States enjoys the use of the immensely richer deposits of Canada. The ores of the Belgian Congo will in peacetime be made available to those countries which can either have the confidence of or coerce the Belgian Government (unless the matter is decided by an international instrument to which Belgium is a party); in a time of general war the same ores would be controlled by the nation or nations whose sea and air power gave them access to the region.

Since the atoms of both U-235 and Pu-239 are normally extremely stable (in technical language: possess a long ``half-life" ), subcritical masses of either material may be stored practically indefinitely. Thus, even a relatively slow rate of production can result over a period of time in a substantial accumulation of bombs. But how slow need the rate of production be? The process of production itself is inevitably a slow one, and even with a huge plant it would require perhaps several months of operation to produce enough fissionable material for the first bomb. But the rate of output thereafter depends entirely on the extent of the facilities devoted to production, which in turn could be geared to the amount of ores being made available for processing. The eminent Danish scientist, Niels Bohr, who was associated with the atomic bomb project, was reported as having stated publicly in October 1945 that the United States was producing three kilograms (6.6 pounds) of U-235 daily.\footnote{\emph{Time}, October 15, 1945, p. 22.} The amount of plutonium being concurrently produced might well be considerably larger. Dr. Harold C. Urey, also a leading figure in the bomb development, considers it not unreasonable to assume that with sufficient effort 10,000 bombs could be produced,\footnote{\textit{New York Times}, October 22, 1945, p. 4.} and other distinguished scientists have not hesitated to put the figure considerably higher. Thus, while the bomb may remain, for the next fifteen or twenty years at least, scarce enough to dictate to its would-be users a fairly rigorous selection of targets and means of delivery, it will not be scarce enough to spare any nation against which it is used from a destruction immeasurably more devastating than that endured by Germany in World War II.

It is of course tempting to leave to the physicist familiar with the bomb all speculation concerning its future increase in power. However, the basic principles which must govern the developments of the future are not difficult to comprehend, and it is satisfying intellectually to have some basis for appraising in terms of probability the random estimates which have been presented to the public. Some of those estimates, it must be said, though emanating from distinguished scientists, are not marked by the scientific discipline which is so rigorously observed in the laboratory. Certainly they cannot be regarded as dispassionate. It might therefore be profitable for us to examine briefly (a) the relation of increase in power to increase of destructive capacity, and (b) the several factors which must determine the inherent power of the bomb. As we have seen, the radius of destruction of a bomb increases only as the third root of the explosive energy released. Thus, if Bomb A has a radius of total destruction of one mile, it would take a bomb of 1,000 times the power (Bomb B) to have a radius of destruction of ten miles.\footnote{Since the Hiroshima bomb had a radius of total destruction of something under 1-1/4 wiles, its power would have to be increased by some 600 times to gain the hypothetical ten mile radius.} In terms of area destroyed the proportion does not look so bad; nevertheless the \emph{area} destroyed by Bomb B would be only 100 times as great as that destroyed by Bomb A. In other words, the ratio of destructive efficiency to energy released would be only one-tenth as great in Bomb B as it is in Bomb A. But when we consider also the fact that the area covered by Bomb B is bound to include to a much greater degree than Bomb A sections of no appreciable military significance (assuming both bombs are perfectly aimed), the military efficiency of the bomb falls off even more rapidly with increasing power of the individual unit than is indicated above.\footnote{The bomb of longer destructive radius would of course not have to be aimed as accurately for any given target; and this fact may prove of importance in very long range rocket fire, which can never be expected to be as accurate as bombing from airplanes. But here again, large numbers of missiles will also make up for the inaccuracy of the individual missile.} What this means is that even if it were technically feasible to accomplish it, an increase in the power of the bomb gained only by a proportionate increase in the mass of the scarce and expensive fissionable material within it would be very poor economy. It would be much bettor to use the extra quantities to make extra bombs.

It so happens, however, that in atomic bombs the total amount of energy released per kilogram of fissionable material (i.e., the efficiency of energy release) \emph{increases} with the size of the bomb.\footnote{\emph{Smyth Report}, paragraph 2:18. This phenomenon is no doubt due to the fact that the greater the margin above the critical mass limit, the \emph{faster} the reaction and hence the greater the proportion of material which undergoes fission before the heat generated expands and disrupts the bomb. It might be noted also that even if there were no expansion or bursting to halt it, the reaction would cease at about the time the fissionable material remaining fell below critical mass conditions, which would also tend to put a premium on having a large margin above critical mass limits. At any rate, anything like 100 per cent detonation of the explosive contents of the atomic bomb is totally out of the question. In this respect atomic explosives differ markedly from ordinary ``high explosives" like TNT or torpex, where there is no difficulty in getting a 100 per cent reaction and where the energy released is therefore directly proportionate to the amount of explosive filler in the bomb.} This factor, weighed against those mentioned in the previous paragraph, indicates that there is a theoretical optimum size for the bomb which has perhaps not yet been determined and which may very well be appreciably or even considerably larger than the Nagasaki bomb. But it should be observed that considerations of military economy are not the only factors which hold down the optimum size. One factor, already noted, is the steeply ascending difficulty as the number of subcritical masses increases of securing simultaneous and perfect union among them. Another is the problem of the envelope or tamper. If the increase of weight of the tamper is at all proportionate either to the increase in the amount of fissionable material used or to the amount of energy released, the gross weight of the bomb might quickly press against the technically usable limits. In short, the fact that an enormous increase in the power of the bomb is theoretically conceivable does not mean that it is likely to occur, either soon or later. It has always been theoretically possible to pour 20,000 tons of TNT together in one case and detonate it as a single bomb; but after some forty years or more of its use, the largest amount of it poured into a single lump was about six tons.\footnote{In the 10-ton bomb, of which it is fair to estimate that at least 40 per cent of the weight must be attributed to the metal case. In armor-piercing shells and bombs the proportion of weight devoted to metal is very much higher, running above the 95 per cent mark in major-caliber naval shells.}

To be sure, greater power in the bomb will no doubt be attained by increasing the efficiency of the explosion without necessarily adding to the quantities of fissionable materials used. But the curve of progress in this direction is bound to flatten out and to remain far short of 100 per cent. The bomb is, to be sure, in its ``infancy", but that statement is misleading if it implies that we may expect the kind of progress which we have witnessed over the past century in the steam engine. The bomb is new, but the people who developed it were able to avail themselves of the fabulously elaborate and advanced technology already existing. Any new device created today is already at birth a highly perfected instrument.

One cannot dismiss the matter of increasing efficiency of the bomb without noting that the military uses of radio-activity may not be confined to bombs. Even if the project to produce the bomb had ultimately failed, the by-products formed from some of the intermediate processes could have been used as an extremely vicious form of poison gas. It was estimated by two members of the ``Manhattan District" project that the radioactive by-products formed in one day's run of a 100,000 kW chain-reacting pile for the production of plutonium (the production rate at Hanford, Washington was from five to fifteen times as great) might be sufficient to make a large area uninhabitable.\footnote{\emph{Smyth Report}, paragraphs 4.26-28.} Fortunately, however, materials which are dangerously radioactive tend to lose their radioactivity rather quickly and therefore cannot be stored.

\begin{enumerate}[resume*]

\item \textbf{Regardless of American decisions concerning retention of its present secrets, other powers besides Britain and Canada will be producing the bombs in quantity in a period of five to ten years.}

\end{enumerate}

This proposition of course ignores the possibility of effective regulation of bomb production being imposed by international action within such time period. A discussion of that possibility is left to subsequent chapters. One may anticipate that discussion, however, to the extent of pointing out that there is little to induce nations like the Soviet Union or France to agree to such regulation until they can start out on a position of parity with the United States - parity not alone in bombs but in ability to produce the bomb. In any case, what we are primarily concerned with in the present discussion is not whether other nations will actually be producing the bomb but whether they will be in a position to do so if they choose.

Statements of public officials and of journalists indicate an enormous confusion concerning the extent and character of the secret now in the possession of the United States. Opinions vary from the observation that ``there is no secret" to the blunt comment of Dr. Walter R. G. Baker, Vice-President of the General Electric Company, that no nation other than the United States has sufficient wealth, materials, and industrial resources to produce the bomb.\footnote{\textit{New York Times}, October 2, 1945, p. 6.}

Some clarification is discernible in President Truman' s message to Congress of October 3, 1945, in which the President recommended the establishment of security regulations and the prescription of suitable penalties for their violation and went on to add the following: ``Scientific opinion appears to be practically unanimous that the essential theoretical knowledge upon which the discovery is based is already widely known. There is also substantial agreement that foreign research can come abreast of our present theoretical knowledge in time." The emphasis, it should be noted, is on ``theoretical knowledge." A good deal of basic scientific data is still bound by rigorous secrecy, but such data is apparently not considered to be crucial. While the retention of such secrets would impose upon the scientists of other nations the necessity of carrying through a good deal of time-consuming research which would merely duplicate that already done in this country, there seems to be little question that countries like the Soviet Union and France and probably several of the lesser nations of Europe have the resources in scientific talent to accomplish it. It is (a) the technical and engineering details of the manufacturing process for the fissionable materials and (b) the design of the bomb itself which are thought to be the critical hurdles.

At a public meeting in Washington on December 11, 1945, Major General Leslie R. Groves permitted himself the observation that the bomb was not a problem for us but for our grandchildren. What he obviously intended that statement to convey was the idea that it would take other nations, like Russia, many years to duplicate our feat. When it was submitted to him that the scientists who worked on the problem were practically unanimous in their disagreement, he responded that they did not understand the problem. The difficulties to be overcome, he insists, are not primarily of a scientific but of an engineering character. And while the Soviet Union may have first-rate scientists, it clearly does not have the great resources in engineering talent or the industrial laboratories that we enjoy.

Perhaps no; but there are a few pertinent facts which bear on such a surmise. First of all, it has always been axiomatic in the armed services that the only way really to keep a device secret is to keep the fact of its existence secret. Thus, the essential basis of secrecy of the atomic bomb disappeared on August 6, 1945. But the same day saw the release of the \emph{Smyth Report}, which was subsequently published in book form and widely distributed. Members of the War Department who approved its publication, including General Groves himself, insist that it reveals nothing of importance. But scientists close to the project point out that the \emph{Smyth Report} reveals substantially everything that the American and associated scientists themselves knew up to the close of 1942. It in fact tells much of the subsequent findings as well. In any case, from the end of 1942 it was only two and one-half yours before we had the bomb.

The \emph{Smyth Report} reveals among other things that five distinct and separate processes for producing fissionable materials were pursued, and that \emph{all were} successful. These involved four processes for the separation of the U-235 isotope from the more common forms of uranium and one basic process for the production of plutonium. One of the isotope separation processes, the so-called ``centrifuge process," was never pushed beyond the pilot plant stage, but it was successful as far as it was pursued. It was dropped when the gaseous diffusion and electromagnetic methods of isotope separation promised assured success.\footnote{See \emph{Smyth Report}, chaps. vii-xi, also paragraph 5.21.} The thermal diffusion process was restricted to a small plant. \emph{But any of these processes would have sufficed to produce the fissionable materials for the bomb}. Each of these processes presented problems for which generally multiple rather than single solutions were discovered. Each of them, furthermore, is described in the report in fairly revealing though general terns. Finally, the report probably reveals enough to indicate to the careful reader which of the processes presents the fewest problems and offers the most profitable yield. Another nation wishing to produce the bomb can confine its efforts to that one process or to some modification of it.

Enough is said in the \emph{Smyth Report} about the bomb itself to give one a good idea of its basic character. Superficially at least, the problem of bomb design seems a bottleneck, since the same bomb is required to handle the materials produced by any of the five processes mentioned above. But that is like saying that while gasoline can be produced in several different ways, only one kind of engine can utilize it effectively. The bomb is gadgetry, and it is a commonplace in the history of technology that mechanical devices of radically different design have been perfected to achieve a common end. The machine gun has several variants which operate on basically different principles, and the same is no doubt true of dish washing machines.

Some of those who were associated with the bomb design project came away tremendously impressed with the seemingly insuperable difficulties which were overcome. Undoubtedly they were justified in their admiration for the ingenuity displayed. But they are not justified in assuming that aggregations of talented young men in other parts of the world could not display equally brilliant ingenuity. A high-ranking naval officer, who was associated with the Los Alamos Laboratory, in an effort at a recent public meeting to impress his audience with the scale of the obstacles which will beset any other nation that attempts to make a bomb, reported that one particularly trying problem was overcome only because one scientist happened to misunderstand another. It must be submitted that the United States can hardly base its security on the supposition that scientists abroad will be unable to misunderstand each other.

We cannot assume that what took us two and one-half years to accomplish, without the certainty that success was possible, should take another great nation twenty to thirty years to duplicate with the full knowledge that the thing has been done. To do so would be to exhibit an extreme form of ethnocentric smugness. It is true that we mobilized a vast amount of talent, but American ways are frequently wasteful.

We were simultaneously pushing forward on a great many other scientific and engineering fronts having nothing to do with the atomic bomb. Another nation which has fewer engineers and scientists than we have could nevertheless, by concentrating all its pertinent talent on this one job - and there is plenty of motivation - marshal as great a fund of scientific and engineering workers as it would need, perhaps as much as we did. The Japanese, for example, before the recent war, were intent on having a good torpedo, and by concentrating on that end produced a superb torpedo, though they had to accept inferiority to us in practically every other aspect of naval ordnance. One should expect a similar concentration in other countries on the atomic bomb, and one should expect also comparable results.

It is clear also that the money cost is no barrier to any nation worthy the name. The two billion dollars which the bomb development project cost the United States must be considered small for a weapon of such extraordinary military power. Moreover, that sum is by no means the measure of what a comparable development would cost other nations. The American program was pushed during wartime under extreme urgency and under war-inflated prices. Money costs were always considered secondary to the saving of time. The scientists and engineers who designed the plants and equipment were constantly pushing into the unknown. The huge plant at Hanford, Washington for the production of plutonium, for example, was pushed forward on the basis of that amount of knowledge of the properties of the new element which could be gleaned from the study of half a milligram in the laboratories at Chicago.\footnote{\emph{Smyth Report}, paragraph 7.3. A milligram is a thousandth of a gram (one United States dime weighs 2-1/2 grams). See also \emph{ibid}., paragraphs 5.21, 7.43, 8.1, 8.26, and 9.13.} Five separate processes for the production of fissionable materials were pushed concurrently, for the planners had to hedge against the possibility of failure in one or more. There was no room for weighing the relative economy of each. Minor failures and fruitless researches did in fact occur in each process.

It is fairly safe to say that another country, proceeding only on the information available in the \emph{Smyth Report}, would be able to reach something comparable to the American production at less than half the cost - even if we adopt the American price level as a standard. Another country would certainly be able to economize by selecting one of the processes and ignoring the others - no doubt the plutonium production process, since various indices seem to point clearly to its being the least difficult and the most rewarding one - an impression which is confirmed by the public statements of some scientists.\footnote{Dr. J. R. Dunning, Director of Columbia University's Division of War Research and a leading figure in the research which led to the atomic bomb declared before the American Institute of Electrical Engineers that improvements in the plutonium producing process ``have already made the extensive plants at Oak Ridge technically obsolete." \textit{New York Times}, January 24, 1946, p. 7. The large Oak Ridge plants are devoted almost exclusively to the isotope separation processes.} General Groves has revealed that about one-fourth of the entire capital investment in the atomic bomb went into the plutonium production project at Hanford.\footnote{The Hanford, Washington plutonium plant is listed as costing \$350,000,000, and housing for workers at nearby Richland cost an additional \$48,000,000. This out of a total country-wide capital investment, including housing, of \$1,595,000,000. The monthly operating cost of the Hanford plant is estimated at \$3,500,000, as compared with the \$6,000,000 per month for the diffusion plant at Oak Ridge and \$12,000,000 for the electro-magnetic plant, also at Oak Ridge. These figures have, of course, little meaning without some knowledge of the respective yields at the several plants, but it may be significant that in the projection of future operating costs, nothing is said about Hanford. According to General Groves the operating costs of the electro-magnetic plant will diminish, while those of the gaseous diffusion plant will increase only as a result of completion of plant enlargement. Of course, the degree to which less efficient processes were cut back and more efficient ones expanded would depend on considerations of existing capital investment and of the desired rate of current production.} As fuller information seeps out even to the public, as it inevitably will despite security regulations, the signs pointing out to other nations the more fruitful avenues of endeavor will become more abundant. Scientists may be effectively silenced, but they cannot as a body be made to lie. And so long as they talk at all, the hiatuses in their speech may be as eloquent to the informed listener as the speech itself.
