
\chapter[International Control of Atomic Weapons]{International Control of Atomic Weapons}

\vspace{-2pt}

\noindent{\normalsize \textbf{William T. R. Fox}}

\vspace{39pt}

From the Second World War all that victory was expected to bring was one more chance to solve the problems so badly mishandled during the inter-war period. Victory itself was not supposed to provide the answers. What victory was not supposed to bring was a new problem dwarfing in importance all those left over from the war itself and the uneasy peace which preceded the war. The experience of 1919 seems to be repeating itself. In 1919, it was an explosive new idea, the Bolshevik idea, which seemed to be threatening the foundations of Western political life. In 1946, it is an explosive new material force, that of atomic energy. The statesmen of the West are as much appalled by the spectre of the atomic bomb as were their predecessors of a generation ago by the spectre of Bolshevism.

Traditional ways of playing the diplomatic game seemed pitifully inadequate in 1919 and they seem pitifully inadequate today. To their peoples clamoring for a period of calm after the stormy years of war, the statesmen can only repeat with G. K. Chesterton:

\begin{quote}[poetry]
``No more of comfort shall ye get

Than that the sky grows darker yet,

And the sea rises higher."\footnote{Quoted by Eustace Percy in \textit{The Responsibilities of the League}. London, Hodder and Stoughton, 1919, p. 111, when writing of the alleged menace of Communism after the First World War.}
\end{quote}

``The hope of civilization", President Truman has declared, "lies in international arrangements looking, if possible, to the renunciation of the use of the atomic bomb."\footnote{Message to Congress on atomic energy, October 3, 1945.} Many would go further and say that such a revolutionary development in war technology demands a revolutionary change in the organization of peace. Nothing less than the creation of a world authority strong enough to enforce its will even against the greatest states would, they say, abate the menace to mankind of the atomic bomb.

The case for world government right now may in fact at first glance seem impressive. Mankind will pay a terrible price if its leaders make the wrong choices in their efforts to achieve the social control of atomic energy. ``Lack of decision within even a few months", according to one group of nuclear physicists, ``will be preparing the world for unprecedented destruction, not only of other countries but of our own as well."\footnote{Statement issued by Association of Los Alamos Scientists, October 13, 1945. \textit{New York Times}, October 14, 1945.} Does ``world government right now" provide the only intelligent goal around which men of good will who seek to prevent the total destruction of civilization can now unite? Is it in the realm of human affairs the invention which is the counterpart of the atomic bomb in the realm of science? The frantic casting about by the leaders of the great states for some lesser solution and the apparent inadequacy of all such solutions so far suggested would seen to point to an affirmative answer to these questions.

Unfortunately for those who believe that a program of mass education is all that is necessary to make world government right now feasible, that high goal is right now or in the near future impossible of achievement. Even Anthony Eden, who believes that discoveries about atomic energy have made the great-power veto provisions of the United Nations Charter an anachronism, confesses that ``It is yet true that national sentiment is still as strong as ever, and here and there it is strengthened by this further complication - the differing conceptions of forms of government and differing conceptions of what words like freedom and democracy mean."\footnote{Speech in House of Commons, November 22, 1945.} What Mr. Eden means is that neither the Soviet Union nor Great Britain is now ready to surrender its sovereignty to a world authority which might be dominated by the political beliefs of the other. The United States and the other non-Communist states of the world are not ready either, but it is probably the Soviet Union which is and will remain the most adamant in opposing a general surrender of sovereignty to a world authority. It is that country which would be most likely to be outvoted in a parliament of the world.\footnote{When the United Nations Conference on International Organization voted in plenary session to invite Argentina to send a delegation, there was a preliminary show of voting strength as between the United States and the Soviet Union. The vote was 31 in support of the American position and 4 in support of the Soviet position. United Nations Conference on International Organization, \textit{Verbatim Minutes of the Fifth Plenary Session}, April 30, 1945.}

It is therefore not surprising that Soviet comments on the idea of setting up a world government in the near future pour scorn and sarcasm on the proposal. Thus, one Soviet commentator, in writing about those who dare to advocate that the Soviet Union along with other nations should yield up sovereignty, declares: ``At present they are not only talking about a United States of Europe but also a United States of the world, a world parliament, a world government and so forth. Fine phrases, and behind them renunciation of the basis of the struggle against fascist aggression and of what is the foundation of the struggle for a stable peace."\footnote{J. Viktoroff, Soviet radio commentator, quoted in the \textit{New York Times}, December 4, 1945.} Ambassador Gromyko, Russian delegate at the London meeting of the General Assembly of the United Nations, spoke against ``voices... heard from somewhere stating that the Charter had already become obsolete and needs revision."\footnote{\textit{United Nations News}, February, 1946, p. 2.} Evidently, no voluntary yielding of authority to a world government is to be expected from the Soviet leadership at this stage in world history.

According to Clarence Streit and the advocates of ``Union Now", there is no need to wait for Soviet Russia; but a world government whose authority did not extend to the Soviet peoples would be no world government at all. It would be an organization of a substantial part of the world which would unquestionably provoke a counter organization of the rest of the world. It would make atomic warfare not less but more likely. The advocate of world government right now is in fact advocating, in the face of the declared Soviet position, that a great power be coerced into making the necessary surrender of sovereignty. This would make atomic warfare not merely likely but almost certain.

It would be ungracious of the writer not to repeat again at this point that, in his judgment, the United States also is unwilling to surrender a degree of control over its own destinies sufficient to permit a world authority to enforce its declared policy against any challenger. The advocates of world government, however, believe that American public opinion can be brought in the very near future to see the necessity of world government. Even on this assumption, the problem would still remain of securing a similar development in the public opinion of other great states. It is too much to expect such a development in those countries in which no organized agitation is permitted against an officially declared public policy and in which the declared policy is reliance upon the principle of voluntary collaboration among the greatest states. The Soviet Union is such a country. World government right now is therefore not a possibility, and there will almost certainly not in the near future be that revolution in world opinion which alone would make it possible.

But would we want world government right now if we could have it. Is it so desirable, or are all alternatives so undesirable that men of good will should concentrate their efforts on that one-in-a-thousand chance that they could soon achieve world government? What prospect would that government have for achieving an equitable settlement of those international disputes which, prior to the advent of the bomb, were felt to be so vital that the nations concerned were willing to settle them by resort to war or by the threat of war? It would be very dangerous to create a machinery of central force before one created a machinery of central justice. For a machinery of central justice to work satisfactorily, its judgments would have to be based upon a world-wide community of values. That community of values does not exist today. To set up a central machinery of force in the present state of the world might be to create a new instrument of coercion which disaffected peoples would come to regard as intolerable.

It may be said in rejoinder that first attempts at world government or world federation will necessarily be imperfect, that the way to develop the community of values is by creating and operating a machinery of central justice. Reference may be made to the experience of the United States first under the Articles of Confederation and later under the Constitution in perfecting its federal system. This nation's experience in perfecting its federal system unfortunately includes the bitter, bloody, and protracted Civil War. Could a world government afford to perfect itself by experiencing a world-wide civil war? Not if it is true that any large-scale war in an era of atomic warfare threatens the whole future of civilization. Unless the world government from the first promises to settle those disputes formerly settled by war so equitably that there will be little or no pressure to resist the enforcement of its decisions, it offers no sure cure against the threatened extinction of civilization; it offers no certainty that other human values besides survival will be protected any better, or indeed as well, as they are protected under the present admittedly unsatisfactory system of regulating international affairs.

It is the threat of general war which provides the excuse for establishing world government now. To substitute the threat of world-wide civil war for the threat of world-wide international war is to make very little progress in atomic energy control. One can only conclude with Secretary of State Byrnes that ``we must not imagine that overnight there can arise fully grown a world government wise and strong enough to protect all of us and tolerant and democratic enough to command our willing loyalty.\footnote{Charleston speech, November 16, 1945.}

There is still another count in the indictment against a program of mass education for ``world government right now". It is frequently and falsely said that at the very worst an attempt to establish world government immediately could do no harm. It can in fact do harm in two ways. It can divert public attention from the urgent necessity of discovering a less simple and less spectacular solution. A slogan as attractive as ``world government right now" can easily become a mass anodyne, excellent for soothing a disturbed public opinion but unfortunately also effective in distracting attention from the imperative quest for another type of solution to the control problem.

The other danger to which the United States and the world may be exposed in the event that American public opinion is brought to believe in the urgent necessity of world government right now is even more serious. If frustrated in their efforts to achieve world government by voluntary agreement, many would come to believe that forcible unification is better than no unification. They would advocate the alternative route to world unity, via imperial conquest. They would proclaim and believe that they were advocating war only because it was made necessary by the unfortunate unwillingness of the leaders of certain states to grasp the compelling necessity for a surrender of sovereignty.\footnote{See Chapter IV, \textit{supra}.} If survival were the only human value and if the political unification of the world offered the only chance of survival, then a good case might be made out for the reorganization of the world under American hegemony. But survival is not the only human value. In spite of all talk in this country of the bomb as ``a sacred trust" which the Almighty in His wisdom has seen fit to give first to the United States, no American really believes that democratic values can be preserved either here or elsewhere in the world if the United States undertakes to unify the world by using or threatening to use the bomb on any recalcitrant.

Evidence has already been cited to show that the voluntary adherence of the Soviet Union to an agreement to set up world government immediately is not to be expected. The advocates of full-fledged war if necessary to establish a central world machinery of coercion would therefore be advocating in reality a Soviet-American war. The prospect of coercing the Soviet Union into acknowledging the authority of a world government is a grim one. It would involve fighting right now the very war which the advocates of world government insist can only be avoided by establishing world government.\footnote{The evidence is by no means clear that such a war would be the twenty-four hour war which its advocates would promise. See Chapter IV, \textit{supra}.}

If the United States did successfully ``blitz" the Soviet Union or some lesser opponent of forcible unification, it would then stand at the bar of world opinion as the only nation which had ever used the atomic bomb and as a nation which had used it in two successive wars. Our critics would frequently point to the fact that it had been used first against a rapidly collapsing foe and second against a foe whose only crime was not to yield to \emph{force majeure} in the form of the bomb. At the moment of victory, the people of the world would be ill-disposed to permit the United States to run the world.

In the face of an aroused and indignant world opinion, the United States government could not in its hour of victory, even if it wished, then afford to surrender its own sovereignty to a new world authority. It would be driven to attempting the unilateral regulation of world affairs. The United States is ill-equipped for such a task. It lacks both the professional army and the experience in colonial administration. World-wide civil war is a possibility in the event of a voluntary political unification of the world. It is a near certainty in the event of its forcible unification.

This much remains to be said in behalf of those who favor world government right now. They are unlikely to be so successful in converting American opinion to their cause that the dangers suggested in the preceding paragraphs will ever materialize. On the other hand, the world government advocates grasped sooner even than its responsible official leadership the salient fact which must dominate any discussion of atomic energy control, namely, that the bomb is not ``just another weapon". In so far as they serve to awaken American opinion to the seriousness of the problem and to prepare the minds of Americans for what must be novel steps in international organization, their propaganda is beneficial. Furthermore, much of the discussion of world government right now will help to focus opinion here and abroad on the question of the ultimate desirability of world government. It by no means follows that all the arguments adduced in this analysis against working to establish world government in the near future have relevance in a long-range program.

\noindent\hfil\rule{0.4\textwidth}{.4pt}\hfil

\vspace{4pt}

If such obvious lines of action as voluntary unification of the world by establishing world government right now, and its sinister alternative, forcible unification of the world by the use of America's atomic might, are to be ruled out, what is left?

There are two rather simple courses of action which are frequently suggested and which need to be briefly examined at this point. These are the ``tell-all" and the ``do-nothing" proposals. The ``tell-all" school urges that retention by the United States alone of the technical knowledge necessary to produce the bomb will make it impossible for the rest of the world to have confidence in American good intentions. Sharing of atomic knowledge is therefore held to be necessary to dispel the clouds of suspicion which prevent the establishment of effective international controls.

In view of the fact that nuclear physicists are practically unanimous in believing that present secrets are destined to be short-lived, the United States would not appear to be giving away very much; the effect of this proposal might be only to advance the date upon which the United States would have to bargain on equal terms with other states in negotiating international control. If it is true that the secrets are not of as great value as is sometimes implied in the American press, giving them away might not make the spectacular impression on skeptical foreign statesmen that proponents promise.

Whether ``telling-all" would be a quixotic gesture or an act of sublime wisdom is, however, almost beside the point. On one point alone has policy crystallized to such an extent that it is unlikely to be affected by further
public discussion. That has been on the necessity for having ``safeguards" \emph{before} making revelations at least of engineering techniques in atomic energy production.\footnote{It has sometimes been argued that the spirit of free scientific inquiry demands that there be no restriction on the diffusion of basic scientific knowledge, whatever policy is adopted regarding engineering processes and details of weapon construction. General Groves has indicated that data in certain wide fields of basic research are soon to be ``declassified" and made generally available. However, when asked what he meant by ``basic knowledge" he is reported to have replied ``that he thinks of basic knowledge as that which either is generally known or can be easily found out. The Army does not intend to keep secret from American students facts which are openly taught in schools abroad." \textit{Bulletin of the Atomic Scientists of Chicago}, December 24, 1945, p. 2.} In his radio address of August 9, 1945, just after the first announcement had been made of the new weapon, President Truman emphasized that. "The atomic bomb is too dangerous to be loose in a lawless world. That is why Great Britain and the United States, who have the secret of its production, do not intend to reveal the secret until means have been found to control the bomb so as to protect ourselves and the rest of the world from the danger of total destruction." This sentiment has been reiterated in subsequent public discussion. Full revelation is clearly not politically feasible.

Insistence that secrecy must be preserved until ``means have been found to control the bomb" leads naturally, in the minds of those who believe that means of international control of perfect efficacy will not be found, to the ``do-nothing" course of action and to the abandonment even of the quest for common international action. There are two grounds upon which a do-nothing policy has been advocated. On the one hand, it is argued that the atomic age will be an age of plenty, that there will be so much for everybody that no one will covet that which his neighbor has and no nation will covet that which its neighbor has. Although Secretary of Commerce Henry Wallace definitely does not belong to the do-nothing school of thought, his assertion that ``the expectation of a new age of abundance for all will do more to prevent war than the fear of being blown to bits"\footnote{\textit{New York Times}, December 5, 1945.} illustrates the attitude which sees escape from disaster and indeed from the necessity of binding international agreements through a mass distribution of the benefits of atomic energy production. If the new sources of energy developed in the last century and a half made the twentieth century more pacific than the eighteenth or nineteenth, we might gain more comfort from this line of reasoning than we actually do.

The ``tough-minded" argument for a do-nothing policy is somewhat different. It is argued that whatever progress other nations may make in nuclear research, the United States can with its magnificent laboratories and brilliant scientists keep its present lead. If it were true that a better atomic bomb would give security against one not quite so powerful, the United States would indeed be in an advantageous position. Its present lead will, however, seem less important when it first becomes known that some other nation has learned how to produce even the most primitive bombs. As Dr. J. R. Oppenheimer, director of the group which actually designed the first bomb, has declared, ``from the armament race that would almost certainly follow, the United States might or might not emerge the winner, nor would it greatly matter. It is not necessary for a nation to be able to produce more or bigger or better bombs, but only for it to decide to proceed independently with its own atom bomb program, after which with very few bombs it could put any other nation, our own included, out of action."\footnote{Testimony before Senate committee, October 17, 1945; quoted in the \textit{New York Times}, October 18, 1945.} When dealing with the absolute weapon, arguments based on relative advantage lose their point.\footnote{See Chapter I, \textit{supra}.}

``Tell-all" and ``do-nothing" have much in common. Both call for a great act of faith on the part of the American people. In the first case, they are asked to believe that a spontaneous sharing of our present atomic knowledge will work a revolution in the minds and hearts of men and so banish the spectre of atomic war. In the second case, they are asked to believe that the United States is the only country to which the Lord will see fit to entrust the bomb, at least until atomic energy has become so plentiful that there will be nothing left for men to fight about.

The two policies have another feature in common. They are unilateral policies. Under neither plan would the United States have to bargain with other sovereign states. Only a solution which accords to each major power a position in world affairs consonant with its position under the pre-atomic age distribution of power will be considered desirable by those great states who together represent the minimum essential nucleus for agreement.

Nothing can guarantee the indefinite prolongation of such a pattern. It is, for example, possible that in a generation fifteen or twenty nations will have the scientific and engineering knowledge and the industrial capacity to make enough atomic bombs to destroy the major cities of even the greatest state. In such a situation, the Big Three will have become a Big Twenty, and states will be equal in a sense hitherto unknown in our Western state system. That, however, is for the future. If and when it happens, it will be time enough to negotiate an international agreement appropriate to that pattern of power.

In the meantime, agreement must be sought on the basis of the present recognized pattern, the bipolar pattern of the super-powers. In this pattern the Soviet Union and the United States find themselves the nuclei of attraction around which other states tend to group.\footnote{For an exposition of this pattern, see William T. R. Fox, \textit{The Super-Powers}, New York, Harcourt, Brace and Company, 1944.} One may think of the present as the age of the Big Two or the Big Three or even the Big Five. It is not yet the age of the Big One, and no international agreement to control the use of the bomb will make it so.

A proposal which would leave the United States in permanent possession of a  stockpile of atomic bombs while denying to all other powers the right to have them or permission to manufacture them would therefore be ruled out. Governments other than that of the United States do not need to sign such an agreement in order to bring about a situation of American monopoly. They would have nothing to gain by formally acquiescing in such an unequal arrangement. They might feel that they had a great deal to lose since they would never be sure that the successors to the present American leadership might not be tempted at some future date in some as yet unforeseen conflict to resolve that conflict by use of bombs which the United States would then alone possess. Many governments would, therefore, feel more secure if the possible existence at a future date of a stockpile not under American control were not forbidden. Its existence would furnish from their point of view a needed deterrent to any American government tempted to use the bomb for its own national purposes.

The requirement that an acceptable plan not disturb too drastically the existing balance of interests leads to the conclusion that certain other states are not prepared to negotiate with the United States voluntary agreements which will significantly prolong the period of American monopoly. American policy must be planned for the not too distant day when at least some other countries will bargain on an equal footing with the United States.

There is another corollary to the principle that an international control agreement not disturb the existing balance which can be stated more positively. The agreement must offer effective guarantees that bad faith in carrying out such an agreement will achieve no radical disturbance of the present power pattern. If, for example, nations agreed to forego the right of possessing atomic bombs at all, then a single nation which violated the agreement could enforce its will against those which had acted in good faith. Undertakings to abolish or limit drastically the possession of bombs or of atomic energy installations would have to be accompanied by provisions for close inspection.

\label{V-narco1}

Is an inspection scheme really feasible? It would have to be one in which all states had full confidence. It would have to work with equal effectiveness in all countries. Previous experience with international attempts to regulate the narcotics traffic demonstrates the feasibility of detecting many violations of such an international agreement. However, that particular inspection scheme has never been one hundred per cent effective. It has hardly been effective at all against violations committed with the tacit approval of national authorities. It has certainly not been effective to the degree necessary to justify a nation in placing sole reliance upon a similar inspection system for the control of atomic energy production.\footnote{See L. E. C. Eisenlohr, \textit{International Narcotics Control}, London, Allen and Unwin, 1934. In the applicability of the experience in controlling the traffic in narcotic drugs to the problems of inspection and regulation of the arms traffic in general, see ``Analogies between the Problem of the Traffic in Narcotic Drugs and That of the Trade in and Manufacture of Arms", League of Nations, Disarmament Section, Conference for the Reduction and Document IX, Disarmament.1935. IX. 4). This analysis prepared in the League secretariat for the use of the Conference points to the great differences in the two problems, since in the case of narcotics it is private illicit traffic which the agreement seeks to suppress and in the case of arms production it is action taken ``with the active or passive complicity of the Government" which is most likely to constitute a violation of the agreement. See also ``Chemical, Incendiary and Bacterial Weapons: Reply to the Questionnaire Submitted by the Bureau to the Special Committee", \textit{ibid}., 448-72. Some of the conclusions there reached regarding the impracticability of prohibiting the manufacture, import, export or possession of implements or substances capable of both pacific and military utilization apply with even greater force to prohibitions in atomic energy production. Other conclusions also suggest the extent to which discussion of atomic energy control is traversing anew ground already covered in considering previously known ``instruments of mass destruction", e.g.: ``The more highly the chemical industry is developed, the less would production in war time be delayed by a prohibition of the manufacture of the compounds exclusively suitable for chemical warfare (p. 454)." ``The prohibition of preparations for chemical warfare must not hinder chemical and pharmacological research lest such prohibition should prevent the growth of human knowledge and the prospects of overcoming the forces of nature and of combating the scourge of disease (p. 456)." ``We must therefore have the courage to acknowledge that, if leaving on one side the question of its moral value, we only consider the purely technical value of the prohibition to prepare chemical warfare, we must conclude that this prohibition is not of much practical effect (p. 459)."}

\label{V-narco2}

Amidst the welter of assertion and counter-assertion regarding the feasibility of this or that of any system of inspection and control, only one fact stands out clearly. The social scientist working on the problem of control does not have the scientific or engineering data necessary for him to make an intelligent forecast about the feasibility of inspection and control. Nor do many of the physical scientists have the data necessary for such a forecast. All that any physicist or engineer has been permitted to know about atomic energy development is that segment of knowledge which was indispensable for the performance of his own job. As a result, according to the \emph{Bulletin of the Atomic Scientists of Chicago}, ``Because of the secrecy of and compartmentation limitations in the Manhattan Project, it has been impossible for experts in each branch to consider any problem which involved a detailed knowledge of the information available in any other branch. This not only slows down the development of atomic energy, but also prevents an integrated study of the technical feasibility of inspection."\footnote{Vol. I, No. 3, January 10, 1946, p. 2.}

Once it is possible for the scientists and engineers to state more fully the facts upon which their conclusions have been based, the social and political implications and the problems of public policy can be sketched out in greater detail. Meanwhile, the social analyst has at his disposal only a series of vigorous assertions of the necessity of inspection and control made by certain physical scientists. These scientists have displayed a high and admirable sense of civic responsibility, but they are not under present security regulations in a position even to indicate how complete or incomplete their own knowledge of production processes is. Presumably what they regard as necessary they believe to be feasible.

As a matter of fact, no one questions the capacity of a national government to protect itself against the illegal production of bombs within its territory. It follows, therefore, that there are unlikely to be insuperable scientific or technical obstacles to effective inspection and control. The obstacles, if they exist, are political. All that a social scientist can now say is that if adequate inspection is possible through careful inspection of a few strategic control points - like the sites of known uranium deposits, for example - the prospects are better than if adequate inspection requires the policing of the internal affairs of each country so complete that that country's basic social institutions are threatened. It would be premature for policy-makers to make long-term decisions of fundamental importance until the analysis of the feasibility of inspection is more complete than it now appears to be.

In the meantime, the United States must have some policy. This policy must be able to win for the nations of the world time to make a more profound study of the problem of controlling atomic energy on a long-term basis. So long as the policy is clearly understood to be a short-run policy, the necessity for evolving a long-term solution will not be forgotten. Neither will the necessity of keeping the short-term policy in harmony with ultimate goals.

Judged by these standards, how adequate is the beginning made by the United States in the international control of atomic energy? Two three-power conferences have been held. The first, the so-called Potomac Conference resulted on November 15, 1945, in the ``Agreed Declaration" by President Truman and Prime Ministers Attlee and King.\footnote{\textit{Department of State Bulletin}, November 18, 1945, p. 781.} The second, held the following month at Moscow, was at the Foreign Minister level, and resulted on December 27, 1945, in a joint communiqu\'e by Messrs. Bevin, Byrnes, and Molotov.\footnote{\textit{Ibid}., December 30, 1945, p. 1027.} Four-power agreement was thus secured regarding preliminary steps. The United States, Great Britain, Canada, and the Soviet Union agreed to urge the creation of a special United Nations commission to study and report on atomic energy regulation, to facilitate mutual voluntary disclosure of scientific data by the exchange of scientists, scientific publications, and scientific materials and to work step by step for the eventual elimination of the bomb and other instruments of mass destruction from the arsenals of nations.

The governments of France and China joined in the move to have a special commission on atomic energy created by the General Assembly of the United Nations at its first meeting. There was thus six-power agreement on the initial step. As was to have been expected, the six-power proposal was unanimously approved by the Assembly on January 24, 1946.\footnote{The Philippine delegate, alone among the smaller powers' representatives, voiced a widely held sentiment against the slight role allotted to the Assembly either in specifying the membership of the new commission or in supervision of its activities.} That the new commission contains only the representatives of states with seats in the Security Council plus a representative of Canada means that security aspects of atomic energy control are to be no more and no less ``democratically" dealt with than other security problems.

This very moderate program will certainly win time.\footnote{See Chapter III, \textit{supra}, for a fuller discussion of this program.} At the very least, it will win some months during which the United Nations' new commission will be studying the control problem and preparing to report. It can do more. An orderly program of investigation will give the national governments an opportunity for a complete exchange of views and lay the groundwork for broader agreement at a later date. If meanwhile a program for voluntary reciprocal scientific disclosure is vigorously pushed, an atmosphere will have been created which will be more conducive to agreement upon provisions for inspection and ultimately control. So far as they go, these first steps seem unexceptionable.

Can they be criticized as so timid as to be wholly inadequate? Could a forthright American leadership have secured agreement to a bolder program? Contrary to common belief, the American bargaining position in pressing for a fundamental solution to the control problem is not overwhelmingly strong. The present United States monopoly in the manufacture of atomic bombs may even be a weakness for the purpose of these negotiations. The other nations of the world already have that protection against the bomb which comes from its being in the sole possession of a war-weary and non-aggressive country. While it would be clearly in the American interest to get an effective limitation scheme adopted before the Soviet Union er any other country was producing bombs, there seems to be no equivalent advantage on the other side unless the limitation proposal is accompanied by an American willingness to scale down or share or turn over to the United Nations Organization its own stockpile and possibly even to destroy its installations for the manufacture of bombs.\footnote{As has been shown in a preceding chapter, a proposal to turn the control of a stockpile of bombs over to some organ of the United Nations Organization is in fact a variant of proposals either to retain or to share the existing stockpile. See Chapter IV, \textit{supra}.} At some future date American willingness to sacrifice its own stockpile may be greater than it is at present. Or successful production of the bomb by some other country may increase that other country's willingness to see all producers of the bomb including the United States and itself brought under control.

In the light of the present apparent inability of American officials to secure agreement of a more far-reaching character, critics should be slow to condemn the rather modest start made toward the solution of the control problem during the first six months after the dropping of the bombs on Hiroshima. Only those Americans who are able to demonstrate the desirability of immediate destruction of the American stockpile and of American installations for manufacturing further bombs, or some equally radical American sacrifice, are in a position to criticize the government of the United States for not at this time pressing for an effective control system.

The United Nations Atomic Energy Commission has now been created. Whether or not a bolder attack on the control problem would have been possible, it is clear that in these first months no irreparable missteps were taken. There may have been an unjustified American delay in initiating negotiations, and the Anglo-Canadian-American Potomac Conference may have given an appearance of anti-Soviet exclusiveness; but Soviet collaboration in the first steps at least was secured by the subsequent Moscow Conference.

With a special United Nations commission considering the problem, the pressure for immediate action by American government officials may be relaxed for a period of several months. This interval of relaxation must not be wasted. At the end of the period, the United States must have canvassed thoroughly alternative control policies on the basis of a careful evaluation of American interests and an accurate estimate of the position of other governments.

There is another way in which the respite won by the creation of the Atomic Energy Commission can be and must be used. It must be used to create an enlightened public opinion. American officials must be protected against sniping on the home front by those who believe or say they believe that their government is giving away precious scientific secrets, knowledge of which may shortly be turned against our own country. The initial reaction of Senator Austin and Senator Vandenberg to the publication of agreements reached at the Moscow Conference of the three foreign ministers in December, 1945, shows that even the modest step there taken toward international agreement regarding the bomb can be challenged as foolhardy. The records of Senators Austin and Vandenberg by no means suggest that they are narrow nationalists. When criticism comes from responsible senators whose past record shows a willingness to support international collaboration, when and if they are convinced of its desirability and of the public demand for it, the necessity for building up an informed public opinion on questions of atomic energy control policy becomes apparent.

Our discussion of world government and of the policy of full revelation leads to the conclusion that one wing of public opinion as to be educated as to the very narrow limits within which international action to control the bomb now seems possible. The Vandenberg-Austin wing, on the other hand, needs even more to be made to understand the very moderate character of the steps now being taken. It may be unfair to denounce them as recklessly bold. In a country in which each step in foreign relations has to be considered in the light of both domestic and foreign repercussions, it is not enough for high policy-makers to know what is right. They need support from an electorate which also knows what is right. There can be no substitute for an understanding public opinion if American officials are to have the freedom and the guidance which they need. If they do not have this home front support, they will surely fail. The time is short in which to develop it.

\noindent\hfil\rule{0.4\textwidth}{.4pt}\hfil

\vspace{4pt}

Even though it seems probable that the scope of the agreements immediately forthcoming on matters connected with atomic energy will be very modest, it is not too soon to begin speculating on the nature of a successful long-term atomic energy control policy.

There is general agreement regarding the long-term control objectives only on two points. Control is to be established step by step. Eventually, there is to be an inspection system. Beyond these two points, a long term control program, to be successful, must be based on the following considerations.

\begin{enumerate}[1.]

\item The control problem is inseparable from the general problem of relations among the great powers. It is most intimately related of course to Soviet-American relations. No serious consideration therefore should be given to types of solutions which stand no chance of being accepted by either the United States or the Soviet Union.\footnote{See Chapter IV, \textit{supra}.} World government right now has already been ruled out on this count.

\item The powers and especially the great powers must be prepared to accept a substantial narrowing in their range of free choice of policy. Current talk about sacrificing sovereignty recognizes this necessity. The difficulty with the phrase ``sacrificing sovereignty" is that it seems to imply that the sovereignty is to be handed over to some supra-national authority. To endow a supra-national authority with great power might make the national authorities more apprehensive of it than each other. It is at least conceivable that the powers can contrive some scheme for narrowing their own freedom of action so as to reassure each other without at the same time broadening the scope of free action of the supra-national authority. The powers might, for example, agree that the bomb is not to be used at all except in the most narrowly defined circumstances. This would be far different from creating a world authority which itself had bombs at its disposal.

\item Any legal undertaking limiting the right of states to produce, possess or use atomic armaments must be self-enforcing. Only if as the result of the legal undertaking, a factual situation is created in which the powers are not tempted to break the agreement would this condition be met. An agreement outlawing the production or use of atomic bombs would have to be accompanied by provisions for inspection and penalties for violation to meet this test. The failure of belligerents in the Second World War to use poison gas tempts one to assert that simple international agreements outlawing the use of a weapon might be effective. The experience with poison gas, however, is not wholly reassuring.\footnote{See Chapter II, \textit{supra}.} Gas has not proved a decisive weapon. Had Hitler or Tojo been capable of averting defeat by using gas, few doubt that they would have used it.

\item The limitation agreement must be in fact as well as in form binding on the United States as much as on other interested parties. There is no way in which the United States by pressing for international agreement to control the atomic bomb can hope to preserve its own advantage in this field. Few states and certainly none of the great states will be prepared simply to accept American assurances that our present stockpile will never be used except against an aggressor. This will be especially true so long as the United States is the power which determines whether or not a given act constitutes aggression.

\end{enumerate}

How can this description of the minimum conditions of a successful control scheme be translated into a prescription for statesmen charged with the grave responsibility of avoiding atomic war? If the problem of atomic energy control is indeed inseparable from the problem of Soviet-American relations, then the principle upon which these good relations are to be preserved must be strengthened and not scrapped. Specifically, a control proposal which is to have any chance of general acceptance must not require the elimination of the voting procedure developed at Yalta.

\label{V-UNSC}

A careful comparison of the Agreed Declaration emanating from the Potomac Conference and the joint communiqu\'e of the three Foreign Ministers after the Moscow Conference suggests that the Western powers made an abortive attempt to maximize the role of the General Assembly in atomic energy control. John Foster Dulles declared on November 16, 1945, the day after the publication of the Agreed Declaration: ``We have set up a General Assembly to be the `town meeting of the world.' Let us invite, and heed, its judgment of what we should do. I have no idea what the Assembly would recommend, and it is not of primary importance. What is most important is that we accept a procedure which shows that we really mean it when we say that we are merely a trustee of atomic power (\textit{New York Times}, November 16, 1945)." The Moscow communiqu\'e on the other hand made it abundantly clear that the functions of the Security Council are in no way being impaired by the creation of a special atomic energy commission. Thus, the integrity of the principle of voluntary collaboration among the greatest states, which is implicit throughout the whole field of Security Council action, has been preserved.

As the special United Nations commission on atomic energy control begins to operate, it will not find it useful to recommend principles for control which do not take full account of the special position of the permanent members of the Security Council within the Organization. Indeed, there is slight probability that it will do so; for the Commission contains only the representatives of those powers with seats on the Council, plus a Canadian representative whenever that power does not possess a Council seat.

With the veto principle intact, it became possible for the Security Council, or its alter ego, the special commission on atomic energy control, to exercise the broad powers of regulation and supervision which the Charter already grants them. The Council now has, and might delegate to the commission, primary responsibility for prescribing the conditions under which the production, possession, or use of atomic energy is permitted.\footnote{Article 24, United Nations Charter.}

There is one use of atomic bombs which is at this moment legal and which the Council will not want to forbid. This is its use as part of the enforcement arrangements of the Security Council and its Military Staff Committee. In the unhappy event that Germany or Japan should again in our time attain military power sufficient to make themselves major threats to the peace of the world, the bombs might be used against them. Given the present voting arrangements in the Security Council, there are no other potential major aggressors against which the Council might apply this terrible sanction.

There is another use of the bomb which its possessors ought not only to be permitted but to be obligated to make of it. This would be to retaliate immediately against any power using the bomb which was not acting with the express authorization of the Security Council. Only retaliatory action which was not expected to be immediate and certain would not be an effective deterrent against aggression committed with atomic weapons. There would not be time for the Security Council to act after receiving word of an illegal use of the bomb, nor would its present organization and voting procedure permit it to act in the unhappy event that one of the great states were to use the bomb. It would thus be necessary, in order to insure retaliatory action, to make provision separate from the regular procedures for enforcement action and in advance of the aggression. Advance provision for automatic retaliation by all other nations possessing the bomb against any one which had illegally used it would be a powerful deterrent to a would-be atomic aggressor.

Separate advance provision for automatic obligatory retaliation by-passes the great power veto. Would such a provision be acceptable to the great powers? Here reference should be made to Article 51 of the United Nations Charter. This article specifically reaffirms ``the inherent right of individual or collective self-defense if an armed attack occurs."\footnote{See Chapter III, \textit{supra}.} Legitimate collective self-defense
against atomic attack surely includes the right to negotiate bilateral or multilateral treaties in which the possessors of the bomb undertake the obligation of automatic retaliation. There might even be a single general pact specifying this obligation.\footnote{This is the suggestion of E. L. Woodward, Montagu Burton Professor of International Relations at Oxford, in \textit{Some Political Consequences of the Atomic Bomb}, London, Oxford University Press, 1946, p. 25, except that Professor Woodward would provide for obligatory retaliation unless the bomb had first been used with the unanimous consent of the Council. It would probably be preferable and certainly more practical if the Council's authorization were given in accordance with its usual voting procedure, as laid down in the so-called Yalta voting formula, which does not require the unanimous consent of the non-permanent members of the Council.}

If a general obligation of instant and automatic retaliation were the sole safeguard evolved by the international community against the new weapon, unlimited production of atomic bombs would be permitted. It might be argued that no arms limitation is desirable since disarmament would make the disarmed nations feel insecure and would also weaken the effectiveness of the retaliatory sanction. Furthermore, a system which required no limitation would require no inspection, and the insecurities which would arise from doubts about the feasibility of inspection would be avoided.

There are nevertheless cogent reasons why states should not be content simply with the primitive and drastic safeguard of retaliation. A world in which two or more states were sitting on powder kegs powerful enough to destroy every major city on earth would be a world of half-peace at best. For perhaps a generation, no state would press any dispute to the point of war because of the fear of atomic counterattack.\footnote{See Chapter II, \textit{supra}.} In so far as this fear is a restraining influence on state behavior, it would exist even if there were no general obligation of automatic retaliation. Many states, however, acting with the knowledge of the reluctance of the other party to be drawn into war, might pursue policies which their opponents would regard as only slightly less intolerable than atomic war itself. In such a situation, there might well be a long-run gradual rise in the tension level of international politics until some state came to regard war as less intolerable than the half-peace of unbearable tension.

Sole reliance should be placed upon the retaliatory sanction only during an interim period. Meanwhile, efforts should be made to bring down the level of permitted atomic armament to a point at which no single state's action would reduce the earth to a smoldering ruin.

If no bombs were to be permitted to exist anywhere, then that nation which successfully produced bombs in violation of its agreement not to do so would have the more peace-loving remainder of the world at its mercy. Furthermore, the sanction of obligatory retaliation would have been destroyed. Is there some level of permitted atomic armament low enough to prevent the first contingency and high enough to prevent the second?

Theoretically at least, there may be such a level. Suppose that each of the great states and also powers capable of independent production of the bomb were permitted to keep a small supply of bombs. The total number of bombs permitted to exist should perhaps be not much greater than that calculated to be sufficient to bring about the capitulation of the greatest state. The number of bombs permitted to any one state would therefore be very much less than that sufficient to bring about such a capitulation. The number of bombs beyond the control of any given state would on the other hand be such that that state would pay dearly for an attempted aggression in terms of the devastation of its territories and might even be almost totally destroyed.\footnote{One possible objection to a proposal of this character is that it might render even more difficult the inspection problem. The enforcement of a particular distribution of atomic weapons might require a more detailed inspection than the enforcement of an agreement which forbade totally the possession or production of atomic bombs.} In this situation, the effectiveness of the retaliatory sanction would be preserved.

Such a situation of drastic atomic arms limitation would require detailed and close inspection of national armaments under the supervision of the United Nations Organization. Inspection would not, however, be the only safeguard. Discovery of a violation of the limitation agreement would not mean that all was already lost. Such a discovery would be the signal for a general atomic rearmament and for political action to enforce compliance by the offending state.

Long experience with detailed and close inspection for enforcement of atomic arms limitation agreements might ultimately permit such great confidence to be placed in the efficacy of inspection that the complete abolition of atomic armaments would become possible. This third stage of atomic arms regulation is clearly not for our own decade. Whether and how soon it will become politically feasible is not for this writer to say.

\noindent\hfil\rule{0.4\textwidth}{.4pt}\hfil

\vspace{4pt}

It is not too soon to take the first step in the control pattern sketched in the preceding pages. The members of the United Nations should agree now to undertake instantaneous retaliation. The second step, the agreement for a drastic limitation on permitted atomic armaments and for a detailed and close inspection by an international agency of those armaments, may be taken when other nations have discovered independently how to produce the bomb. Is there any earlier date at which this step towards a fundamental solution of the control problem will become possible? Probably not, unless the United States is  willing to make a gesture which many people would regard as even more quixotic than ``telling-all." This would be to give to the Soviet Union and to other members of the Big Five a limited number of bombs and, perhaps also, the information necessary to make some more. The requisite number would not have to be so great as to permit any other government to destroy the major cities of the United States. It would have to be great enough so that the world would be sure the United States would not be tempted to settle current international differences by using or threatening to use the bomb. Needless to say, this is not a proposal which, in the present state of American opinion, the United States is prepared to make.

Our conclusion must therefore be that a spectacular and permanent solution to the vexing and grave problem of international control of atomic weapons is not now within our grasp. What we can do now is to take the first in a series of steps which promise to prevent atomic warfare until that date when other nations have learned how to produce the weapon and a more fundamental consideration of the problem is in order.
