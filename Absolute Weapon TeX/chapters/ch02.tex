
\chapter{Implications for Military Policy}

\vspace{-2pt}

\noindent{\normalsize \textbf{Bernard Brodie}}

\vspace{39pt}

Under conditions existing before the atomic bomb, it was possible to contemplate methods of air defense keeping pace with and perhaps even outdistancing the means of offense. Long-range rockets baffled the defense, but they were extremely expensive per unit for inaccurate, single-blow weapons. Against bombing aircraft, on the other hand, fighter planes and antiaircraft guns could be extremely effective. Progress in speed and altitude performance of all types of aircraft, which on the whole tends to favor the attacker, was more or less offset by technological progress in other fields where the net result tends to favor the defender (e.g., radar search and tracking, proximity fused projectiles, etc.).

At any rate, a future war between great powers could be visualized as one in which the decisive effects of strategic bombing would be contingent upon the \emph{cumulative effect of prolonged bombardment efforts}, which would in turn be governed by aerial battles and even whole campaigns for mastery of the air. Meanwhile - if the recent war can serve as a pattern - the older forms of warfare on land and sea would exercise a telling effect not only on the ultimate decision but on the effectiveness of the strategic bombing itself. Conversely, the strategic bombing would, as was certainly true against Germany, influence or determine the decision mainly through its effects on the ground campaigns.

The atomic bomb seems, however, to erase the pattern described above, first of all because its enormous destructive potency is bound vastly to reduce the time necessary to achieve the results which accrue from strategic bombing - and there can no longer be any dispute about the decisiveness of strategic bombing. In fact, the essential change introduced by the atomic bomb is not primarily that - it will make war more violent - a city can be as effectively destroyed with TNT and incendiaries - but that it will concentrate the violence in terms of time. A world accustomed to thinking it horrible that wars should last four or five years is now appalled at the prospect that future wars may last only a few days.

One of the results of such a change would be that a far greater proportion of human lives would be lost even in relation to the greater physical damage done. The problem of alerting the population of a great city and permitting resort to air raid shelters is one thing when the destruction of that city requires the concentrated efforts of a great enemy air/force; it is quite another when the job can be done by a few aircraft flying at extreme altitudes. Moreover, the feasibility of building adequate air raid shelters against the atomic bomb is more than dubious when one considers that the New Mexico bomb, which was detonated over 100 feet above the ground, caused powerful earth tremors of an unprecedented type lasting over twenty seconds.\footnote{\textit{Time}, January 28, 1946, p. 75.} The problem merely of ventilating deep shelters, which would require the shutting out of dangerously radioactive gases, is considered by some scientists to be practically insuperable. It would appear that the only way of safeguarding the lives of city dwellers is to evacuate them from their cities entirely in periods of crisis. But such a project too entails some nearly insuperable problems.

What do the facts presented in the preceding pages add up to for our military policy? Is it worth-while even to consider military policy as having any consequence at all in an age of atomic bombs? A good many intelligent people think not. The passionate and \emph{exclusive} preoccupation of some scientists and laymen with proposals for ``world government" and the like - in which the arguments are posed on an ``or else" basis that permits no question of feasibility - argues a profound conviction that the safeguards to security formerly provided by military might are no longer of any use.

Indeed the postulates set forth and argued in the preceding chapter would seem to admit of no other conclusion. If our cities can be wiped out in a day, if there is no good reason to expect the development of specific defenses against the bomb, if all the great powers are already within striking range of each other, if even substantial superiority in numbers of aircraft and bombs offers no real security, of what possible avail can large armies and navies be? Unless we can strike first and eliminate a threat before it is realized in action - something which our national Constitution effectively forbids - we are bound to perish under attack without even an opportunity to mobilize resistance. Such at least seems to be the prevailing conception among those who, if they give any thought at all to the military implications of the bomb, content themselves with stressing its character as a weapon of aggression.

The conviction that the bomb represents the apotheosis of aggressive instruments is especially marked among the scientists who developed it. They know the bomb and its power. They also know their own limitations as producers of miracles. They are therefore much less sanguine than many laymen or military officers of their capacity to provide the instrument which will rob the bomb of its terrors. One of the most outstanding among them, Professor J. Robert Oppenheimer, has expressed himself quite forcibly on the subject:

\begin{quote}
``The pattern of the use of atomic weapons was set at Hiroshima. They are weapons of aggression, of surprise, and of terror. If they are ever used again it may well be by the thousands, or perhaps by the tens of thousands; their method of delivery may well be different, and may reflect new possibilities of interception, and the strategy of their use may well be different from what it was against an essentially defeated enemy. But it is a weapon for aggressors, and the elements of surprise and of terror are as intrinsic to it as are the fissionable nuclei."\footnote{``Atomic Weapons and the Crisis in Science", \textit{Saturday Review of Literature}, November 24, 1945, p. 10.}
\end{quote}

The truth of Professor Oppenheimer's statement depends on one vital but unexpressed assumption: that the nation which proposes to launch the attack will not need to fear retaliation. If it must fear retaliation, the fact that it destroys its opponent's cities some hours or even days before its own are destroyed may avail it little. It may indeed commence the evacuation of its own cities at the same moment it is hitting the enemy's cities (to do so earlier would provoke a like move on the opponent's part) and thus present to retaliation cities which are empty. But the success even of such a move would depend on the time interval between hitting and being hit. It certainly would not save the enormous physical plant which is contained in the cities and which over any length of time is indispensable to the life of the national community. Thus the element of surprise may be less important than is generally assumed.\footnote{A superior army which advances by surprise on a critical objective obliges the opponent to grapple with it at a place and time of its own choosing. A bombing attack has no such confining effect on the initiative of the enemy so long as his means of retaliation remain relatively intact. Bombs of any kind are generally not used against each other, and the advantages which follow from surprise in their use are usually of a tactical rather than strategic nature.}

If the aggressor state must fear retaliation, it will know that even if it is the victor it will suffer a degree of physical destruction incomparably greater than that suffered by any defeated nation of history, incomparably greater, that is, than that suffered by Germany in the recent war. Under those circumstances no victory, even if guaranteed in advance - which it never is - would be worth the price. The threat of retaliation does not have to be 100 per cent certain; it is sufficient if there is a good chance of it. But that chance has to be evident. The prediction is more important than the fact.

The argument that the victim of an attack might not know where the bombs are coming from is almost too preposterous to be worth answering, but it has been made so often by otherwise responsible persons that it cannot be wholly ignored. That the geographical location of the launching sites of long-range rockets may remain for a time unknown is conceivable, though unlikely, but that the identity of the attacker should remain unknown is not in modern times conceivable. The fear that one's country might suddenly be attacked in the midst of apparently profound peace has often been voiced, but, at least in the last century and a half, it has never been realized. As advancing technology makes war more horrible, it also makes the decision to resort to it more dependent on an elaborate psychological preparation. In international politics today few things are more certain than that an attack must have an antecedent dispute of obviously grave character. Even those statesmen who remain blind to the most blatant warnings will understand the significance of those warnings once the attack occurs.\footnote{It is possible, of course, that a state which has resolved to fight as a result of a political crisis may for tactical reasons await the partial dissipation of the crisis tension, perhaps furthering the process by a deceptive acquiescence or surrender; but even if this were likely - which it is not - the identity of the attacker would still be known.} Especially today, when there are only two or three powers of the first rank, the identity of the major rival is unambiguous. In fact, as Professor Jacob Viner has pointed out, it is the lack of ambiguity concerning the major rival which makes the bi-polar power system so dangerous.

There is happily little disposition to believe that the atomic bomb by its mere existence and by the horror implicit in it ``makes war impossible." In the sense that war is something not to be endured if any reasonable alternative remains, it has long been ``impossible." But for that very reason we cannot hope that the bomb makes war impossible in the narrower sense of the word. Even without it the conditions of modern war should have been a sufficient deterrent but proved not to be such. If the atomic bomb can be used without fear of substantial retaliation in kind, it will clearly encourage aggression. So much the more reason, therefore, to take all possible steps to assure that multilateral possession of the bomb, should that prove inevitable, be attended by arrangements to make as nearly certain as possible that the aggressor who uses the bomb will have it used against him.

If such arrangements are made, the bomb cannot but prove in the net a powerful inhibition to aggression. It would make relatively little difference if one power had more bombs and were better prepared to resist them than its opponent. It would in any case undergo incalculable destruction of life and property. It is clear that there existed in the thirties a deeper and probably more generalized revulsion against war than in any other era of history. Under those circumstances the breeding of a new war required a situation combining dictators of singular irresponsibility with a notion among them and their general staffs that aggression would be both successful and cheap. The possibility of irresponsible or desperate men again becoming rulers of powerful states cannot under the prevailing system of international politics be ruled out in the future. But it does seem possible to erase the idea - if not among madman rulers then at least among their military supporters - that aggression will be cheap.

Thus, the first and most vital step in any American security program for the age of atomic bombs is to take measures to guarantee to ourselves in case of attack the possibility of retaliation in kind. The writer in making that statement is not for the moment concerned about who will \emph{win} the next war in which atomic bombs are used. Thus far the chief purpose of our military establishment has been to win wars. From now on its chief purpose must be to avert them. It can have almost no other useful purpose.

Neither is the writer especially concerned with whether the guarantee of retaliation is based on national or international power. However, one cannot be unmindful of one obvious fact: for the period immediately ahead, we must evolve our plans with the knowledge that there is a vast difference between what a nation can do domestically of its own volition and on its own initiative and what it can do with respect to programs which depend on achieving agreement with other nations. Naturally, our domestic policies concerning the atomic bomb and the national defense generally should not be such as to prejudice real opportunities for achieving world security agreements of a worth-while sort. That is an important proviso and may become a markedly restraining one.

Some means of international protection for those states which cannot protect themselves will remain as necessary in the future as it has been in the past.\footnote{The argument has been made that once the middle or small powers have atomic bombs they will have restored to them the ability to resist effectively the aggressions of their great power neighbors - an ability which otherwise has well-nigh disappeared. This is of course an interesting speculation on which no final answer is forthcoming. It is true that a small power, while admitting that it could not win a war against a great neighbor, could nevertheless threaten to use the bomb as a penalizing instrument if it were invaded. But it is also true that the great-power aggressor could make counter threats concerning its conduct while occupying the country which had used atomic bombs against it. It seems to this writer highly unlikely that a small power would dare threaten use of the bomb against a great neighbor which was sure to overrun it quickly once hostilities began, especially since such a threat could serve as a justification, if one were needed, for the use of the bomb by the great-power aggressor.} Upon the security of such states our own security must ultimately depend. But only a great state which has taken the necessary steps to reduce its own direct vulnerability to atomic bomb attack is in a position to offer the necessary support. Reducing vulnerability is at least one way of reducing temptation to potential aggressors. And if the technological realities make reduction of vulnerability largely synonymous with preservation of striking power, that is a fact which must be faced. Under those circumstances any domestic measures which effectively guaranteed such preservation of striking power under attack would contribute to a more solid basis for the operation of an international security system.

\noindent\hfil\rule{0.4\textwidth}{.4pt}\hfil

\vspace{4pt}

It is necessary therefore to explore all conceivable situations where the aggressor's fear of retaliation will be at a minimum and to seek to eliminate them. The first and most obvious such situation is that in which the aggressor has a monopoly of the bombs. The United States has a monopoly today, but trusts to its reputation for benignity and - what is more impressive - its conspicuous weariness of war to still the perturbations of other powers. In any case, that special situation is bound to be short-lived. The possibility of a recurrence of monopoly in the future would seem to be restricted to a situation in which controls for the rigorous suppression of atomic bomb production had been imposed by international agreement but had been evaded or violated by one power without the knowledge of the others. Evasion or violation, to be sure, need not be due to aggressive designs. It might stem simply from a fear that other nations were doing likewise and a desire to be on the safe side. Nevertheless, a situation of concealed monopoly would be one of the most disastrous imaginable from the point of view of world peace and security. It is therefore entirely reasonable to insist that any system for the international control or suppression of bomb production should include safeguards promising practically 100 per cent effectiveness.

The use of secret agents to plant bombs in the the major cities of an intended victim was discussed in the previous chapter, where it was concluded that except in port cities easily accessible to foreign ships such a mode of attack could hardly commend itself to an aggressor. Nevertheless, to the degree that such planting of bombs is reasonably possible, it suggests that one side might gain before the opening of hostilities an enormous advantage in the \emph{deployment} of its bombs. Clearly such an ascendancy would contain no absolute guarantee against retaliation unless the advantage in deployment were associated with a marked advantage in psychological preparation for resistance. But it is clear also that the relative position of two states concerning ability to use the atomic bomb depends not alone on the number of bombs in the possession of each but also on a host of other conditions, including respective positions concerning deployment of the bombs and psychological preparation against attack.

One of the most important of those conditions concerns the relative position of the rival powers in technological development, particularly as it affects the vehicle for carrying the bombs. At present the only instrument for bombardment at distances of over 200 miles is the airplane (with or without crew). The controlled rocket capable of thousands of miles of range is still very much in the future. The experience of the recent war was analyzed in the previous chapter as indicating that an inferior air force can usually penetrate the aerial defenses of its opponent so long as it is willing to accept a high loss ratio. \label{II-Superior} Nevertheless, the same experience shows also that one side can be so superior quantitatively and qualitatively in both aerial offense and defense as to be able to range practically undisturbed over the enemy's territories while shutting him out largely, even if not completely, from incursions over its own. While such a disparity is likely to be of less importance in a war of atomic bombs than it has been in the past, its residual importance is by no means insignificant.\footnote{It was stated in the previous chapter, p. \pageref{I-frustrate}, that before we can consider a defense against atomic bombs effective, ``the frustration of the attack for any given target area must be complete." The emphasis in that statement is on a specific and limited target area such as a small or medium size city. For a whole nation containing many cities such absolute standards are obviously inapplicable. The requirements for a ``reasonably effective" defense would still be far higher than would be the case with ordinary TNT bombs, but it would certainly not have to reach 100 per cent frustration of the attack. All of which says little more than that a nation can absorb more atomic bombs than can a single city.} And in so far as the development of rockets nullifies that type of disparity in offensive power, it should be noted that the development of rockets is not likely to proceed at an equal pace among all the larger powers. One or several will far outstrip the others, depending not alone on the degree of scientific and engineering talent available to each country but also on the effort which its government causes to be channelled into such an enterprise. In any case, the possibilities of an enormous lead on the part of one power in effective use of the atomic bomb are inseparable from technological development in vehicles - at least up to a certain common level, beyond which additional development may matter little.

\label{II-Retaliation1}

The consequences of a marked disparity between opponents in the spatial concentration of populations and industry is left to a separate discussion later in this chapter. But one of the aspects of the problem which might be mentioned here, particularly as it pertains to the United States, is that of having concentrated in a single city not only the main agencies of national government but also the whole of the executive branch, including the several successors to the presidency and the topmost military authorities. While an aggressor could hardly count upon destroying at one blow all the persons who might assume leadership in a crisis, he might, unless there were considerably greater geographic decentralization of national leadership than exists at present, do enough damage with one bomb to create complete confusion in the mobilization of resistance.

It goes without saying that the governments and populations of different countries will show different levels of apprehension concerning the effects of the bomb. It might be argued that a totalitarian state would be less unready than would a democracy to see the destruction of its cities rather than yield on a crucial political question. The real political effect of such a disparity, however - if it actually exists, which is doubtful - can easily be exaggerated. \emph{For in no case is the fear of the consequences of atomic bomb attack likely to be low}. More important is the likelihood that totalitarian countries can impose more easily on their populations than can democracies those mass movements of peoples and industries necessary to disperse urban concentrations.

The most dangerous situation of all would arise from a failure not only of the political leaders but especially of the military authorities of a nation like our own to adjust to the atomic bomb in their thinking and planning. The possibility of such a situation developing in the United States is very real and very grave. We are familiar with the example of the French General Staff, which failed to adjust in advance to the kind of warfare obtaining in 1940. There are other examples, less well-known, which lie much closer home. In all the investigations and hearings on the Pearl Harbor disaster, there has at this writing not yet been mention of a fact which is as pertinent as any - that our ships were virtually naked in respect to antiaircraft defense. They were certainly naked in comparison to what was considered necessary a brief two years later, when the close-in antiaircraft effectiveness of our older battleships was estimated by the then Chief of the Bureau of Ordnance to have increased by no less than 100 times! That achievement was in great part the redemption of past errors of omission. The admirals who had spent so many of their waking hours denying that the airplane was a grave menace to the battleship had never taken the elementary steps necessary to validate their opinions, the steps, that is, of covering their ships with as many as they could carry of the best antiaircraft guns available

Whatever may be the specific changes indicated, it is clear that our military authorities will have to bestir themselves to a wholly unprecedented degree in revising military concepts inherited from the past. That will not be easy. They must be prepared to dismiss, as possibly irrelevant, experience gained the hard way in the recent war, during which their performance was on the whole brilliant.

Thus far there has been no public evidence that American military authorities have begun really to think in terms of atomic warfare. The test announced with such fanfare for the summer of 1946, when some ninety-seven naval vessels will be subjected to the blast effect of atomic bombs, merely serves to confirm this impression. Presumably the test is intended to gauge the defensive efficacy of tactical dispersion, since there can be little doubt of the consequences to any one ship of a near hit. While such tests are certainly useful it should be recognized at the outset that they can provide no answer to the basic question of the utility of sea power in the future.

Ships at sea are in any case not among the most attractive of military targets for atomic bomb attack. Their ability to disperse makes then comparatively wasteful targets for bombs of such concentrated power and relative scarcity; their mobility makes them practically impossible to hit with super-rockets of great range; and those of the United States Navy at least have shown themselves able, with the assistance of their own aircraft, to impose an impressively high ratio of casualties upon hostile planes endeavoring to approach them. But the question of how their own security is affected is not the essential point. \emph{For it is still possible for navies to lose all reason for being even if they themselves remain completely immune}.

A nation which had lost most of its larger cities and thus the major part of its industrial plant might have small use for a fleet. One of the basic purposes for which a navy exists is to protect the sea-borne transportation by which the national industry imports its raw materials and exports its finished commodities to the battle lines. Moreover, without the national industrial plant to service it, the fleet would shortly find itself without the means to function. In a word, the strategic issues posed by the atomic bomb transcend all tactical issues, and the 1946 test and the controversy which will inevitably follow it will no doubt serve to becloud that basic point.

\label{II-Retaliation2}

\vspace{12pt}

\noindent\textbf{Outlines of a Defense Program in the Atomic Age}

\vspace{10pt}

What are the criteria by which we can appraise realistic military thinking in the age of atomic bombs? The burden of the answer will depend primarily on whether one accepts as true the several postulates presented and argued in the previous chapter. One might go farther and say that since none of them is obviously untrue, no program of military preparedness which fails to consider the likelihood of their being true can be regarded as comprehensive or even reasonably adequate.

It is of course always possible that the world may see another major war in which the atomic bomb is not used. The awful menace to both parties of a reciprocal use of the bomb may prevent the resort to that weapon by either side, even if it does not prevent the actual outbreak of hostilities. It is, for reasons which will presently be indicated, highly unlikely that such a situation will occur. But even if it did occur, the shadow of the atomic bomb would so govern the strategic and tactical dispositions of either side as to create a
wholly novel form of war. The kind of spatial concentrations of force by which in the past great decisions have been achieved would be considered too risky. The whole economy of war would be affected, for even if the governments were willing to assume responsibility for keeping the urban populations in their homes, the spontaneous exodus of those populations from the cities might reach such proportions as to make it difficult to service the machines of war. The conclusion is inescapable that war will be vastly different because of the atomic bomb whether or not the bomb is actually used.

But let us now consider the degree of probability inherent in each of the three main situations which might follow from a failure to prevent a major war. These three situations may be listed as follows:
\begin{enumerate}[(a)]
\item a war fought without atomic bombs or other forms of radioactive energy;
\item a war in which atomic bombs were introduced only considerably after the outbreak of hostilities;
\item a war in which atomic bombs were used at or near the very outset of hostilities.
\end{enumerate}
\noindent We are assuming that this hypothetical conflict occurs at a time when each of the opposing sides possesses at least the ``know-how" of bomb production, a situation which, as argued in the previous chapter, approximates the realities to be expected not more than five to ten years hence.

Under such conditions the situation described under (a) above could obtain only as a result of a mutual fear of retaliation, perhaps supported by international instruments outlawing the bomb as a weapon of war. It would \emph{not} be likely to result from the operation of an international system for the suppression of bomb production, since such a system would almost certainly not survive the outbreak of a major war. If such a system were in fact effective at the opening of hostilities, the situation resulting would be far more likely to fall under (b) than under (a), unless the war were very short. For the race to get the bomb would not be an even one, and the side which got it first in quantity would be under enormous temptation to use it before the opponent had it. Of course, it is more reasonable to assume that an international situation which had so far deteriorated as to permit the outbreak of a major war would have long since seen the collapse of whatever arrangements for bomb production control had previously been imposed, unless the conflict were indeed precipitated by an exercise of sanctions for the violation of such a control system.

Thus we see that a war in which atomic bombs are not used is more likely to occur if both sides have the bomb in quantity from the beginning than if neither side has it at the outset or if only one side has it.\footnote{One can almost rule out too the possibility that war would break out between two great powers where both knew that only one of them had the bombs in quantity. It is one of the old maxims of power politics that \emph{c'est une crime de faire la guerre sans compter sur la sup\'eriorit\'e}, and certainly a monopoly of atomic bombs would be a sufficiently clear definition of superiority to dissuade the other side from accepting the gauge of war unless directly attacked.} But how likely is it to occur? Since the prime motive in refraining from using it would be fear of retaliation, it is difficult to see why such a fear should be strong enough to prevent the use of the bomb without being strong enough to prevent the outbreak of war in the first place. In other words, the whole situation would argue a kind of marginal behavior which is foreign to human nature.

The fact is that once hostilities broke out, the pressures to use the bomb would swiftly reach unbearable proportions. One side or the other would feel that its relative position respecting ability to use the bomb might deteriorate as the war progressed, and that if it failed to use the bomb while it had the chance it might not have the chance later on. The side which was decidedly weaker in terms of industrial capacity for war would be inclined to use it in order to equalize the situation on a lower common level of capacity - for it is clear that the side with the more elaborate and intricate industrial system would, other things being equal, be more disadvantaged by mutual use of the bomb than its opponent. In so far as those ``other things" were not equal, the disparities involved would also militate for the use of the bomb by one side or the other. And hovering over the situation from beginning to end would be the intolerable fear on each side that the enemy might at any moment resort to this dreaded weapon, a fear which could hardly fail to stimulate an anticipatory reaction.

Some observers in considering the chances of effectively outlawing the atomic bomb have taken a good deal of comfort from the fact that poison gases were not used, or at least not used on any considerable scale, during the recent war. There is little warrant, however, for assuming that the two problems are analogous. Apart from the fact that the recent war presents only a single case and argues little for the experience of another war even with respect to gas, it is clear that poison gas and atomic bombs represent two wholly different orders of magnitude in military utility. The existence of the treaty outlawing gas was important, but at least equally important was the conviction in the minds of the military policy-makers that TNT bombs and tanks of gelatinized gasoline - with which the gas bombs would have had to compete in airplane carrying capacity - were just as effective as gas if not more so. Both sides were prepared not only to retaliate with gas against gas attack but also to neutralize with gas masks and ``decontamination units" the chemicals to which they might be exposed. There is visible today no comparable neutralization agent for atomic bombs.

Neither side in the recent war wished to bear the onus for violation of the obligation not to use gas when such violation promised no particular military advantage. But, unlike gas, the atomic bomb is a weapon which can scarcely fail to be decisive if used at all. That is not to say that any effort to outlaw use of the bomb is arrant nonsense, since such outlawry might prove the indispensable crystalizer of a state of balance which operates against use of the bomb. But without the existence of the state of balance - in terms of reciprocal ability to retaliate in kind if the bomb is used - any treaty purposing to outlaw the bomb in war would have thrust upon it a burden far heavier than such a treaty can normally bear.

If the analysis presented in the preceding paragraphs is correct, we must conclude that of the three situations listed above, that described under (b) is considerably more likely to occur than that presented under (a), and for much the same reasons the situation listed under (c) has a greater degree of probability of occurrence than (b). In other words, if the fear of reciprocal use of the bomb is not sufficient to prevent a war from breaking out in the first place, it is hardly likely to be sufficient to prevent the bomb from being used, and if the bomb is going to be used at all in a conflict it is likely to be used early rather than late.

What do these conclusions mean concerning the defense preparations of a nation like the United States? In answering this question, it is necessary first to anticipate the argument that ``the best defense is a strong offense", an argument which it is now fashionable to link with animadversions on the ``Maginot complex." In so far as this doctrine becomes dogma, it may prejudice the security interests of the country and of the world. Although the doctrine is basically true as a general proposition, especially when applied to hostilities already under way, the political facts of life concerning the United States government under its present Constitution make it most probable that if war comes we will receive the first blow rather than deliver it. Thus, our most urgent military problem is to reorganize ourselves to survive a vastly more destructive ``Pearl Harbor" than occurred in 1941. Otherwise we shall not be able to take the offensive at all.

\pageref{II-Retaliation3}

The atomic bomb will be introduced into the conflict only on a gigantic scale. No belligerent would be stupid enough, in opening itself to reprisals in kind, to use only a few bombs. The initial stages of the attack will certainly involve hundreds of the bombs, more likely thousands of them. Unless the argument of Postulate II and IV in the previous chapter is wholly preposterous, the target state will have little chance of effectively halting or fending off the attack. If its defenses are highly efficient it may down nine planes out of every ten attacking, but it will suffer the destruction of its cities. That destruction may be accomplished in a day, or it may take a week or more. But there will be no opportunity to incorporate the strength residing in the cities, whether in the form of industry or personnel, into the forces of resistance or counter-attack. \emph{The ability to fight back after an atomic bomb attack will depend on the degree to which the armed forces have made themselves independent of the urban communities and their industries for supply and support}.

The proposition just made is the basic proposition of atomic bomb warfare, and it is the one which our military authorities continue consistently to overlook. They continue to speak in terms of peacetime military establishments which are simply cadres and which are expected to undergo an enormous but slow expansion \emph{after} the outbreak of hostilities.\footnote{General H. H. Arnold's \textit{Third Report to the Secretary of War} is in general outstanding for the breadth of vision it displays. Yet one finds in it statements like the following: ``An Air Force is always verging on obsolescence and, in time of peace, its size and replacement rate will always be inadequate to meet the full demands of war. Military Air Power should, therefore, be measured to a large extent by the ability of the existing Air Force to absorb in time of emergency the increase required by war together with new ideas and techniques" (page 62). Elsewhere in the same \textit{Report} (page 65) similar remarks are made about the expansion of personnel which, it is presumed, will always follow upon the outbreak of hostilities. But \emph{nowhere} in the \textit{Report} is the possibility envisaged that in a war which began with an atomic bomb attack there might be no opportunity for the expansion or even replacement either of planes or personnel. The same omission, needless to say, is discovered in practically all the pronouncements of top-ranking Army and Navy officers concerning their own plans for the future.} Therein lies the essence of what may be called ``pre-atomic thinking." The idea which must be driven home above all else is that a military establishment which is expected to fight on after the nation has undergone atomic bomb attack must be prepared to fight with the men already mobilized and with the equipment already in the arsenals. And those arsenals must be in caves in the wilderness. The cities will be vast catastrophe areas, and the normal channels of transportation and communications will be in unutterable confusion. The rural areas and the smaller towns, though perhaps not struck directly, will be in varying degrees of disorganization as a result of the collapse of the metropolitan centers with which their economies are intertwined.

Naturally, the actual degree of disorganization in both the struck and nonstruck areas will depend on the degree to which we provide beforehand against the event. A good deal can be done in the way of decentralization and reorganization of vital industries and services to avoid complete paralysis of the nation. More will be said on this subject later in the present chapter. But the idea that a nation which had undergone days or weeks of atomic bomb attack would be able to achieve a production for war purposes even remotely comparable in character and magnitude to American production in World War II simply does not make sense. The war of atomic bombs must be fought with stockpiles of arms in finished or semi-finished state. A superiority in raw materials will be about as important as a superiority in gold resources was in World War II though it was not so long ago that gold was the essential sinew of war.

All that is being presumed here is the kind of destruction which Germany actually underwent in the last year of the Second World War, only telescoped in time and considerably multiplied in magnitude. If such a presumption is held to be unduly alarmist, the burden of proof must lie in the discovery of basic errors in the argument of the preceding chapter. The essence of that argument is simply that what Germany suffered because of her inferiority in the air may now well be suffered in greater degree and in far less time, so long as atomic bombs are used, even by the power which enjoys air superiority. And while the armed forces must still prepare against the possibility that atomic bombs will not be used in another war - a situation which might permit full mobilization of the national resources in the traditional manner - they must be at least equally ready to fight a war in which no such grand mobilization is permitted.

The forces which will carry on the war after a large-scale atomic bomb attack may be divided into three main categories according to their respective functions. The first category will comprise the force reserved for the retaliatory attacks with atomic bombs; the second will have the mission of invading and occupying enemy territory; and the third will have the purpose of resisting enemy invasion and of organizing relief for devastated areas. Professional military officers will perhaps be less disturbed at the absence of any distinction between land, sea, and air forces than they will be at the sharp distinction between offensive and defensive functions in the latter two categories. In the past it was more or less the same army which was either on the offensive or the defensive, depending on its strength and on the current fortunes of war, but, for reasons which will presently be made clear, a much sharper distinction between offensive and defensive forces seems to be in prospect for the future.

The force delegated to the retaliatory attack with atomic bombs will have to be maintained in rather sharp isolation from the national community. Its functions must not be compromised in the slightest by the demands for relief of struck areas. Whether its operations are with aircraft or rockets or both, it will have to be spread over a large number of widely dispersed reservations, each of considerable area, in which the bombs and their carriers are secreted and as far as possible protected by storage underground. These reservations will of course have a completely integrated and independent system of inter-communication, and the commander of the force should have a sufficient autonomy of authority to be able to act as soon as he has established the fact that the country is being hit with atomic bombs. He should not have to wait for orders which may never be forthcoming.

Before discussing the character of the force set apart for the job of invasion, it is necessary to consider whether invasion and occupation remain indispensable to victory in an era of atomic anergy. Certain scientists have argued privately that they are not, that a nation committing aggression with atomic bombs would have so paralyzed its opponent as to make invasion wholly superfluous. It might be alleged that such an argument does not give due credit to the atomic bomb, since it neglects the necessity of preventing or minimizing retaliation in kind. If the experience with the V-1 and V-2 launching sites in World War II means anything at all, it indicates that only occupation of such sites will finally prevent their being used. Perhaps the greater destructiveness of the atomic bomb as compared with the bombs used against the V-1 and V-2 sites will make an essential difference in this respect, but it should be remembered that thousands of tons of bombs were dropped on those sites. At any rate, it is unlikely that any aggressor will be able, count upon eliminating with his initial blow the enemy's entire means of retaliation. If he knows the location of the crucial areas, he will seek to have his troops descend upon and seize them.

But even apart from the question of direct retaliation with atomic bombs, invasion to consolidate the effects of an atomic bomb attack will still be necessary. A nation which had inflicted enormous human and material damage upon another would find it intolerable to stop short of eliciting from the latter an acknowledgement of defeat implemented by a readiness to accept control. Wars, in other words, are fought to be terminated, and to be terminated decisively. Regardless of technological changes, war remains, as Clausewitz put it, an ``instrument of policy", a means of realizing a political end. To be sure, a nation may admit defeat and agree to occupation prior to actual invasion of its homeland, as the Japanese did. But it by no means follows that such will be the rule. Japan was completely defeated strategically before the atomic bombs were used against her. She not only lacked means of retaliation with that particular weapon but was without hope of being able to take aggressive action of my kind or of ameliorating her desperate military position to the slightest degree. There is no reason to suppose that a nation which had made reasonable preparations for war with atomic bombs would inevitably be in a mood to surrender after suffering the first blow.

An invasion designed to prevent large-scale retaliation with atomic bombs to any considerable degree would have to be incredibly swift and sufficiently powerful to overwhelm instantly any opposition. Moreover, it would have to descend in one fell swoop upon points scattered throughout the length and breadth of the enemy territory. The question arises whether such an operation is possible, especially across broad water barriers, against any great power which is not completely asleep and which has sizable armed forces at its disposal. It is clear that existing types of forces can be much easier reorganized to resist the kind of invasion here envisaged than to enable them to conduct so rapid an offensive.

Extreme swiftness of invasion would demand aircraft for transport and supply rather than surface vessels guarded by sea power. But the mere necessity of speed does not create the conditions under which an invasion solely by air will be successful, especially against large and well-organized forces deployed over considerable space. In the recent war the specialized air-borne infantry divisions comprised a very small proportion of the armies of each of the belligerents. The bases from which they were launched were in every case relatively close to the objective, and except at Crete their mission was always to cooperate with much larger forces approaching by land or sea. To be sure, if the air forces are relieved by the atomic bomb of the burden of devoting great numbers of aircraft to strategic bombing with ordinary bombs, they will be able to accept to a much greater extent than heretofore the task of serving as a medium of transport and supply for the infantry. But it should be noticed that the enormous extension of range for bombing purposes which the atomic bomb makes possible does not apply to the transport of troops and supplies.\footnote{See above, pp. \pageref{I-range1}-\pageref{I-range2}} For such operations distance remains a formidable barrier.

The invasion and occupation of a great country solely or even chiefly by air would be an incredibly difficult task even if one assumes a minimum of air opposition. The magnitude of the preparations necessary for such an operation might make very dubious the chance of achieving the required measure of surprise. It may well prove that the difficulty of consolidating by invasion the advantages gained through atomic bomb attack may act as an added and perhaps decisive deterrent to launching such an attack, especially since those same difficulties make retaliation all the more probable. But all hinges on the quality of preparation of the intended victim. If it has not prepared itself for atomic bomb warfare, the initial devastating attack will undoubtedly paralyze it and make its conquest easy even by a small invading force. And if it has not prepared itself for such warfare its helplessness will no doubt be sufficiently apparent before the event as to invite aggression.

It is obvious that the force set apart for invasion or counter-invasion purposes will have to be relatively small, completely professional, and trained to the uttermost. But there must also be a very large force ready to resist and defeat invasion by the enemy. Here is the place for the citizen army, though it too must be comprised of trained men. There will be no time for training once the atomic bomb is used. Perhaps the old ideal of the ``minute-man" with his musket over his fireplace will be resurrected, in suitably modernized form. In any case, provision must be made for instant mobilization of trained reserves, for a maximum decentralization of arms and supply depots and of tactical authority, and for flexibility of operation. The trend towards greater mobility in land forces will have to be enormously accelerated, and strategic concentrations will have to to be achieved in ways which avoid a high spatial density of military forces. And it must be again repeated, the arms, supplies, and vehicles of transportation to be depended upon are those which are \emph{stockpiled} in as secure a manner as possible.

At this point it should be clear how drastic are the changes in character, equipment, and outlook which the traditional armed forces must undergo if they are to act as real deterrents to aggression in an age of atomic bombs. Whether or not the ideas presented above are entirely valid, they may perhaps stimulate those to whom our military security is entrusted to a more rigorous and better informed kind of analysis which will reach sounder conclusions.

In the above discussion the reader will no doubt observe the absence of any considerable role for the Navy. And it is indisputable that the traditional concepts of military security which this country has developed over the last fifty years - in which the Navy was quite correctly avowed to be our ``first line of defense" - seem due for revision, or at least for reconsideration.

For in the main sea power has throughout history proved decisive only when it was applied and exploited over a period of considerable time, and in atomic bomb warfare that time may well be lacking. Where wars are destined to be short, superior sea power may prove wholly useless. The French naval superiority over Prussia in 1870 did not prevent the collapse of the French armies in a few months, nor did Anglo-French naval superiority in 1940 prevent an even quicker conquest of France - one which might very well have ended the war.

World War II was in fact destined to prove the conflict in which sea power reached the culmination of its influence on history. The greatest of air wars and the one which saw the most titanic battles of all time on land was also the greatest of naval wars. It could hardly have been otherwise in a war which was truly global, where the pooling of resources of the great allies depended upon their ability to traverse the highways of the seas and where American men and materials played a decisive part in remote theaters which could be reached with the requisite burdens only by ships. That period of greatest influence of sea power coincided with the emergence of the United States as the unrivaled sea power of the world. Yet in many respects all this mighty power seems at the moment of its greatest glory to have become redundant.

Yet certain vital tasks may remain for fleets to perform even in a war of atomic bombs. One function which a superior fleet serves at every moment of its existence - and which therefore requires no time for its application - is the defense of coasts against sea-borne invasion. Only since the surrender of Germany, which made available to us the observations of members of the German High Command, has the public been made aware of something which had previously been obvious only to close students of the war - that it was the Royal Navy even more than the R.A.F. which kept Hitler from leaping across the Channel in 1940. The R.A.F. was too inferior to the Luftwaffe to count for much in itself, and was important largely as a means of protecting the ships which the British would have interposed against any invasion attempt.

We have noticed that if swiftness were essential to the execution of any invasion plan, the invader would be obliged to depend mainly if not exclusively on transport by air. But we also observed that the difficulties in the way of such an enterprise might be such as to make it quite impossible of achievement. For the overseas movement of armies of any size and especially of their larger arms and supplies, sea-borne transportation proved quite indispensable even in an era when gigantic air forces had been built up by fully mobilized countries over four years of war. The difference in weight-carrying capacity between ships and planes is altogether too great to permit us to expect that it will become militarily unimportant in fifty years or more.\footnote{See Bernard Brodie, \textit{A Guide to Naval Strategy} (Princeton, 3rd ed.) p. 215.} A force which is able to keep the enemy from using the seas is bound to remain for a long time an enormously important defense against overseas invasion.

However, the defense of coasts against sea-borne invasion is something which powerful and superior air forces are also able to carry out, though perhaps somewhat less reliably. If that were the sole function remaining to the Navy, the maintenance of huge fleets would hardly be justified. One must consider also the possible offensive value of a fleet which has atomic bombs at its disposal.

It was argued in the previous chapter that the atomic bomb enormously extends the effective range of bombing aircraft, and that even today the cities of every great power are inside the effective bombing range of planes based on the territories of any other great power. The future development of aircraft will no doubt make bombing at six and seven thousand miles range even more feasible than it is today, and the tendency towards even higher cruising altitudes will ultimately bring planes above the levels where weather hazards are an important barrier to long flights. The ability to bring one's planes relatively close to the target before launching them, as naval carrier forces are able to do, must certainly diminish in military importance. But it will not wholly cease to be important, even for atomic bombs; and if the emphasis in vehicles is shifted from aircraft to long-range rockets, there will again be an enormous advantage in having one's missiles close to the target.

Even more important, perhaps, is the fact that a fleet at sea is not easily located and even less easily destroyed. The ability to retaliate if attacked is certainly enhanced by having a bomb-launching base which cannot be plotted on a map. A fleet armed with atomic bombs which had disappeared into the vastness of the seas during a crisis would be just one additional element to give pause to an aggressor. It must, however, be again repeated that the possession of such a fleet or of advanced bases \emph{will not be essential} to the execution of bombing missions at extreme ranges.

If there should be a war in which atomic bombs were not used - a possibility which must always be provided against - the fleet would retain all the functions it has ever exercised. We know also that there are certain policing obligations entailed in various American continents, especially that of the United Nations Organization. The idea of using atomic bombs for such policing operations, as some have advocated, is not only callous in the extreme but stupid. Even general bombing with ordinary bombs is the worst possible way to coerce states of relatively low military power, for it combines the maximum of indiscriminate destruction with the minimum of direct control.\footnote{There has been a good deal of confusion between automaticity and immediacy in the execution of sanctions. Those who stress the importance of bringing military pressure to bear \emph{at once} in the case of aggression are as a rule really less concerned with having sanctions imposed quickly than they are with having them appear certain. To be sure, the atomic bomb gives the necessity for quickness of military response a wholly new meaning; but in the kinds of aggression with which the U.N.O. is now set up to deal, atomic bombs are not likely to be important for a very long time.}

At any rate, if the United States retains a strong navy, as it no doubt will, we should insist upon that navy retaining the maximum flexibility and adaptability to new conditions. The public can assist in this process by examining critically any effort of the service to freeze naval armaments at high quantitative levels, for there is nothing more deadening to technological progress especially in the Navy than the maintenance in active or reserve commission a number of ships far exceeding any current needs. It is not primarily a question of how much money is spent or how much man power is absorbed but rather of how efficiently money and man power are being utilized. Money spent on keeping in commission ships built for the last war is money which might be devoted to additional research and experimentation, and existing ships discourage new construction. For that matter, money spent on maintaining a huge navy is perhaps money taken from other services and other instruments of defense which may be of far greater relative importance in the early stages of a future crisis than they have been in the past.

\vspace{12pt}

\noindent\textbf{The Dispersion of Cities as a Defense Against the Bomb}

\vspace{10pt}

We have seen that the atomic bomb drastically alters the significance of distance \emph{between} rival powers. It also raises to the first order of importance as a factor of power the precise spatial arrangement of industry and population \emph{within} each country. The enormous concentration of power in the individual bomb, irreducible below a certain high limit except through deliberate and purposeless wastage of efficiency, is such as to demand for the full realization of that power targets in which the enemy's basic strength is comparably concentrated. Thus, the city is a made-to-order target, and the degree of urbanization of a country furnishes a rough index of its relative vulnerability to the atomic bomb.

And since a single properly-aimed bomb can destroy a city of 100,000 about as effectively as it can one of 25,000, it is obviously an advantage to the attacker if the units of 25,000 are combined into units of 100,000. Moreover, a city is after all a fairly integrated community in terms of vital services and transportation. If half to two-thirds of its area is obliterated, one may count on it that the rest of the city will, under prevailing conditions, be effectively prostrated. Thus, the more the population and industry of a state are concentrated into urban areas and the larger individually those concentrations become, the fewer are the atomic bombs necessary to effect their destruction.\footnote{In this respect the atomic bomb differs markedly from the TNT bomb, due to the much smaller radius of destruction of the latter. The amount of destruction the TNT bomb accomplishes depends not on what is in the general locality but on what is in the immediate proximity of the burst. A factory of given size requires a given number of bombs to destroy it regardless of the size of the city in which it is situated. To be sure, the ``misses" count for more in a large city, but from the point of view of the defender there are certain compensating advantages in having the objects to be defended gathered in large concentrations. It makes a good deal easier the effective deployment of fighter patrols and antiaircraft guns. But the latter advantage does not count for much in the case of atomic bombs, since, as argued in the previous chapter, it is practically hopeless to expect fighter planes and antiaircraft guns to stop atomic bomb attack so completely as to save the city.}

In 1940 there were in the United States five cities with 1,000,000 or more inhabitants (one of which, Los Angeles, is spread out over more than 400 square miles), nine cities between 500,000 and 1,000,000, twenty-three cities between 250,000 and 500,000, fifty-five between 100,000 and 250,000 and one hundred and seven between 50,000 and 100,000 population. Thus, there were ninety-two cities with a population of 100,000 and over, and these contained approximately 29 per cent of our total population. Reaching down to the level of 50,000 or more, the number of cities is increased to 199 and the population contained in them is increased to some 34 per cent. Naturally, the proportion of the nation's factories contained in those 199 cities is far greater than the proportion of the population.

This is a considerably higher ratio of urban to non-urban population than is to be found in any other great power except Great Britain. Regardless of what international measures are undertaken to cope with the atomic bomb menace, the United States cannot afford to remain complacent about it. This measure of vulnerability, to be sure, must be qualified by a host of other considerations, such as the architectural character of the cities,\footnote{The difference between American and Japanese cities in vulnerability to bombing attack has unquestionably been exaggerated. Most commentators who stress the difference forget the many square miles of predominantly wooden frame houses to be found in almost any American city. And those who were impressed with the pictures of ferro-concrete buildings standing relatively intact in the midst of otherwise total devastation at Hiroshima and Nagasaki will not be comforted by Dr. Philip Morrison's testimony before the MacMahon Committee on December 6, 1945. Dr. Morrison, who inspected both cities, testified that the interiors of those buildings were completely destroyed and the people in them killed. Brick buildings, he pointed out, and even steel-frame buildings with brick walls proved extremely vulnerable. ``Of those people within a thousand yards of the blast", he added, ``about one in every house or two escaped death from blast or burn. But they died anyway from the effects of the rays emitted at the instant of explosion." He expressed himself as convinced that an American city similarly bombed ``would be as badly damaged as a Japanese city, though it would look less wrecked from the air."

Perhaps Dr. Morrison is exaggerating in the opposite direction. Obviously there must be a considerable difference among structures in their capacity to withstand blast from atomic bombs and to shelter the people within then. But that difference is likely to make itself felt mostly in the peripheral portions of a blasted area. Within a radius of one mile from the center of burst it is not likely to be of consequence.} the manner in which they are individually laid out, and above all the degree of interdependence of industry and services between different parts of the individual city, between the city and its hinterland, and between the different urban areas Each city is, together with its hinterland, an economic and social organism, with a character somewhat distinct from other comparable organisms.

A number of students have been busily at work evolving plans for the dispersal of our cities and the resettlement of our population and industries in a manner calculated to reduce the number of casualties and the amount of physical destruction that a given number of atomic bombs can cause. In their most drastic form these plans, many of which will shortly reach the public eye, involve the redistribution of our urban concentrations into ``linear" or ``cellular" cities.

The linear or ``ribbon" city is one which is very much longer than it is wide, with its industries and services as well as population distributed along its entire length. Of two cities occupying nine square miles, the one which was one mile wide and nine long would clearly suffer less destruction from one atomic bomb, however perfectly aimed, than the one which was three miles square. "The principle of the cellular city, on the other hand, would be realized if a city of the same nine-square-miles size were dispersed into nine units of about one square mile each and situated in such a pattern that each unit was three to five miles distant from another.

Such ``planning" seems to this writer to show a singular lack of appreciation of the forces which have given birth to our cities and caused them to expand and multiply. There are always important geographic and economic reasons for the birth and growth of a city and profound political and social resistance to interference with the results of ``natural" growth. Cities like New York and Chicago are not going to dissolve themselves by direction from the government, even if they could find areas to dissolve themselves into. As a linear city New York would be as long as the state of Pennsylvania, and would certainly have no organic meaning as a city. ``Solutions" like these are not only politically and socially unrealistic but physically impossible.

Nor does it seem that the military benefits would be at all commensurate with the cost, even if the programs were physically possible and politically feasible. We have no way of estimating the absolute limit to the number of bombs which will be available to an attacker, but we know that unless production of atomic bombs is drastically limited or completely suppressed by international agreement, the number available in the world will progress far more rapidly and involve infinitely less cost of production and use than any concurrent dissolution or realignment of cities designed to offset that multiplication. If a city three miles square can be largely destroyed by one well aimed bomb, it will require only three well spaced bombs to destroy utterly a city nine miles long and one mile wide. And the effort required in producing and delivering the two extra bombs is infinitesimal compared to that involved in converting a square city into a linear one.

Unquestionably an invulnerable home front is beyond price, but there is no hope of gaining such a thing in any case. What the city-dispersion-planners are advocating is a colossal effort and expenditure (estimated by some of them to amount to 300 billions of dollars) and a ruthless suppression of the inevitable resistance to such dispersion in order to achieve what is at best a marginal diminution of vulnerability. No such program has the slightest chance of being accepted.

However, it is clear that the United States can be made a good deal less vulnerable to atomic bomb attack than it is at present, that such reduction can be made great enough to count as a deterrent in the calculations of future aggressors, and that it can be done at immeasurably less economic and social cost and in a manner which will arouse far less resistance than any of the drastic solutions described above.

But first we must make clear in our minds what our ends are. Our first purpose, clearly, is to reduce the likelihood that a sudden attack upon us will be so paralyzing in its effects as to rob us of all chance of effective resistance. And we are interested in sustaining our power to retaliate primarily to make the prospect of aggression much less attractive to the aggressor. In other words, we wish to reduce our vulnerability in order to reduce the chances of our being hit at all. Secondly, we wish to reduce the number of casualties and of material damage which will result from an attack upon us of any given level of intensity.

These two ends are of course intimately interrelated, but they are also to a degree distinguishable. And it is necessary to pursue that distinction. We should notice also that while most industries are ultimately convertible or applicable to the prosecution of war, it is possible to distinguish between industries in the degree of their immediate indispensability for war purposes. Finally, while industries attract population and vice versa, modern means of transportation make possible a locational flexibility between an industry and those people who service it and whom it serves.

Thus it would seem that the first step in reducing our national vulnerability is to catalog the industries especially and immediately necessary to atomic bomb warfare - a relatively small proportion of the total - and to move them out of our cities entirely. Where those industries utilize massive plants, those plants should as far as possible be broken up into smaller units. Involved in such a movement would be the labor forces which directly service those industries. The great mass of remaining industries can be left where they are within the cities, but the population which remains with them can be encouraged, through the further development of suburban building, to spread over a greater amount of space. Whole areas deserving to be condemned in any case could be converted into public parks or even airfields. The important element in reducing casualties is after all not the shape of the individual city but the spatial density of population within it.

Furthermore, the systems providing essential services, such as those supplying or distributing food, fuel, water, communications, and medical care, could and should be rearranged geographically. Medical services, for example, tend to be concentrated not merely within cities but in particular sections of those cities. The conception which might govern the relocation of services within the cities is that which has long been familiar in warship design-\emph{compartmentation}. And obviously where essential services for large rural areas are unnecessarily concentrated in cities, they should be moved out of them. That situation pertains especially to communications.

It would be desirable also to initiate a series of tests on the resistance of various kinds of structures to atomic bomb blast. It might be found that one type of structure has far greater resistance than another without being correspondingly more costly. If so, it would behoove the government to encourage that kind of construction in new building. Over a long period of years, the gain in resistance to attack of our urban areas might be considerable, and the costs involved would be marginal.

So far as safeguarding the lives of urban populations is concerned, the above suggestions are meaningful only for the initial stages of an attack. They would permit a larger number to survive the initial attacks and thereby to engage in that exodus from the cities by which alone their lives can be safeguarded. And the preparation for such an exodus would involve a vast program for the construction of temporary shelter in the countryside and the planting of emergency stores of food. What we would then have in effect is the dispersal not of cities but of air-raid shelters.

The writer is here presenting merely some general principles which might be
considered in any plan for reducing our general vulnerability. Obviously, the
actual content of such a plan would have to be derived from the findings of intensive
study by experts in a rather large number of fields. It is imperative,
however, that such a study be got under way at once. The country is about to
launch into a great construction program, both for dwellings and for expanding
industries. New sources of power are to be created by new dams. The opportunities
thus afforded for ``vulnerability control" are tremendous, and should not be
permitted to slip away - at least not without intensive study of their feasibility.

\noindent\hfil\rule{0.4\textwidth}{.4pt}\hfil

\vspace{4pt}

Those who have been predicting attacks of 15,000 atomic bombs and upward will no doubt look with jaundiced eye upon these speculations. For they will say that a country so struck will not merely be overwhelmed but for all practical purposes will vanish. Those areas not directly struck will be covered with clouds of radioactive dust under which all living beings will perish.

No doubt there is a possibility that an initial attack can be so overwhelming as to void all possibility of resistance or retaliation, regardless of the precautions taken in the target state. Not \emph{all} eventualities can be provided against. But preparation to launch such an attack would have to be on so gigantic a scale as to eliminate all chances of surprise. Moreover, while there is perhaps little solace in the thought that the lethal effect of radioactivity is generally considerably delayed, the idea will not be lost on the aggressor. The more horrible the results of attack, the more he will be deterred by even a marginal chance of retaliation.

Finally, one can scarcely assume that the world will remain either long ignorant of or acquiescent in the accumulations of such vast stockpiles of atomic bombs. International organization may seem at the moment pitifully inadequate to cope with the problem of controlling bomb production, but a runaway competition in such production would certainly bring new forces into the picture. In this chapter and in the preceding one, the writer has been under no illusions concerning the value of a purely military solution.

Concern with the efficiency of the national defenses is obviously inadequate in itself as an approach to the problem of the atomic bomb. In so far as such concern prevails over the more fundamental consideration of eliminating war or at least of reducing the chance of its recurrence, it clearly defeats its purpose. That has perhaps always been true, but it is a truth which is less escapable today than ever before. Nations can still save themselves by their own armed strength from subjugation, but not from a destruction so colossal as to involve complete ruin. Nevertheless, it also remains true that a nation which is as well girded for its own defense as is reasonably possible is not a tempting target to an aggressor. Such a nation is therefore better able to pursue actively that progressive improvement in world affairs by which alone it finds its true security.
